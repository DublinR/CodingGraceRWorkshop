%=====================================================================================================%
[verbatim]
The dplyr package, plotrix (for the plot at the end), and the FSAdata package (for the data file) must be loaded.
<pre>
<code>
library(dplyr)
library(FSAdata)
library(plotrix)
</code>
</pre>
<p>


%=====================================================================================================%
[verbatim]
The RuffeSLRH92 data frame is then loaded.

<pre>
<code>
data(RuffeSLRH92)
str(RuffeSLRH92)
</code>
</pre>
<p>


%=======================================================================================================%
[verbatim]
frametitle{dplyr: Select (columns) Example}
Columns can be selected from a data.frame with select(), given the original data.frame as the first argument and the variables to select, or include, as further arguments. The following creates a data.frame without the fish.id, species, day and year variables (they are not very useful in this context and will make the output further below easier to read).
<pre>
<code>
RuffeSLRH92 <- select(RuffeSLRH92,-fish.id,-species,-day,-year)
head(RuffeSLRH92)
</code>
</pre>
<p>


%=======================================================================================================%
[verbatim]

The following creates a data.frame of just the length and weight variables.

<pre>
<code>
ruffeLW <- select(RuffeSLRH92,length,weight)
head(ruffeLW)

</code>
</pre>
<p>


%=======================================================================================================%
[verbatim]

         * 
The dplyr package contains a variety of helpers for selecting. 
         * As one example, the following will select all variables that contains the letter “l”.

<pre>
<code>
ruffeL <- select(RuffeSLRH92,contains("l"))
str(ruffeL)
</code>
</pre>
<p>


%=======================================================================================================%

frametitle{Filtering Example}

         * The filter() function can be used similarly to subset() to select a set of rows from an original data.frame according to some 
conditioning statement. 
         * As with subset(), filter() returns an object that maintains a list of the original levels whether those levels exist in the new data.frame or not.
          * Use droplevels() to restrict the levels to only those that exist in the data.frame. 
         * The example below finds just the males from the original data.frame.

<pre>
<code>male <- filter(RuffeSLRH92,sex=="male")
xtabs(~sex,data=male)
</code>
</pre>
<p>


%=======================================================================================================%


<pre>
<code>
male <- droplevels(male)
xtabs(~sex,data=male)
## sex
## male 
##  201
</code>
</pre>
<p>

%=======================================================================================================%

Multiple conditioning statements can be strung together as additional arguments to filter(). The example below finds males that are also ripe.

<pre>
<code>
maleripe <- filter(RuffeSLRH92,sex=="male",maturity=="ripe")
xtabs(~sex+maturity,data=maleripe)
</code>
</pre>
<p>


