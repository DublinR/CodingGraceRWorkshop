

% - http://cran.rstudio.com/web/packages/dplyr/vignettes/introduction.html

<p>
###        * {Single table verbs}

Dplyr aims to provide a function for each basic verb of data manipulating:
begin{ }
	         * texttt{ filter() } (and texttt{  slice() })
	         * texttt{ arrange() }
	         * texttt{ select() } (and texttt{  rename() })
	         * texttt{ distinct() }
	         * texttt{ mutate() } (and texttt{  transmute() })
	         * texttt{ summarise() }
	         * texttt{ sample_n() } and texttt{  sample_frac() }
end{ }

% If you've used plyr before, many of these will be familar.

%========================================================================%

<p>
###        * {Grouped operations}

These verbs are useful, but they become really powerful when you combine them with the idea of “group by”, repeating the operation individually on groups of observations within the dataset. In dplyr, you use the group_by()} function to describe how to break a dataset down into groups of rows. You can then use the resulting object in exactly the same functions as above; they'll automatically work “by group” when the input is a grouped.

The verbs are affected by grouping as follows:
begin{ }
	         * grouped texttt{ select()} is the same ungrouped texttt{ select()}, excepted that retains grouping variables are always retained. 
	
	         * 
	grouped texttt{  arrange()} orders first by grouping variables
	
	         * 
	texttt{ mutate()} and texttt{ filter()} are most useful in conjunction with window functions (like texttt{ rank()}, or texttt{ min(x) == x), and are described in detail in vignette("window-function").
		
		         * 
		texttt{ sample_n()} and texttt{ sample_frac()} sample the specified number/fraction of rows in each group.
		
		         * 
		texttt{ slice()} extracts rows within each group.
		
		         * 
		texttt{ summarise()} is easy to understand and very useful, and is described in more detail below.
	end{ }
	
	%=====================================%
	
