
%----------------------------------------------------------------%
\section{Matrices}
\subsection{Creating a matrix}

Matrices can be created using the \texttt{matrix()} command. The arguments to be supplied are 
\begin{enumerate}
	* vector of values to be entered
	*  Dimensions of the matrix, specifying either the numbers of rows or columns.
\end{enumerate}

Additionally you can specify if the values are to be allocated by row or column. By default they are allocated by column.

<pre>
	<code>
	Vec1=c(1,4,5,6,4,5,5,7,9)		# 9 elements
	A=matrix(Vec,nrow=3)		#3 by 3 matrix. Values assigned by column.
	A
	B= matrix(c(1,6,7,0.6,0.5,0.3,1,2,1),ncol=3,byrow =TRUE)
	B				          #3 by 3 matrix. Values assigned by row.
	</code>
</pre>



If you have assigned values by column, but require that they are assigned by row, you can use the transpose function \texttt{t()}.
<pre>
	<code>
	t(A)	# Transpose
	A=t(A)	
	</code>
</pre>

Another method of creating a matrix is to “bind” a number of vectors together, either by row or by column. The commands are \texttt{rbind() }.and \texttt{cbind()} respectively.


<pre>
	<code>
	x1 =c(1,2) ; x2 = c(3,6)
	rbind(x1,x2)
	cbind(x1,x2)
	</code>
</pre>


\section{Matrices}
\subsection{Creating a matrix}
Matrices can be created using the \texttt{matrix()} command. The arguments to be supplied are 1) vector of values to be entered
2) Dimensions of the matrix, specifying either the numbers of rows or columns.

Additionally you can specify if the values are to be allocated by row or column. By default they are allocated by column.
<code>
Vec1=c(1,4,5,6,4,5,5,7,9)		# 9 elements
A=matrix(Vec,nrow=3)		#3 by 3 matrix. Values assigned by column.
A
B= matrix(c(1,6,7,0.6,0.5,0.3,1,2,1),ncol=3,byrow =TRUE)
B				          #3 by 3 matrix. Values assigned by row.
</code>	
If you have assigned values by column, but require that they are assigned by row, you can use the transpose function
<code>
t().
t(A)				# Transpose
A=t(A)	
</code>

Another methods of creating a matrix is to "bind" a number of vectors together, either by row or by column. The commands are rbind() and cbind() respectively.
<code>
x1 =c(1,2) ; x2 = c(3,6)
rbind(x1,x2)
cbind(x1,x2)
</code>




Particular rows and columns can be accessed by specifying the row number or column number, leaving the other value blank.
<code>
A[1,]	  # access first row of A
B[,2]   # access first column of B
</code>
Addition and subtractions
For matrices, addition and subtraction works on an element- wise basis. The first elements of the respective matrices are added, and so on.
A+B
A-B
