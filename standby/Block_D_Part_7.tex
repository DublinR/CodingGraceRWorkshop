
### {The R Programming Language}

The R Programming Language is a statistical , data analysis , etc

R is a free software environment for statistical computing and graphics.

<p>
### {Writing R scripts}
Editing your R script ``R Editor".

On the menu of the R console, click on ‘file’.
Select ‘open script’ or ‘new script’ as appropriate.
Navigate to your working directory and select your ‘texttt{.R}’ file
A new dialogue box ``texttt{the R editor}" will open up.
Input or select code you wish to compile.
To compile this code, highlight it. Click the ‘edit button’ on the menu.
Select either ``Run Line" or ``Run Selection or All".
Your code should now compile.
To save your code, clink on ``file" and then ``texttt{save as}".
Save the file with the ``texttt{.R}" extension to your working directory.


<p>
### {Vector types}
texttt{R} operates on named data structures. The simplest such structure is the
vector, which is a single entity consisting of an ordered collection of
Numbers or characters.


Numeric vectors
Character vectors
Logical vectors
(also complex number vectors and colour vectors)


To create a vector, use the assignment operator and the concatenate function.
For numeric vectors, the values are simply numbers.

<code>
># week8.r
>NumVec<-c(10.4,5.6,3.1,6.4)
</code>

Alternatively we can use the texttt{assign()} command

For character vectors, the values are simply characters, specified with
quotation marks.A logical vectors is a vector whose elements are TRUE, FALSE or NA

<code>
>CharVec<-c(``blue", ``green", ``yellow")
>LogVec<-c(TRUE, FALSE)
</code>

<p>






%----------------------------------------------------------------%
<p>
### {Some Useful Operations}
<p>
#### {Sampling}

The texttt{sample()} function.

<p>
#### {Set Theory Operations}

<p>
#### {Controlling Precision and Integerization}
<pre>
<code>
pi
round(pi,3)
round(pi,2)
floor(pi)
ceiling(pi)
</code>
</pre>
<p>

%----------------------------------------------------------------%

<p>
### {Important Introductory Topics}
<p>
#### {The texttt{head()} and texttt{tail()} functions}
<p>
#### {Randomly Generated Numbers}
With $a$ and $b$ as the lower and upper bound of the continous uniform distribution.
[X sim U(a,b)]
<p>
#### {The texttt{as} and texttt{is} families of functions}
<p>
#### {The texttt{apply} family of functions}
<p>
#### {Writing your own function}


%----------------------------------------------------------------%
<p>
### {Lists and Data Frames}
<p>
#### {Lists}
<p>
#### {Named Components}
<p>
#### {Data Frames}
<pre>
<code>
framename = data.frame()
</code>
</pre>
<p>
%----------------------------------------------------------------%
<p>
### {Important Graphical Procedures}
begin{enumerate}
Histograms
Box-plots
Scatter-plots
end{enumerate}


%----------------------------------------------------------------%
