
%=====================================================================================================================%
\begin{frame}

\frametitle{Introduction to R}

R is the product of an active movement among statisticians for a powerful, programmable, portable, and open computing environment, applicable to the most complex and sophsticated problems, as well as“routine”analysis.
 
There are no restrictions on access or use.


Statisticians have implemented hundreds of specialised statistical procedures for a wide variety of applications as contributed packages, which are also freely-available and which integrate directly into R.

 
R has data handling and storage facilities, a suite of operators for calculations on arrays in particular matrices. A large coherent integrated collection of intermediate tools for data analysis
A large selection of demonstration datasets used in the illustration of many statistical methods.
Graphical facilities for data analysis and display either directly.at the computes or on hardcopy.
 
\end{frame}
%=====================================================================================================================%
\begin{frame}
R is a flexible language that is object oriented and thus allows the manipulation of a complex data structures in a condensed and efficient manner.


R's graphical abilities are also remarkable with possible interfacing with text processors such as Latex with the package sweave.

R offers the addtional advantage of being a free and opensource system under the GNU general public licence.

R is primarily a statistical language. R can be installed free of charge from www.r-project.org
\end{frame}
%=====================================================================================================================%
\begin{frame}
An online guide "An Introduction to R" can be access by typing help.start() at the command prompt to access this.

R is a statistical Environment for statistical computing and graphics, which is available for windows, Unix and Mac OS platforms.

R is maintained and distributed by an international team of statisticians and computers scientists.
R is one of the major tools used in statistical research and in applications of statistics research.

R is an open-source (GPL) statistical environment modeled after S and S-Plus. The S language was developed in the late 1980s at AT&T labs. The R project was started by Robert Gentleman and Ross Ihaka of the Statistics Department of the University of Auckland in 1995. It has quickly gained a widespread audience. 
\end{frame}
%=====================================================================================================================%
\begin{frame}
R is currently maintained by the R core-development team, a hard-working, international team of volunteer developers. 

The R project web page ( http://www.r-project.org ) is the main site for information on R. At this site are directions for obtaining the software, accompanying packages and other sources of documentation.

\end{frame}

%=====================================================================================================================%
\begin{frame}

For example, to load the fda package:

> library(fda)

One important thing to note is that if you terminate your session and start a new session with the saved workspace, you must load  the packages again



install.packages("evir")
 
To get out of R, just type: q(). 

\end{frame}

\begin{frame}

Section 10: Simulation
Simulation Study : Random Walks
Simulation Study: Distribution of pairwise maxima and minima
Simulation Study : Gamblers Ruins
Simulation Study: Probability of Gambler Ruin

\end{frame}
%=====================================================================================================================%
\begin{frame}

Section 1: Basic R commands and Functions
Installing R on your computer
R can be easily downloaded from the Comprehenive R Archive Network (CRAN) website.
\end{frame}
%=====================================================================================================================%
\begin{frame}

Editing your Data

x=c(0 ,5)     	      # create a vector x
data.entry(x)  	   # edit the values using spreadsheet interface.
x  	                     # print to screen
x=edit(x)	          # the 'edit' function to call the script editor
x  	                     # print to screen

\end{frame}

%=====================================================================================================================%
\begin{frame}
Manipulating Characters
> nchar("oscar")
[1] 5

Objects
During an R session, objects are created and stored by name. The command "ls()" displays all currently-stored objects (workspace). Objects can be removed using the "rm()" function.

\end{frame}
%=====================================================================================================================%
\begin{frame}

ls()
rm(x, a, temp, wt.males)
rm(list=ls())								#removes all of the objects in the workspace.


At the end of each R session, you are prompted to save your workspace. If you click Yes, all objects are written to the ".RData" file. 
When R is re-started, it reloads the workspace from this file and the command history stored in ".Rhistory" is also reloaded.

%==================================================================================================================%
