

%------------------------------------------

\begin{slide}{Bivariate Data}

\begin{itemize}
* A dataset with two variables contains what is called bivariate data.

* Each pair of values describes the same instance. For example, the first value of both variables describe the same individual or item.
    
* It is often of interest to describe the relationship between two variables.
\end{itemize}
<p>
%------------------------------------------

\begin{slide}{Data used today}

\begin{verbatim}
>x<-c(5.98, 8.80, 6.89, 8.49, 8.48, 7.47,
    7.97,5.94, 7.32, 6.64, 6.94, 3.51)
>
>
>y<-c(5.56, 7.80, 6.13, 8.15, 7.95, 7.87,
    8.03, 5.67, 7.11, 6.65, 7.02, 3.88)

\end{verbatim}
It is important the both data sets have same length.
\begin{verbatim}
>length(x)
>length(y)

\end{verbatim}
<p>

%------------------------------------------
\begin{slide}{Regression Models}

\begin{itemize}
* Simple linear regression (today).
    \begin{itemize}
    * Estimates for slope and intercept.
    * Accessing those estimates.
    * Inference on those estimates.
    \end{itemize}
* Multiple linear regression (later).
    \begin{itemize}
    * Model selection
    \end{itemize}
* Non-linear regression (later).
* Diagnostics (later).
\end{itemize}
<p>
%------------------------------------------
\begin{slide}{Simple linear regression}
\begin{itemize}
* Simple linear regression is used to describe the relationship
between two variables `x' and `y'. * For example, you may want to describe the
relationship between age and blood pressure or the relationship
between scores in a midterm exam and scores in the final exam,
etc.
* `x' is the independent (i.e. predictor) variable
* `y' is the dependent (i.e. response) variable.
* Necessarily both x and y should be of equal length.

* One of the first steps in a regression analysis is to determine if any
kind of relationship exists between `x' and `y'.



\end{itemize}
<p>
%------------------------------------------
\begin{slide}{Simple linear regression}
\begin{itemize}

* A scatterplot can created and can initially be used to get an idea
about the nature of the relationship between the variables, e.g. if
the relationship is linear, curvilinear, or no relationship exists.

* We can see from a scatterplot that there is a linear relationship
between x and y.

* Simple linear regression is only useful when there is evidences of a linear relationship.
In other cases, such as quadratic relationships, other types of regression may be useful.

\end{itemize}
<p>
%------------------------------------------

\begin{slide}{Scatterplot}
\begin{itemize}
* To make a simple scatterplot of the bivariate data, we simply use the
``plot()" command. * The independent variable (the variable to go along the x-axis) is always specified first.
\begin{verbatim}
>plot(x,y)
\end{verbatim}
* In future classes, we will look at how to improve and enhance scatter-plots, by controlling graphical parameters.
\end{itemize}
<p>

%------------------------------------------
\begin{slide}{Correlation}
\begin{itemize}

* The Pearson product-moment correlation coefficient is a measure of the strength of the linear relationship between two variables. *  It is referred to as Pearson's correlation or simply as the correlation coefficient. * If the relationship between the variables is not linear, then the correlation coefficient does not adequately represent the strength of the relationship between the variables.
\end{itemize}
<p>

%------------------------------------------



\begin{slide}{Correlation}
\begin{itemize}
* To compute the Pearson correlation coefficient (``r"), we use the ``cor()" command.
\begin{verbatim}
> cor(x,y)
[1] 0.9581898
\end{verbatim}
* The coefficient should be between $-1$ and $1$.
* Recall, the higher the absolute value of the correlation coefficient, the stronger the linear relationship.
* A positive correlation coefficient indicates a positive relationship.
* A negative correlation coefficient indicates a negative (inverse) relationship.
\end{itemize}
<p>

\begin{slide}{Correlation and Covariance}
\begin{itemize}
* Other types of correlation coefficient are possible, such as the Spearman coefficient, and the Kendall Tau coefficient.
* To specify one of these methods, add the argument to the command, as shown below.
 \begin{verbatim}
> cor(x,y,method="kendall")
[1] 0.7878788
> cor(x,y,method="spearman")
[1] 0.909091
\end{verbatim}

* To compute the covariance, we use the ``cov()" command.
\begin{verbatim}
> cov(x,y)
[1] 1.824429
\end{verbatim}

\end{itemize}
<p>
%------------------------------------------

\begin{slide}{Simple Linear Regression}



\begin{itemize}
* Basic regression model :
$y=\beta_{0} + \beta_{1}x + \epsilon$

\item
The intercept $\beta_{0}$ describes the point at which the line intersects
the y axis
* The slope $\beta_{1}$ describes the change in `y'  for every unit increase in `x'.

* From the data set, we determine the regression coefficients, i.e estimates for slope and intercept.

\begin{itemize}
\item
 $\hat{\beta}_{0}$ : the intercept estimate.
* $\hat{\beta}_{1}$ : the slope estimate.
\end{itemize}
* Fitted model : $\hat{y}=\hat{\beta}_{0} + \hat{\beta}_{1}x $
\end{itemize}


<p>

%------------------------------------------

\begin{slide}{The lm() command.}
\begin{itemize}
* The command lm() is used to fit linear models. * Firstly the response variable is specified, then the predictor variable. * The tilde sign is used to denote the dependent relationship (i.e. y depends on x).
* The regression coefficients are then determined.
\end{itemize}
\begin{verbatim}
> lm(y~x)

Call:
lm(formula = y ~ x)

Coefficients:
(Intercept)            x
     0.7812       0.8581
\end{verbatim}

<p>
%------------------------------------------

\begin{slide}{Simple linear regression}
\begin{itemize} * A more detailed model (i.e. more than just the coefficients) is generated in the form of a data object. * We can give a name to the model, and view all of the results of the calculation, including \begin {itemize} * The regression coefficients \item
The fitted $\hat{y}$ values (i.e. the estimated `y' values for the x date set) * The residuals (i.e. the differences between  the estimated `y' values and the observed `y' values). \end{itemize}\item
In common with all data structures we can use the names() function and `\$' to access components.\end{itemize}
<p>
%------------------------------------------

\begin{slide}{Simple linear regression}

\begin{verbatim}
> fit1 = lm(y~x)
>
> names(fit1)
 [1] "coefficients"  "residuals"
 [3] "effects"       "rank"
 [5] "fitted.values" "assign"
 [7] "qr"            "df.residual"
 [9] "xlevels"       "call"
[11] "terms"         "model"
>
>summary(fit1)
\end{verbatim}

<p>

%------------------------------------------

\begin{slide}{Simple linear regression}
We can access components using the `\$'.
\begin{verbatim}
>
> fit1$coefficients
(Intercept)           x
  0.7812216   0.8580521

>
> fit1$coefficients[1]  #intercept
(Intercept)
  0.7812216
>
> fit1$coefficients[2] #slope
       x
0.8580521
\end{verbatim}

<p>
%------------------------------------------

\begin{slide}{Simple Linear Regression}
An alternative method is to use the following commands.
\begin{itemize}
* coef() - returns the regression coefficients of the model.
* fitted() - returns the fitted values of the model.
* resid() - returns the residuals of the model.


\end{itemize}
 \begin{verbatim}
 > coef(fit1)
(Intercept)           x
  0.7812216   0.8580521
\end{verbatim}


<p>

%------------------------------------------

\begin{slide}{Coefficient of Determination}
\begin{itemize}
* The coefficient of determination $R^2$ is  the proportion of variability in a data set
that is accounted for by the linear model.
* $R^2$ provides a measure of how well future outcomes are likely to be predicted by the model.
* For simple linear regression, it can also be computed by squaring the correlation coefficient.
\end{itemize}

\begin{verbatim}
> summary(fit1)$r.squared
[1] 0.9181277
\end{verbatim}


<p>

%------------------------------------------

\begin{slide}{p-values}
\begin{itemize}
\item
We will begin to use hypothesis testing in out analyses.
* We will mostly be using ``p-values".
* If the p-value is very low, we reject the null hypothesis.
* If it is above an arbitrary threshold, we ``fail to reject" the null hypothesis.
* We will use 0.01 (1\%) as our arbitrary threshold.
* The relevant hypotheses will be discussed for each methodology.
\end{itemize}
<p>

%------------------------------------------

\begin{slide}{Inference for regression}
\begin{itemize}
\item
We can use the ``summary()" command to determine
test statistics and p-values for the both
regression coefficients.
* In both cases the null hypothesis is that the true value is zero.
* Consequently the alternative hypothesis is that they are not zero in both cases.
* Stating that the slope is zero is equivalent to saying that there is no relationship between `x' and `y'.
\end{itemize}
<p>
%------------------------------------------

\begin{slide}{Inference for regression}
\begin{verbatim}
> summary(fit1)

Call:
lm(formula = y ~ x)

Residuals:
     Min       1Q   Median       3Q      Max
-0.56320 -0.24413  0.06588  0.19946  0.67913

Coefficients:
        Estimate Std. Error t value Pr(>|t|)
(Int.)  0.78122    0.58121   1.344    0.209
x       0.85805    0.08103  10.590 9.38e-07 ***

.....

\end{verbatim}
<p>
%------------------------------------------

\begin{slide}{Inference for regression}
\begin{itemize}
* The p value for the intercept is 0.209. This means we fail to reject the null hypothesis that true intercept is zero.
* The p value for the slope is extremely small. This means we reject the null hypothesis that it is zero.
* Consequently we reject the hypothesis that there is no relationship between `x' and `y'.
* Notice that the stars beside the p-value. The more stars, the lower the p-value.
\end{itemize}
<p>

