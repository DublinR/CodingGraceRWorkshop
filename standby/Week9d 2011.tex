

%------------------------------------------

\begin{slide}{Bivariate Data}

\begin{itemize}
* A dataset with two variables contains what is called bivariate data.

* Each pair of values describes the same instance. For example, the first value of both variables describe the same individual or item.
    
* It is often of interest to describe the relationship between two variables.
\end{itemize}
<p>
%------------------------------------------

\begin{slide}{Data used today}

\begin{verbatim}
>x<-c(5.98, 8.80, 6.89, 8.49, 8.48, 7.47,
    7.97,5.94, 7.32, 6.64, 6.94, 3.51)
>
>
>y<-c(5.56, 7.80, 6.13, 8.15, 7.95, 7.87,
    8.03, 5.67, 7.11, 6.65, 7.02, 3.88)

\end{verbatim}
It is important the both data sets have same length.
\begin{verbatim}
>length(x)
>length(y)

\end{verbatim}
<p>

%------------------------------------------
\begin{slide}{Regression Models}

\begin{itemize}
* Simple linear regression (today).
    \begin{itemize}
    * Estimates for slope and intercept.
    * Accessing those estimates.
    * Inference on those estimates.
    \end{itemize}
* Multiple linear regression (later).
    \begin{itemize}
    * Model selection
    \end{itemize}
* Non-linear regression (later).
* Diagnostics (later).
\end{itemize}
<p>
%------------------------------------------
\begin{slide}{Simple linear regression}
\begin{itemize}
* Simple linear regression is used to describe the relationship
between two variables `x' and `y'. * For example, you may want to describe the
relationship between age and blood pressure or the relationship
between scores in a midterm exam and scores in the final exam,
etc.
* `x' is the independent (i.e. predictor) variable
* `y' is the dependent (i.e. response) variable.
* Necessarily both x and y should be of equal length.

* One of the first steps in a regression analysis is to determine if any
kind of relationship exists between `x' and `y'.



\end{itemize}
<p>
%------------------------------------------
\begin{slide}{Simple linear regression}
\begin{itemize}

* A scatterplot can created and can initially be used to get an idea
about the nature of the relationship between the variables, e.g. if
the relationship is linear, curvilinear, or no relationship exists.

* We can see from a scatterplot that there is a linear relationship
between x and y.

* Simple linear regression is only useful when there is evidences of a linear relationship.
In other cases, such as quadratic relationships, other types of regression may be useful.

\end{itemize}
<p>
%------------------------------------------

\begin{slide}{Scatterplot}
\begin{itemize}
* To make a simple scatterplot of the bivariate data, we simply use the
``plot()" command. * The independent variable (the variable to go along the x-axis) is always specified first.
\begin{verbatim}
>plot(x,y)
\end{verbatim}
* In future classes, we will look at how to improve and enhance scatter-plots, by controlling graphical parameters.
\end{itemize}
<p>
