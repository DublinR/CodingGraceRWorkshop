
%--------------------------------------------------------Inference Procedures and testing for Normality-%

chapter{Normality Assumptions and Outliers}

#### {Grubbs Test for outliers}
<p>
#### {Anderson Darling Test}
<p>
#### {Normal Probability plots}

#### { Kolmogorov Smirnov Test}






<p>
#### {Subsetting datasets by rows}

Suppose we wish to divide a data frame into two different <p>
### . The simplest approach we can take is to create two new data sets, each assigned data from the relevant rows of the original data set.

Suppose our dataset ``Info" has the dimensions of 200 rows and 4 columns. We wish to separate "Info" into two subsets , with the first and second 100 rows respectively. ( We call these new subsets "Info.1" and "Info.2".)
<code>
Info.1 = Info[1:100,]		#assigning "info" rows 1 to 100
Info.2 = Info[101:200,]		#assigning "info" rows 101 to 200
</code>

More useful commands such as rbind() and cbind()  can be used to manipulate vectors.

Part 2 Strategies for Data project

Exploratory Data Analysis

The first part of your report should contain some descriptive statistics and summary values. Also include some tests for normality.

      {Regression}
You should have a data set with multiple columns, suitable for regression analysis.
Familiarize yourself with the data, and decide which variable is the dependent variable.

Also determine the independent variables that you will use as part of your analysis.

      {Correlation Analysis}
Compute the Pearson correlation for the dependent variable with the respective independent variables.  As part of your report, mention the confidence interval for the correlation estimate
Choose the independent variables with the highest correlation as your candidate variables.
For these independent variables, perform a series of simple linear regression procedures.
<code>
lm(y~x1)
lm(y~x2)
</code>
Comment on the slope and intercept estimates and their respective p-values. Also comment on the coefficient of determination (multiple R squared). Remember to write the regression equations.
Perform a series of multiple linear regressions, using pairs of candidate independent variables.
<code>
lm(y~x1 +x2)
lm(y~x2 +x3)
</code>
Again, comment on the slope and intercept estimates, and their respective p-values.
In this instance, compare each of the models using the coefficient of determinations. Which model explains the data best?
<p>
#### {Analysis of residuals}
Perform an analysis of regression residuals ( you can pick the best regression model from last <p>
### ).
Are the residuals normally distributed?
	Histogram /  Boxplot / QQ plot / Shapiro Wilk Test
Also you can plot the residuals to check that there is constant variance.
<code>
y=rnorm(10)
x=rnorm(10)
fit1=lm(y~x)
res.fit1 = resid(fit1)
plot(res.fit1)
</code>




%---------------------------------------------------------------------------Probability Distributions ----%

chapter{Probability Distributions}
<p>
### {Generating a set of random numbers}

%begin{myindentpar}{1cm}
 <code>
rnorm(10)
</code>
%end{myindentpar}

<p>
### {The Poisson Distribution}
<p>
### {The Binomial Distribution}
<p>
### {Using probability distributions for simulations}
<p>
### {Probability Distributions}
<p>
#### {Generate random numbers }

%----------------------------------------------------------------------------Graphical Methods--%

chapter{Graphical methods}

<p>
### {Scatterplots}
%begin{figure}
%  % Requires usepackage{graphicx}
%  includegraphics[scale=0.40]{MTCARSmpgwt.png}
%  caption{Scatterplot}label{mpgwt}
%end{figure}


<p>
### {Adding titles, lines, points to plots}


 <code>
library(MASS)
# Colour points and choose plotting symbols according to a levels of a factor
plot(Cars93$Weight, Cars93$EngineSize, col=as.numeric(Cars93$Type),
pch=as.numeric(Cars93$Type))

# Adds x and y axes labels and a title.
plot(Cars93$Weight, Cars93$EngineSize, ylab="Engine Size",
xlab="Weight", main="My plot")
# Add lines to the plot.
lines(x=c(min(Cars93$Weight), max(Cars93$Weight)), y=c(min(Cars93$EngineSize),
max(Cars93$EngineSize)), lwd=4, lty=3, col="green")
abline(h=3, lty=2)
abline(v=1999, lty=4)
# Add points to the plot.
</code>


chapter{Programming}







%----------------------------------------------------%

#### {Two Sample t test}

The two-sample t test is used to test the hypothesis that two samples may
be assumed to come from distributions with the same mean.

The theory for the two-sample t test is not very different in principle from
that of the one-sample test. Data are now from two groups, $x_{11}, . . . , x_{1n1}$
and $x_{21}, . . . , x_{2n2}$ , which we assume are sampled from the normal distributions
$N(µ_{1}, sigma^{1}_{2} )$ and
$N(µ_{2}, sigma^{2}_{2} )$, and it is desired to test the null hypothesis
$mu_{1} = mu_{2}$. You then calculate

[
t = frac{bar{X}_{1}-bar{X}_{2}}{S.E.(bar{X}_{1}-bar{X}_{2})}
]




%---------------------------------------------------%

#### {slide234}
The TS are <equation here>  
The p-values for both of these tests are 0 and so there is enough evidence to reject $H_0$ and conclude that both 0 and 1 are not 0, i.e. there is a significant linear relationship between x and y. 
Also given are the $R^2$ and $R^2$ adjusted values. Here $R^2 = SSR/SST = 0.8813$ and so $88.13%$ of the variation in y is being explained by x. 
The final line gives the result of using the ANOVA table to assess the model t.

%----------------------------------------------------%


#### {slide235}

In SLR, the ANOVA table tests <EQN>The TS is the F value and the critical value and p-values are found
in the F tables with (p - 1) and (n - p) degrees of freedom.

This output gives the p-value = 0, therefore there is enough evidence to reject H0 and conclude that there is a signicant linear relationship between y and x. The full ANOVA table can be accessed using :

<TABLE HERE>




#### {slide236}
Once the model has been tted, must then check the residuals.
The residuals should be independent and normally distributed with
mean of 0 and constant variance.
A Q-Q plot checks the assumption of normality (can also use a
histogram as in MINITAB) while a, plot of the residuals versus fitted values gives an indication as to whether the assumption of constant variance holds.

<HISTOGRAM>




<p>
### {Introduction to texttt{R}}
texttt{R} consists of a base package and many additional packages
texttt{R} was originally designed as a command language.  
Commands were typed into a text-based input area on the computer screen and the program responded with a response to each command.
The texttt{R} console opens with information and then a prompt mark  ``>"  it is ready to accept commands
texttt{R}  is an open source software package, meaning that the code written to implement the various functions can be freely examined and modified.
texttt{R} can be installed free of charge from the texttt{R}-project website.




