
%--------------------------------------------------------Inference Procedures and testing for Normality-%

chapter{Normality Assumptions and Outliers}

#### {Grubbs Test for outliers}
<p>
#### {Anderson Darling Test}
<p>
#### {Normal Probability plots}

#### { Kolmogorov Smirnov Test}






<p>
#### {Subsetting datasets by rows}

Suppose we wish to divide a data frame into two different <p>
### . The simplest approach we can take is to create two new data sets, each assigned data from the relevant rows of the original data set.

Suppose our dataset ``Info" has the dimensions of 200 rows and 4 columns. We wish to separate "Info" into two subsets , with the first and second 100 rows respectively. ( We call these new subsets "Info.1" and "Info.2".)
<code>
Info.1 = Info[1:100,]		#assigning "info" rows 1 to 100
Info.2 = Info[101:200,]		#assigning "info" rows 101 to 200
</code>

More useful commands such as rbind() and cbind()  can be used to manipulate vectors.

Part 2 Strategies for Data project

Exploratory Data Analysis

The first part of your report should contain some descriptive statistics and summary values. Also include some tests for normality.

      {Regression}
You should have a data set with multiple columns, suitable for regression analysis.
Familiarize yourself with the data, and decide which variable is the dependent variable.

Also determine the independent variables that you will use as part of your analysis.

      {Correlation Analysis}
Compute the Pearson correlation for the dependent variable with the respective independent variables.  As part of your report, mention the confidence interval for the correlation estimate
Choose the independent variables with the highest correlation as your candidate variables.
For these independent variables, perform a series of simple linear regression procedures.
<code>
lm(y~x1)
lm(y~x2)
</code>
Comment on the slope and intercept estimates and their respective p-values. Also comment on the coefficient of determination (multiple R squared). Remember to write the regression equations.
Perform a series of multiple linear regressions, using pairs of candidate independent variables.
<code>
lm(y~x1 +x2)
lm(y~x2 +x3)
</code>
Again, comment on the slope and intercept estimates, and their respective p-values.
In this instance, compare each of the models using the coefficient of determinations. Which model explains the data best?
<p>
#### {Analysis of residuals}
Perform an analysis of regression residuals ( you can pick the best regression model from last <p>
### ).
Are the residuals normally distributed?
	Histogram /  Boxplot / QQ plot / Shapiro Wilk Test
Also you can plot the residuals to check that there is constant variance.
<code>
y=rnorm(10)
x=rnorm(10)
fit1=lm(y~x)
res.fit1 = resid(fit1)
plot(res.fit1)
</code>




%---------------------------------------------------------------------------Probability Distributions ----%

chapter{Probability Distributions}
<p>
### {Generating a set of random numbers}

%begin{myindentpar}{1cm}
 <code>
rnorm(10)
</code>
%end{myindentpar}

<p>
### {The Poisson Distribution}
<p>
### {The Binomial Distribution}
<p>
### {Using probability distributions for simulations}
<p>
### {Probability Distributions}
<p>
#### {Generate random numbers }

%----------------------------------------------------------------------------Graphical Methods--%

chapter{Graphical methods}

<p>
### {Scatterplots}
%begin{figure}
%  % Requires usepackage{graphicx}
%  includegraphics[scale=0.40]{MTCARSmpgwt.png}
%  caption{Scatterplot}label{mpgwt}
%end{figure}


<p>
### {Adding titles, lines, points to plots}


 <code>
library(MASS)
# Colour points and choose plotting symbols according to a levels of a factor
plot(Cars93$Weight, Cars93$EngineSize, col=as.numeric(Cars93$Type),
pch=as.numeric(Cars93$Type))

# Adds x and y axes labels and a title.
plot(Cars93$Weight, Cars93$EngineSize, ylab="Engine Size",
xlab="Weight", main="My plot")
# Add lines to the plot.
lines(x=c(min(Cars93$Weight), max(Cars93$Weight)), y=c(min(Cars93$EngineSize),
max(Cars93$EngineSize)), lwd=4, lty=3, col="green")
abline(h=3, lty=2)
abline(v=1999, lty=4)
# Add points to the plot.
</code>


chapter{Programming}







%----------------------------------------------------%

#### {Two Sample t test}

The two-sample t test is used to test the hypothesis that two samples may
be assumed to come from distributions with the same mean.

The theory for the two-sample t test is not very different in principle from
that of the one-sample test. Data are now from two groups, $x_{11}, . . . , x_{1n1}$
and $x_{21}, . . . , x_{2n2}$ , which we assume are sampled from the normal distributions
$N(µ_{1}, sigma^{1}_{2} )$ and
$N(µ_{2}, sigma^{2}_{2} )$, and it is desired to test the null hypothesis
$mu_{1} = mu_{2}$. You then calculate

[
t = frac{bar{X}_{1}-bar{X}_{2}}{S.E.(bar{X}_{1}-bar{X}_{2})}
]




%---------------------------------------------------%

#### {slide234}
The TS are <equation here>  
The p-values for both of these tests are 0 and so there is enough evidence to reject $H_0$ and conclude that both 0 and 1 are not 0, i.e. there is a significant linear relationship between x and y. 
Also given are the $R^2$ and $R^2$ adjusted values. Here $R^2 = SSR/SST = 0.8813$ and so $88.13%$ of the variation in y is being explained by x. 
The final line gives the result of using the ANOVA table to assess the model t.

%----------------------------------------------------%


#### {slide235}

In SLR, the ANOVA table tests <EQN>The TS is the F value and the critical value and p-values are found
in the F tables with (p - 1) and (n - p) degrees of freedom.

This output gives the p-value = 0, therefore there is enough evidence to reject H0 and conclude that there is a signicant linear relationship between y and x. The full ANOVA table can be accessed using :

<TABLE HERE>




#### {slide236}
Once the model has been tted, must then check the residuals.
The residuals should be independent and normally distributed with
mean of 0 and constant variance.
A Q-Q plot checks the assumption of normality (can also use a
histogram as in MINITAB) while a, plot of the residuals versus fitted values gives an indication as to whether the assumption of constant variance holds.

<HISTOGRAM>




<p>
### {Introduction to texttt{R}}
texttt{R} consists of a base package and many additional packages
texttt{R} was originally designed as a command language.  
Commands were typed into a text-based input area on the computer screen and the program responded with a response to each command.
The texttt{R} console opens with information and then a prompt mark  ``>"  it is ready to accept commands
texttt{R}  is an open source software package, meaning that the code written to implement the various functions can be freely examined and modified.
texttt{R} can be installed free of charge from the texttt{R}-project website.

%----------------------------------------------------%

#### {slidename}

 <code>
> xbar <- 83
> sigma <- 12
> n <- 5
> sem <- sigma/sqrt(n)
> sem
[1] 5.366563
> xbar + semqnorm(0.025)
[1] 72.48173
> xbar + semqnorm(0.975)
[1] 93.51827
 </code>



#### {Testing the slope (II)}

You can compute a
t test for that hypothesis simply by dividing the estimate by its standard
error
begin{equation}
t = frac{hat{beta}}{S.E.(hat{beta})}
end{equation}
which follows a t distribution on n - 2 degrees of freedom if the true $beta$ is
zero.


%----------------------------------------------------%

The standard $chi^{2}$ test  in chisq.test works with data in matrix form, like fisher.test does.
For a 2 by 2 table, the test is exactly equivalent to prop.test.



 <code>
> chisq.test(lewitt.machin)
</code>







chapter { R Graphics}
<p>
###  Enhancing your scatter plots
<p>
#### {Adding lines}
Previously we have used scatter plots to plot bivariate data. They were constructed using the plot() command.
Recall that we can use the arguments texttt{xlim} and texttt{ylim} to control the vertical and horizontal range of the plots, by specifying a two element vector (min and max) for each.

Using the texttt{abline()} command, we can add lines to our scatter plots. We specify the argument according to the type of line required. A demonstration of three types of line is provided below.
Additionally we change the colour of the added lines, by specifying a colour in the texttt{col} argument. We can also change the line type to one of four possible types, using the texttt{lty} argument.

The line types are follows

      	texttt{lty =1}   Normal full line (default)
      	texttt{lty =2}   Dashed line
      	texttt{lty =3}   Dotted line
      	texttt{lty =4}   Dash-dot line

 <code>
x=rnorm(10)
y=rnorm(10)
plot(x,y)
plot(x,y,xlim=c(-4,4),ylim=c(-4,4))
abline(v =0 , lty =2 )    # add a vertical dotted line (here the y-axis) to the plot
abline(h=0  ,lty =3)    # add a horizontal dotted line (here the x-axis) to the plot
abline(a=0,b=1,col="green") # add a line to your plot with intercept "a" and slope "b"
 </code>

<p>
#### {Changing your plot character}

To change the plot character (the symbol for each covariate, we supply an additional argument to the plot() function.  This argument is formulated as pch=n where n is some number.
Additionally we change the colour of the characters, by specifying a colour in the col argument.
 <code>
plot(x,y,pch=15,col="red")		#Square plot symbols
plot(x,y,pch=16,col="green")		#Orb plot symbols
plot(x,y,pch=17,col="mauve")		#Triangular plot symbols
plot(x,y,pch=36	,col="amber")		#Dollar sign plot symbols
</code>
Recall that we can add new variates to an existing scatterplot using the points() function. Remember to set the vertical and horizontal limits accordingly.
 <code>
y1 = rnorm(10); y2 = rnorm(10)
plot(x,y1, pch=8,col="purple" ,xlim=c(-5,5),ylim=c(-5,5))
points(x,y2,pch=12,col="green")
</code>
<p>
#### {Adding the regression model line}

The texttt{abline()} function can be used to add a regression model line  by supplying as an argument the texttt{coef()} values for intercept and slope estimates .These estimates can be inputted directly by using both functions in conjunction.

 <code>
Fit1 =lm(y1~x);  coef(Fit1)
abline(coef(Fit1))	
</code>

<p>
#### {Adding a title }

It is good practice to label your scatterplots properly. You can specify the following argument

      	main="Scatterplot Example", 	This provides the plot with a title
      	sub="Subtitle",   This adds a subtitle
      	xlab="X variable ",				This command labels the x axis 
  ylab="y variable ",				This command labels the y-axis

We can also add text to each margin, using the texttt{mtext()} command.  
We simply require the number of the side. (1 = bottom, 2=left,3=top,4=right). 
We can change the colour using the col argument.
 <code>
plot(x,y,main="Scatterplot Example",   sub="subtitle",    xlab="X variable ", ylab="y variable ")	
mtext("Enhanced Scatterplot", side=4,col="red ")
</code>
Alternatively , we can also use the command title() to add a title to an existing scatterplot.
 <code>
title(main="Scatterplot Example)	
</code>


<p>
### {Combining plots}
It is possible to combine two plots. We used the graphical parameters command texttt{par()} to create an array. 
Often we just require two plots side by side or above and below. We simply specify the numbers of rows and columns of this array using the texttt{mfrow} argument, passed as a vector.

<code>
par(mfrow=c(1,2))
plot(x,y1)			# draw first plot
plot(x,y2)			# draw second plot
par(mfrow=c(1,1))		# reset to default setting.
</code>

<p>
### {Plot of single vectors}
If only one vector is specified i.e. texttt{plot(x)},  the plot created will simply be a scatter-plot of the values of x against their indices.

$plot(x)$
Suppose we wish to examine a trend that these points represent. We can connect each covariate using a line.

$plot(x, type = "l")$
If we wish to have both lines and points, we would input the following code. This is quite useful if we wish to see how a trend develops over time.
$plot(x, type = "b")$









<p>
### {Exercise} The following are measurements (in mm) of a critical
dimension on a sample of twelve engine crankshafts:

<code>
224.120 	224.001 	224.017 	223.982 	223.989 	223.961
223.960 	224.089 	223.987 	223.976 	223.902 	223.980
</code>
(a) Calculate the mean and standard deviation for these data.
(b) The process mean is supposed to be ? = 224mm. Is this the
case? Give reasons for your answer.
(c) Construct a 99% confidence interval for these data and interpret.
(d) Check that the normality assumption is valid using 2 suitable plots.

<code>
> x<-c(224.120,224.001,224.017,223.982 ,223.989 ,223.961,
+ 223.960 ,224.089 ,223.987 ,223.976 , 223.902 ,223.980)
>
> mean(x)
[1] 223.997
>
> sd(x)
[1] 0.05785405
>
> t.test(x,mu=224,conf.level=0.99)

 One Sample t-test

data:  x
t = -0.1796, df = 11, p-value = 0.8607
alternative hypothesis: true mean is not equal to 224
99 percent confidence interval:
 223.9451 224.0489
sample estimates:
mean of x
  223.997

</code>
<p>
### {Exercise 2} 
The height of 12 Americans and 10 Japanese was measured. Test for a difference in the heights of both populations.
<code>
Americans
174.68   	169.87 	   	165.07    	165.95 		204.99 		177.61 	
170.11 	 	170.71 	   	181.52 		167.68 		158.62 		182.90
Japanese
158.76  		168.85  		159.64  		180.02  		164.24
161.91  		163.99  		152.71  		157.32  		147.20
</code>
<code>
> t.test(A,J)
 Welch Two Sample t-test
data:  A and J
t = 2.8398, df = 19.815, p-value = 0.01018
alternative hypothesis: true difference in means is not equal to 0
95 percent confidence interval:
  3.360121 21.996879
sample estimates:
mean of x mean of y
 174.1425  161.4640
</code>

<p>
### {Exercise 3}

A large group of students each took two exams. The marks obtained in both exams by a sample of eight students is given below

<code>
Student	1	2	3	4	5	6	7	8
Exam 1	57	76	47	39	62	56	49	81
Exam 2	67	81	62	49	57	61	59	71
</code>
Test the hypothesis that in the group as a whole the mean mark gained did not vary according to the exam against the hypothesis that the mean mark in the second exam was higher
<code>
>
> Ex1<-c(57,76,47,39,62,56,49,81)
> Ex2<-c(67,81,62,49,57,61,59,71)
> t.test(Ex1-Ex2)

 One Sample t-test

data:  Ex1 - Ex2
t = -1.6733, df = 7, p-value = 0.1382
alternative hypothesis: true mean is not equal to 0
95 percent confidence interval:
 -12.065666   2.065666
sample estimates:
mean of x
-5
</code>

<p>
### {Exercise 4}
A poll on social issues interviewed 1025 people randomly selected from the United States. 450 of people said that they do not get enough time to themselves. A report claims that over 41% of the population are not satisfied with personal time. Is this the case?

<code>

> prop.test(450,1025,p=0.40,alternative="greater")

 1-sample proportions test with continuity correction

data:  450 out of 1025, null probability 0.4
X-squared = 6.3425, df = 1, p-value = 0.005894
alternative hypothesis: true p is greater than 0.4
95 percent confidence interval:
 0.413238 1.000000
sample estimates:
 p
0.4390244
</code>

Exercise 23b:  A company wants to investigate the proportion of males and females promoted in the last year. 45 out of 400 female candidates were promoted, while 520 out of 3270 male candidates were promoted. Is there evidence of sexism in the company?
<code>
> x.vec=c(45,520)
> n.vec=c(400,3270)
>  prop.test(x.vec,n.vec)

 2-sample test for equality of proportions with continuity correction

data:  x.vec out of n.vec
X-squared = 5.5702, df = 1, p-value = 0.01827
alternative hypothesis: two.sided
95 percent confidence interval:
 -0.08133043 -0.01171238
sample estimates:
   prop 1    prop 2
0.1125000 0.1590214
</code>

?
<p>
### {Exercise}

Generate a histogram for data set 'scores', with an accompanying box-and-whisker plot.
The colour of the histogram's bar should be yellow. The orientation for the boxplot should be horizontal.

<code>
scores <-c(23,19,22,22,19,20,25,26,26,19,24,23,17,21,28,26)

par(mfrow=c(2,1)) 	# two rows , one column

hist(scores,main="Distribution of scores",xlab="scores",col="yellow")

boxplot(scores ,horizontal=TRUE)

par(mfrow =c(1,1)) 	#reset
</code>

%----------------------------------------------------------------%
<p>
### {Vector Operations}

$R$ operates on named data structures. The simplest such
structure is the vector, which is a single entity consisting of an
ordered collection of numbers or characters.

The most common types of vectors are:

Numeric vectors Character vectors Logical
vectors


There are, of course, other types of vectors.

Colour vectors - potentially useful later on.
Order vectors - The rankings ofs in a vector.
Complex number vectors - not part of this course.


<p>
#### {Ordering Vector Operations}
<pre>
<code>
sort(x)  # sort x into ascending order
rev(x)
rev(sort(x))
</code>
</pre>
<p>

<pre>
<code>
x=c(15, 34, 7, 12, 18, 9, 1, 42, 56, 28, 13, 24, 35)

length(x)  # How manys in x
median(x)  # median of data set x
sort(x)[7] # 7th when x is in ascending order
quantile(x,0.75)  # Compute the third quartile
quantile(x,0.25)  # Compute the first quartile
IQR(x)     
fivenum(x)

# code is committed
</code>
</pre>
<p>
