
%--------------------------------------------------------Inference Procedures and testing for Normality-%

chapter{Normality Assumptions and Outliers}

#### {Grubbs Test for outliers}
<p>
#### {Anderson Darling Test}
<p>
#### {Normal Probability plots}

#### { Kolmogorov Smirnov Test}






<p>
#### {Subsetting datasets by rows}

Suppose we wish to divide a data frame into two different <p>
### . The simplest approach we can take is to create two new data sets, each assigned data from the relevant rows of the original data set.

Suppose our dataset ``Info" has the dimensions of 200 rows and 4 columns. We wish to separate "Info" into two subsets , with the first and second 100 rows respectively. ( We call these new subsets "Info.1" and "Info.2".)
<code>
Info.1 = Info[1:100,]		#assigning "info" rows 1 to 100
Info.2 = Info[101:200,]		#assigning "info" rows 101 to 200
</code>

More useful commands such as rbind() and cbind()  can be used to manipulate vectors.

Part 2 Strategies for Data project

Exploratory Data Analysis

The first part of your report should contain some descriptive statistics and summary values. Also include some tests for normality.

      {Regression}
You should have a data set with multiple columns, suitable for regression analysis.
Familiarize yourself with the data, and decide which variable is the dependent variable.

Also determine the independent variables that you will use as part of your analysis.

  
