### {Graphical data entry interface}

texttt{Data.entry()} is a useful  command for inputting or editing data sets. Any
changes are saved automatically (i.e. don’t need to use the assignment
operator). We can also used the texttt{edit()} command, which calls the texttt{R Editor}.

<code>
>data.entry(NumVec)
>NumVec <- edit(NumVec)
</code>

Another method of creating vectors is to use the following
<code>
numeric (length = n)
character (length = n)
logical (length = n)
</code>
These commands create empty vectors, of the appropriate kind, of length $n$. You can then use the graphical data entry interface to populate your data sets.


#### {Accessing specified elements of a vector}

The $n$th element of vector ``Vec" can be accessed by specifying its index when
calling ``Vec".
<code>>Vec[n]
</code>
A sequence of  elements of vector ``Vec" can be accessed by specifying its index
when calling ``Vec".
<code>>Vec[l:u]
</code>
Omitting and deleting the $n$th element of vector ``Vec"
<code>
>Vec[-n]
>Vec <- Vec[-n]
</code>

<p>
### {Reading data}


<p>
#### {inputting data}
 Concatenation

<p>
#### {using help}

?mean

<p>
#### {Adding comments}

<p>
#### {Packages}
The capabilities of R are extended through user-submitted packages, which allow specialized statistical techniques, graphical devices, as well as and
import/export capabilities to many external data formats.
