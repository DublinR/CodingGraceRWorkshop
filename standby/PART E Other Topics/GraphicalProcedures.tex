


\section{Bivariate Data}
\footnotesize \begin{myindentpar}{1cm}
\begin{verbatim}
> Y=mtcars$mpg
> X=mtcars$wt
>
> cor(X,Y)          #Correlation
[1] -0.8676594
>
> cov(X,Y)          #Covariance
[1] -5.116685
\end{verbatim}
\end{myindentpar}\normalsize


\section{Histograms}
Histograms can be created using the \texttt{hist()} command.
To create a histogram of the car weights from the Cars93 data set
\footnotesize
\begin{myindentpar}{1cm}
\begin{verbatim}
hist(mtcars$mpg, main="Histogram of MPG (Data: MTCARS) ")
\end{verbatim}
\end{myindentpar}\normalsize
\texttt{R} automatically chooses the number and width of the bars. We can
change this by specifying the location of the break points.
\footnotesize
\begin{myindentpar}{1cm}
\begin{verbatim}hist(Cars93$Weight, breaks=c(1500, 2050, 2300, 2350, 2400,
2500, 3000, 3500, 3570, 4000, 4500), xlab="Weight",
main="Histogram of Weight")
\end{verbatim}
\end{myindentpar}\normalsize



\section{Boxplot}
Boxplots can be used to identify outliers.

By default, the \texttt{boxplot()} command sets the orientation as vertical. By adding the argument \texttt{horizontal=TRUE}, the orientation can be changed to horizontal.
\footnotesize
\begin{myindentpar}{1cm}
\begin{verbatim}
boxplot(mtcars$mpg, horizontal=TRUE, xlab="Miles Per Gallon",
main="Boxplot of MPG")
\end{verbatim}
\end{myindentpar}\normalsize

\begin{figure}
  % Requires \usepackage{graphicx}
  \includegraphics[scale=0.4]{MTCARSboxplot.png}\\
  \caption{Boxplot}\label{boxplot}
\end{figure}

