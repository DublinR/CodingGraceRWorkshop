
### {The R Programming Language}

The R Programming Language is a statistical , data analysis , etc

R is a free software environment for statistical computing and graphics.

<p>
### {Writing R scripts}
Editing your R script ``R Editor".

On the menu of the R console, click on ‘file’.
Select ‘open script’ or ‘new script’ as appropriate.
Navigate to your working directory and select your ‘texttt{.R}’ file
A new dialogue box ``texttt{the R editor}" will open up.
Input or select code you wish to compile.
To compile this code, highlight it. Click the ‘edit button’ on the menu.
Select either ``Run Line" or ``Run Selection or All".
Your code should now compile.
To save your code, clink on ``file" and then ``texttt{save as}".
Save the file with the ``texttt{.R}" extension to your working directory.




#### {Reading data}


<p>
#### {inputting data}
 Concatenation

<p>
#### {using help}

?mean

%<p>
#### {Adding comments}

<p>
#### {Packages}
The capabilities of R are extended through user-submitted packages, which allow specialized statistical techniques, graphical devices, as well as and
import/export capabilities to many external data formats.


<p>
#### {Data frame}
A Data frame is





<p>
### {Bivariate Data}
 begin{myindentpar}{1cm}
<code>
> Y=mtcars$mpg
> X=mtcars$wt
>
> cor(X,Y)   #Correlation
[1] -0.8676594
>
> cov(X,Y)   #Covariance
[1] -5.116685
</code>






chapter{Advanced R code}
<p>
### {Data frame}
A Data frame is
<p>
#### {Merging Data frames}

<p>
### {Functions}
Syntax to define functions

begin{myindentpar}{1cm}
<code>
 myfct <- function(arg1, arg2, ...) { function_body }
</code>

The value returned by a function is the value of the function body, which is usually an unassigned final expression, e.g.: return()

Syntax to call functions
begin{myindentpar}{1cm}
<code>
 myfct(arg1=..., arg2=...)
</code>



<p>
### {Time and Date}
It is useful . The length of time a program takes is interesting.


begin{myindentpar}{1cm}
<code>
date() # returns the current system date and time
</code>




