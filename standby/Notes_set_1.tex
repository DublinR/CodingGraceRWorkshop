

R Workshop
Introduction to R
The R environment
Basic Mathematical Calculations
Data Input and Output
Data Manipulation
Descriptive and Quantile Statistics
Frequency Analysis
Sequences
Data Objects
Preview
Graphical Methods
Task 1 : Bland Altman Plot
Matrices
Packages
Precision
Probability Distributions
Task 2 : Simulating rolls of a die.
Simulating a dice experiment
Task 3: Implementing the Central Limit Theorem
Hypothesis Tests
GUIs and  IDEs
 
 
  Introduction to R
History of R



Probability distributions
The discrete uniform distribution
parameters: min , max.
The default values are 0 and 1.

Other Mathematical Functions
 
Complex numbers
x = -1 ;  sqrt(x)  ;  str(x) ;               # variable is defined as numeric, not complex.
y = -1 +0i ;  sqrt(y)  ;  str(y) ;                  #variable is defined as complex .
Trigonometric  Functions
pi                                                        #returns the value of pi to six decimal places
sin(3.5*pi)                                          # correct answer is -1
cos(3.5*pi)                         	             # correct answer is zero
Useful R commands for linear algebra

 
 
var( Mat1[,1] )                                          # determine the variance of the first column
var ( Mat1[,2] )                                          # determine the variance of the second column
var ( Mat1[3,] )                                          # determine the variance of the third row
cov ( Mat1[,1], Mat1[,2] )              # covariance of the first two columsn
var ( Mat1)                                          # variance covariance matrix of all columns
VCmat=var( Mat1)                            # Save as matrix “VCmat”
cor ( Mat1)                                          # correlation matrix of all columns
cov2cor( VCmat)                            #convert a VC matrix to a correlation matrix
 
seq() and rep() provide convenient ways to a construct vectors with a certain pattern. 
> seq(10) 
 [1]  1  2  3  4  5  6  7  8  9 10 
> seq(0,1,length=10) 
 [1] 0.0000000 0.1111111 0.2222222 0.3333333 0.4444444 0.5555556 0.6666667 
 [8] 0.7777778 0.8888889 1.0000000 
> seq(0,1,by=0.1) 
 [1] 0.0 0.1 0.2 0.3 0.4 0.5 0.6 0.7 0.8 0.9 1.0 
> rep(1,3) 
[1] 1 1 1 
> c(rep(1,3),rep(2,2),rep(-1,4)) 
[1]  1  1  1  2  2 -1 -1 -1 -1 
> rep("Small",3) 
[1] "Small" "Small" "Small" 
> c(rep("Small",3),rep("Medium",4)) 
[1] "Small"  "Small"  "Small"  "Medium" "Medium" "Medium" "Medium" 
> rep(c("Low","High"),3) 
[1] "Low"  "High" "Low"  "High" "Low"  "High"

Descriptive Statistics and Basic Graphical Methods
Measures of Centrality and Dispersion
# Anscombe’s  Quartet
X1 = c(
Y1 =

Histograms
 
>sd(X)
[1]
 
Introduction to Statistical Computing
Broadly speaking, the aim of statistics is to understand data and the relationships between variables. Due to the growth in information technology, data sets have grown considerably as the capacity for data collection and storage has increased.
 
In addition, computer processors are more affordable, so that most modern applied statisticians are involved in computing. To analyse data, statisticians and programmers have developed statistical software and
programming languages.
 
A good example is the S Programming Language, which was designed especially for programming with data. There are now a large number of software libraries written in S which can be used to solve almost any practical statistical problem.
 
To start R in Windows, double click the R icon. To start R in Unix or Linux, type ‘R’ at the command prompt. To get out of R, just type: q().
R has an inbuilt help facility. To get more information on any specific named
function, for example “boxplot”, the command is:
?boxplot
 
 
R has data handling and storage facilities, a suite of operators for calculations on arrays in particular matrices. A large coherent integrated collection of intermediate tools for data analysis
A large selection of demonstration datasets used in the illustration of many statistical methods.
Graphical facilities for data analysis and display either directly.at the computes or on hardcopy.
 
 
%==============================================================% 
 
 




Graphics

# Goal: To make a panel of pictures.

par(mfrow=c(3,2))                       # 3 rows, 2 columns.

\begin{verbatim}
# Now the next 6 pictures will be placed on these 6 regions. 

# 1 --

plot(density(runif(100)), lwd=2)
text(x=0, y=0.2, "100 uniforms")        
abline(h=0, v=0)
	          # All these statements effect the 1st plot.

x=seq(0.01,1,0.01)
par(col="blue")                        # default colour to blue.

# 2 --
plot(x, sin(x), type="l")
lines(x, cos(x), type="l", col="red")

# 3 --
plot(x, exp(x), type="l", col="green")
lines(x, log(x), type="l", col="orange")

# 4 --
plot(x, tan(x), type="l", lwd=3, col="yellow")

# 5 --
plot(x, exp(-x), lwd=2)
lines(x, exp(x), col="green", lwd=3)

# 6 --
plot(x, sin(x*x), type="l")
lines(x, sin(1/x), col="pink")

# code is committed
 
\end{verbatim}

 
%--------------------------------------------------------------------------------------------------------------------------------------%
Probability distributions
The discrete uniform distribution
parameters: min , max.
The default values are 0 and 1.
 



var( Mat1[,1] )			# determine the variance of the first column 
var ( Mat1[,2] )			# determine the variance of the second column
var ( Mat1[3,] )			# determine the variance of the third row
cov ( Mat1[,1], Mat1[,2] )	# covariance of the first two columsn
var ( Mat1)			# variance covariance matrix of all columns
VCmat=var( Mat1)		# Save as matrix “VCmat”
cor ( Mat1)			# correlation matrix of all columns
cov2cor( VCmat)		#convert a VC matrix to a correlation matrix

 
Descriptive Statistics and Basic Graphical Methods
Measures of Centrality and Dispersion
# Anscombe’s  Quartet
X1 = c(
Y1 =
 

Anscombe’s Quartet 
The uniform distribution





logical and relational operators
 
||      logical "Or"      (i.e.  union)
&&     logical "And"   (i.e. intersection)   

 



Data Manipulation
Types of vectors
	logical
	numeric
	character
    Others  - Complex and Colour

creating a vector

scan()

X=c()

data.entry()

data manipulation
sort()
rev()
rep()
length()
order()


Indices

 
