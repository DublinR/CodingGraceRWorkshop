
%----------------------------------------------------%


#### {Chi-squared Test}

A $chi^2$ test is carried out on tabular data containing counts, e.g. the
number of animals that died, the number of days of rain, the
number of stocks that grew in value, etc.

Usually have two qualitative variables, each with a number of
levels, and want to determine if there is a relationship between the
two variables, e.g. hair colour and eye colour, social status and
crime rates, house price and house size, gender and left/right
handedness.

The data are presented in a contingency table:
right-handed left-handed TOTAL

begin{tabular}{|c|c|c|c|}
  hline
  % after : hline or cline{col1-col2} cline{col3-col4} ...
  & right-handed &left-handed & TOTALhline
  Male & 43 & 9 & 52 
  Female & 44 & 4 & 48 
  TOTAL & 87 & 13 & 100 
  hline
end{tabular}


The hypothesis to be tested is
$H0 :$There is no relationship between gender and left/right-handedness
$H1 :$There is a relationship between gender and left/right-handedness
 The values that we collect from our sample are called the observed
(O) frequencies (counts). Now need to calculate the expected (E)
frequencies, i.e. the values we would expect to see in the table, if
H0 was true.






%------------------------------------------------------%

#### {Two Sample Tests}


All of the previous hypothesis tests and confidence intervals can be
extended to the two-sample case.

The same assumptions apply, i.e. data are normally distributed in
each population and we may want to test if the mean in one
population is the same as the mean in the other population, etc.

Normality can be checked using histograms, boxplots and Q-Q
plots as before. The Anderson-Darling test can be used on
each group of data also.


%------------------------------------------------------%

#### {Implementation}

This can be carried out in R by hand:

 <code>
>obs.vals <- matrix(c(43,9,44,4), nrow=2, byrow=T)
>row.tots <- apply(obs.vals, 1, sum)
>col.tots <- apply(obs.vals, 2, sum)
>exp.vals <- row.tots%o%col.tots/sum(obs.vals)
>TS <- sum((obs.vals-exp.vals)^2/exp.vals)
>TS
>[1] 1.777415
 </code>


%------------------------------------------------------%




chapter { R Graphics}
<p>
###  Enhancing your scatter plots
<p>
#### {Adding lines}
Previously we have used scatter plots to plot bivariate data. They were constructed using the plot() command.
Recall that we can use the arguments texttt{xlim} and texttt{ylim} to control the vertical and horizontal range of the plots, by specifying a two element vector (min and max) for each.

Using the texttt{abline()} command, we can add lines to our scatter plots. We specify the argument according to the type of line required. A demonstration of three types of line is provided below.
Additionally we change the colour of the added lines, by specifying a colour in the texttt{col} argument. We can also change the line type to one of four possible types, using the texttt{lty} argument.

The line types are follows

      	texttt{lty =1}   Normal full line (default)
      	texttt{lty =2}   Dashed line
      	texttt{lty =3}   Dotted line
      	texttt{lty =4}   Dash-dot line

 <code>
x=rnorm(10)
y=rnorm(10)
plot(x,y)
plot(x,y,xlim=c(-4,4),ylim=c(-4,4))
abline(v =0 , lty =2 )    # add a vertical dotted line (here the y-axis) to the plot
abline(h=0  ,lty =3)    # add a horizontal dotted line (here the x-axis) to the plot
abline(a=0,b=1,col="green") # add a line to your plot with intercept "a" and slope "b"
 </code>

<p>
#### {Changing your plot character}

To change the plot character (the symbol for each covariate, we supply an additional argument to the plot() function.  This argument is formulated as pch=n where n is some number.
Additionally we change the colour of the characters, by specifying a colour in the col argument.
 <code>
plot(x,y,pch=15,col="red")		#Square plot symbols
plot(x,y,pch=16,col="green")		#Orb plot symbols
plot(x,y,pch=17,col="mauve")		#Triangular plot symbols
plot(x,y,pch=36	,col="amber")		#Dollar sign plot symbols
</code>
Recall that we can add new variates to an existing scatterplot using the points() function. Remember to set the vertical and horizontal limits accordingly.
 <code>
y1 = rnorm(10); y2 = rnorm(10)
plot(x,y1, pch=8,col="purple" ,xlim=c(-5,5),ylim=c(-5,5))
points(x,y2,pch=12,col="green")
</code>
<p>
#### {Adding the regression model line}

The texttt{abline()} function can be used to add a regression model line  by supplying as an argument the texttt{coef()} values for intercept and slope estimates .These estimates can be inputted directly by using both functions in conjunction.

 <code>
Fit1 =lm(y1~x);  coef(Fit1)
abline(coef(Fit1))	
</code>

<p>
#### {Adding a title }

It is good practice to label your scatterplots properly. You can specify the following argument

      	main="Scatterplot Example", 	This provides the plot with a title
      	sub="Subtitle",   This adds a subtitle
      	xlab="X variable ",				This command labels the x axis 
  ylab="y variable ",				This command labels the y-axis

We can also add text to each margin, using the texttt{mtext()} command.  
We simply require the number of the side. (1 = bottom, 2=left,3=top,4=right). 
We can change the colour using the col argument.
 <code>
plot(x,y,main="Scatterplot Example",   sub="subtitle",    xlab="X variable ", ylab="y variable ")	
mtext("Enhanced Scatterplot", side=4,col="red ")
</code>
Alternatively , we can also use the command title() to add a title to an existing scatterplot.
 <code>
title(main="Scatterplot Example)	
</code>
