
 

 
conversion and coercion
as.numeric()
as.integer()
Descriptive and Quantile Statistics
fivenum
quantiles
mean
Measures of Dispersion
  range
  variance
  covariance
  standard deviation

sum

 
 

Frequency Analysis
table()
 
Sequences 
 



\section{Graphical Methods}

barplot
histograms
jitter
plot
identify

abline
add a vertical line to a plot
add a horizontal line to a plot
 
boxplot.stats(count ~ spray, data = InsectSprays, col = "lightgray")
# *add* notches (somewhat funny here):
boxplot(count ~ spray, data = InsectSprays,        notch = TRUE, add = TRUE, col = "blue")



 
Task 1 : Bland Altman Plot
Computation of case-wise differences and averages.
 
Standard deviation of differences
 
abline()
 
 

 
=======
MS4024 Lecture notes set B

MS4024 Lecture notes set B
Basic statistics in R
Probability distributions in R.
Graphics in R
Inference Procedures
 
General Rules
· Prompt for commands in command window: >
· Continuation prompt when command incomplete: +
· Neither of these ever typed by user
· Command can be any length
· If you want to break a long command into multiple lines for readability, make sure R knows that more is to come by making the current line incomplete (Example: end the line with one of the three characters ({,)
· Multiple commands may appear on one line if separated by ;
Basic statistics in R

Summary statistics of a data set can be obtained from summary, or by using individual commands
like mean, sd, mad, and IQR. Standard hypothesis tests are also available, e.g., t.test
yields the standard tests for means of one or two normal samples. 
 
Probability distributions in R.

Standard probability distributions have short names in R as given by Table 2.1. Several
probability functions are available. Their names consists of two parts: the first part is the
name of the function (see Table 2.2), while the second part is the name as in Table 2.1.
E.g., a sample of size 10 from an exponential distribution with mean 3 is generated in R by
rexp(10,1/3) (R uses the failure intensity instead of the mean as parameter).

