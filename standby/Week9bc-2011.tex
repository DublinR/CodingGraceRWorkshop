
#### {Assessments}



%------------------------------------------------------------------------------------------------------%
#### {Some more vector functions}
\begin{itemize}
* cumsum()  - computes the cumulative sums for a vector
* cumprod() - computes the cumulative products for a vector
* diff()  - computes lagged  differences (default difference is 1)

\begin{verbatim}
> s <- c(1,1,3,4,7,11)
> cumsum(s)
[1] 1 2 5 9 16 27
>
> diff(s) # 1-1, 3-1, 4-3, 7-4, 11-7
[1] 0 2 1 3 4
>
> diff(s, lag = 2) # 3-1, 4-1, 7-3, 11-4
[1] 2 3 4 7
>
\end{verbatim}
\end{itemize}




%--------------------------------------------------------------------------------%

#### {The scan function}
\begin{itemize}
* We can use the function  ``scan()" when typing in data.
* It stops adding data when you enter a blank row.
* for more information type ``?scan" , but the basic usage is simple.

\begin{verbatim}
> y=scan()
1: 4 5 6 7 8 9
7:
Read 6 items
> y
[1] 4 5 6 7 8 9
\end{verbatim}
\end{itemize}



%------------------------------------------------------------------------------------------------------%

#### {Datasets Included with R}
\begin{itemize}
* R contains many datasets that are built-in to the software. * These datasets are stored as data
frames. * To see the list of datasets, type
\begin{verbatim}
> data()
\end{verbatim}
* Additionally, we will be using datasets embedded on the MASS package.
* To access those data sets, simply type
\begin{verbatim}
> library(MASS)
> data()
\end{verbatim}
\end{itemize}


%------------------------------------------------------------------------------------------------------%

#### {Datasets Included with R}

* A window will open and the available datasets are listed (many others are accessible from
external user-written packages, however). * To open the dataset called trees, simply type

\begin{verbatim}
> data(trees)
\end{verbatim}

* After doing so, the data frame trees is now in your workspace. *  To learn more about this (or
any other included dataset), type help(trees).


%------------------------------------------------------------------------------------------------------%

#### {Categorical Data}

* 
There is a distinction between types of data in statistics and R knows about some of these differences. In particular,
initially, data can be of three basic types: categorical, discrete numeric and continuous numeric. * Methods for viewing
and summarizing the data depend on the type, and so we need to be aware of how each is handled and what we can
do with it.



%--------------------------------------------------------------------------------%

#### {Categorical Data (contd)}

* Categorical data may be recorded using character vectors.
* For example, a survey asks people if they smoke or not. * The results may be recorded as follows.

\begin{verbatim}
>x=c("Yes","No","No","Yes","Yes","No","No","Yes")
\end{verbatim}

%------------------------------------------------------------------------------------------------------%

#### {Categorical Data (contd)}

* To summarize this data. we can use the ``table()" command

\begin{verbatim}
> table(x)
x
 No Yes
  4   4
\end{verbatim}

* The table command simply adds up the frequency of each unique value of the data.

%------------------------------------------------------------------------------------------------------%

#### {Data Frames : Introduction}
A data frame can be thought of

* as a data matrix or data set * is a generalization of a matrix 
* is a data structure of vectors and/or factors of the same length 
*  The frame has a unique set of row names. 
* Data in the same position across columns come from the same subject. 
* Most data sets are in the form of a data frame.

%------------------------------------------------------------------------------------------------------%

#### {Data Frames: Creating a data frame}

* Data frames can be constructed using component vectors.
* We can create data frames from variables using the ``data.frame()``
command:
\begin{verbatim}
>mean_weight=c(71.5,72.1,73.7,74.3,75.2,74.7)
>
>Gender=c("M", "M", "F", "F", "M", "M")
>
> d <- data.frame(mean_weight,Gender)
\end{verbatim}
\end{itemize}
