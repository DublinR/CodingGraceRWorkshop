
<p>
#### {Correlation  and Regression tests }

#### {Correlation Coefficient}
Strength of a linear relationship between $X$ and $Y$

%<pre>
<code>
M=1000
CorrData=numeric(M)
for (i in 1:M)
{
CorrData[i] = cor(rnorm(10),rnorm(10))
}
</code>
%</pre>
<p>
The null hypothesis is that the correlation coefficient is zero. 

The alternative hypothesis is that the correlation coefficients is greater than zero. 

The slope and intercept estimates 

These tests are given in the "Two Tailed" format. 
The one tailed format compares a null hypothesis where the parameter of interest has a true value of less than or equalt to one versus an alternative hypothesis stating that it has a value greater than zero. 
  



<p>
### {Programming Paradigms}
<p>
#### {While Loops}
The while loop can be used if the number of iterations required is not known beforehand. For example, if we want to continue looping until a certain condition is met, a while loop is useful.

The following is the syntax for a while loop:

<code>
while (condition){
   command
   command
}
</code>
The loop continues while texttt{condition == TRUE}.


Note: texttt{sample()} takes a sample of the specified size (here just one) from a range of values (here integers 1 to 100).
<pre>
<code>

#initialise a counter to zero
niter = 0		
#initialize an empty vector
numvec = numeric()


num = sample(1:100, 1)

#while loop
while(num != 20)
   {
   num = sample(1:100, 1)
   niter = niter + 1
   numvec = c(numvec,num)
   }
numvec
niter

</code>
</pre>
<p>


<p>
#### {Nested Loops}

<p>
#### {Sums of two dice rolls}
<pre>
<code>
#Set Up an Empty Matrix of 6 rows and 6 columns
Dice = matrix(0,6,6)

#Main Loop
for(i in 1:6)
	{
    #Nested Loop
	for(j in 1:6)
		{
		Dice[i,j] = i+j
		}
	}		
Dice   # Print your Results
</code>
</pre>
<p>
<p>
#### {Correlation Structure Example}


