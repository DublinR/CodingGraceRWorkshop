	\documentclass[a4paper,12pt]{article}
%%%%%%%%%%%%%%%%%%%%%%%%%%%%%%%%%%%%%%%%%%%%%%%%%%%%%%%%%%%%%%%%%%%%%%%%%%%%%%%%%%%%%%%%%%%%%%%%%%%%%%%%%%%%%%%%%%%%%%%%%%%%%%%%%%%%%%%%%%%%%%%%%%%%%%%%%%%%%%%%%%%%%%%%%%%%%%%%%%%%%%%%%%%%%%%%%%%%%%%%%%%%%%%%%%%%%%%%%%%%%%%%%%%%%%%%%%%%%%%%%%%%%%%%%%%%
\usepackage{eurosym}
\usepackage{vmargin}
\usepackage{amsmath}
\usepackage{framed}
\usepackage{graphics}
\usepackage{epsfig}
\usepackage{subfigure}
\usepackage{enumerate}
\usepackage{fancyhdr}

\setcounter{MaxMatrixCols}{10}
%TCIDATA{OutputFilter=LATEX.DLL}
%TCIDATA{Version=5.00.0.2570}
%TCIDATA{<META NAME="SaveForMode"CONTENT="1">}
%TCIDATA{LastRevised=Wednesday, February 23, 201113:24:34}
%TCIDATA{<META NAME="GraphicsSave" CONTENT="32">}
%TCIDATA{Language=American English}

\pagestyle{fancy}
\setmarginsrb{20mm}{0mm}{20mm}{25mm}{12mm}{11mm}{0mm}{11mm}
\lhead{Apache Spark} \rhead{Kevin O'Brien} \chead{Decision tree} %\input{tcilatex}

\begin{document}
\subsection*{Factors}
A factor is a special type of vector used to represent categorical
data, e.g. gender, social class, etc.
\begin{itemize}
	\item Stored internally as a numeric vector with values 1, 2, : : : ; k,
	where k is the number of levels.
	\item Can have either ordered and unordered factors.
	\item A factor with k levels is stored internally consisting of 2 items
	\begin{itemize}
		\item[(a)] a vector of k integers
		\item[(b)] a character vector containing strings describing what the k
		levels are.
	\end{itemize}
	\end{itemize}
	

		
		\subsection{Factors: Example}
		Consider a survey that has data on 200 females and 300 males. If
		the first 200 values are from females and the next 300 values are
		from males, one way of representing this is to create a vector
		\begin{framed}
			\begin{verbatim}
			> gender <- c(rep("female", 200), rep("male", 300))
			> #To change this into a factor
			> gender <- factor(gender)
			>
	\end{verbatim}
	\end{framed}	The factor gender is stored internally as
		\begin{enumerate}
	\item female
	\item male
		\end{enumerate}
	
		Each category, i.e. female and male, is called a level of the factor.
		To determine the levels of a factor the function levels() can be
		used:
		\begin{framed}
			\begin{verbatim}
			> levels(gender)
			[1] "female" "male"
			
	\end{verbatim}
	\end{framed}
	
	%-------------------------------------------------------%
		\subsection{Factors: Example}
		Five people are asked to rate the performance of a product on a
		scale of 1-5, with 1 representing very poor performance and 5
		representing very good performance. The following data were
		collected.
		\begin{framed}
			\begin{verbatim}
			> satisfaction <- c(1, 3, 4, 2, 2)
			> fsatisfaction <- factor(satisfaction, levels=1:5)
			> levels(fsatisfaction) <- c("very poor", "poor", "average",
			"good", "very good")
	\end{verbatim}
	\end{framed}	%-------------------------------------------------------%
		\begin{itemize}
			\item The first line creates a numeric vector containing the satisfaction
			levels of the 5 people. Want to treat this as a categorical variable
			and so the second line creates a factor. 
			\item The \texttt{levels=1:5} argument
			indicates that there are 5 levels of the factor. Finally the last line
			sets the names of the levels to the specified character strings.
		\end{itemize}
		\end{document}
