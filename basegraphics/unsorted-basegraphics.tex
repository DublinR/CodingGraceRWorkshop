	\chapter { R Graphics}
	
	\section{Important Graphical Procedures}
	\begin{enumerate}
		\item Histograms
		\item Box-plots
		\item Scatter-plots
	\end{enumerate}
	
	
	%----------------------------------------------------------------%
	
	
			
			\subsection{Adding a title }
			
			It is good practice to label your scatterplots properly. You can specify the following argument
			\begin{itemize}
				\item	main="Scatterplot Example", 	This provides the plot with a title
				\item	sub="Subtitle",                 This adds a subtitle
				\item	xlab="X variable ",				This command labels the x axis 
				\item   ylab="y variable ",				This command labels the y-axis
			\end{itemize}
			We can also add text to each margin, using the \texttt{mtext()} command.  
			We simply require the number of the side. (1 = bottom, 2=left,3=top,4=right). 
			We can change the colour using the col argument.
			\footnotesize \begin{verbatim}
			plot(x,y,main="Scatterplot Example",   sub="subtitle",    xlab="X variable ", ylab="y variable ")	
			mtext("Enhanced Scatterplot", side=4,col="red ")
			\end{verbatim}\normalsize
			Alternatively , we can also use the command title() to add a title to an existing scatterplot.
			\footnotesize \begin{verbatim}
			title(main="Scatterplot Example)	
			\end{verbatim}\normalsize
			
			

			
			\section{Exercise}
			
			Generate a histogram for data set 'scores', with an accompanying box-and-whisker plot.
			The colour of the histogram's bar should be yellow. The orientation for the boxplot should be horizontal.
			
			\begin{verbatim}
			scores <-c(23,19,22,22,19,20,25,26,26,19,24,23,17,21,28,26)
			
			par(mfrow=c(2,1)) 	# two rows , one column
			
			hist(scores,main="Distribution of scores",xlab="scores",col="yellow")
			
			boxplot(scores ,horizontal=TRUE)
			
			par(mfrow =c(1,1)) 	#reset
			\end{verbatim}
