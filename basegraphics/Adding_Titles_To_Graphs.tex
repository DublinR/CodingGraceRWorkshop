
\subsection{Adding a title }

It is good practice to label your scatterplots properly. You can specify the following argument

* 	main="Scatterplot Example", 	This provides the plot with a title
* 	sub="Subtitle",                 This adds a subtitle
* 	xlab="X variable ",				This command labels the x axis 
*    ylab="y variable ",				This command labels the y-axis

We can also add text to each margin, using the \texttt{mtext()} command.  
We simply require the number of the side. (1 = bottom, 2=left,3=top,4=right). 
We can change the colour using the col argument.
<p> 
<pre><code>
plot(x,y,main="Scatterplot Example",   sub="subtitle",    xlab="X variable ", ylab="y variable ")	
mtext("Enhanced Scatterplot", side=4,col="red ")
</code></pre>\normalsize
Alternatively , we can also use the command title() to add a title to an existing scatterplot.
<p> <pre><code>
title(main="Scatterplot Example)	
</code></pre>\normalsize




\section{Plot of single vectors}
If only one vector is specified i.e. \texttt{plot(x)},  the plot created will simply be a scatter-plot of the values of x against their indices.

$plot(x)$
Suppose we wish to examine a trend that these points represent. We can connect each covariate using a line.

$plot(x, type = "l")$
If we wish to have both lines and points, we would input the following code. This is quite useful if we wish to see how a trend develops over time.
$plot(x, type = "b")$


