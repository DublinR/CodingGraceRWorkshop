
%----------------------------------------%

begin{frame}

frametitle{Test for Equality of Variance (a)}
begin{ }
         * In this procedure, we determine whether or not two data sets have the same variance.
         * The assumption of equal variance underpins several inference procedures.
         * We will not get into too much detail about that, but concentrate on how to assess equality of variance.
         * The null and alternative hypotheses are as follows:
[ H_0: sigma^2_1 = sigma^2_2 ]
[ H_1: sigma^2_1 neq sigma^2_2 ]
end{ }


begin{ }
         * When using texttt{R} it would be convenient to consider the null and alternative in terms of variance ratios.
         * Two data sets have equal variance if the variance ration is 1.
end{ }

[ H_0: sigma^2_1 / sigma^2_2 = 1 ]
[ H_1: sigma^2_1 / sigma^2_2 neq 1 ]

%----------------------------------------%
% - x=c(14,13,16,20,12,18,11,09,13,11)
% - y=c(15,13,18,20,10,17,23,11,10)

frametitle{Test for Equality of Variance(c)}
You would be required to compute the test statistic for this procedure.
The test statistic is the ratio of the variances for both data sets.
[ TS = frac{s^2_x}{s^2_y} ]
The standard deviations would be provided in the texttt{R} code. <code>
> sd(x)
[1] 3.40098
> sd(y)
[1] 4.630815
</code>
To compute the test statistic.
[ TS = frac{3.40^2}{4.63^2} = frac{11.56}{21.43} = 0.5394 ]



<code>
> var.test(x,y)

        F test to compare two variances

data:  x and y
F = 0.5394, num df = 9, denom df = 8, p-value = 0.3764
alternative hypothesis: true ratio of variances is not equal to 1
95 percent confidence interval:
 0.1237892 2.2125056
sample estimates:
ratio of variances
         0.5393782
</code>


         * The p-value is 0.3764, above the threshold level of 0.0250.
         * We fail to reject the null hypothesis.
         * We can assume that there is no significant difference in sample size.
         * Additionally the $95%$ confidence interval (0.1237, 2.2125) contains the null values i.e. 1.

