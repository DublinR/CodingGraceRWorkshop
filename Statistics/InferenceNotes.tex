{P-values}
Lets simulate 100 rolls of a die and sum up the numbers.
<p>
*  The command \color{blue}\texttt{runif(100,1,7)}\color{black} generates 100 random numbers uniformly distributed between 1 and 7.
*  The command \color{blue}\texttt{floor()}\color{black} discretizes those values.
*  The command \color{blue}\texttt{sum()}\color{black} computes the sum of those 100 values.
<p>


\texttt{R} Code:
\begin{verbatim}
>
> x<-sum(floor(runif(100,1,7)))
> x
[1] 357
>
\end{verbatim}

We could have used some other approachs in implementing this.
<p>
*  The command \color{blue}\texttt{runif(100,0,6)}\color{black} generates 100 random numbers uniformly distributed between 0 and 6.
*  The command \color{blue}\texttt{ceiling()}\color{black} discretizes those values.
*  The command \color{blue}\texttt{sum()}\color{black} computes the sum of those 100 values.
<p>


\texttt{R} Code:
\begin{verbatim}
>
> x<-sum(ceiling(runif(100,0,6)))
> x
[1] 372
>
\end{verbatim}
%


%-----------------------------------------------
\subsection{Statistical Inference}
<p>
*  $R$ commands for statistical inference procedures * 
t.test() - testing procedure for means.
<p>
*  One sample *  Two sample *  Paired
<p>
*  prop.test() - testing procedure for proportions.
<p>
*  One sample *  Two sample
<p>
*  var.test() - testing procedure for variances.
<p>
%




%-----------------------------------------------
\subsection{Single sample inference}

If we have a single sample we might want to answer several
questions:
<p>
*  What is the mean value? *  Is the mean value
significantly different from current theory? (Hypothesis test)
*  What is the level of uncertainty associated with our
estimate of the mean value? (Confidence interval)
<p>
To ensure that our analysis is correct we need to check for
outliers in the data (i.e. boxplots) and we also need to check
whether the data are normally distributed or not.
%

%-----------------------------------------------
\subsection{Checking normality}


Graphical methods are often used to check that the data being
analysed are normally distributed. We can use
<p>
*  Histogram - check for symmetry *  Boxplot - symmetry and
outliers *  Normal probability (Q-Q) plot

*  Other procedures
<p>
*  Kolmogorov-Smirnov test (ks.test())*  Shapiro Wilk test (shapiro.test()) * 
Grubb's test *  Anderson Darling test
<p>
<p>
We shall revert to these tests later.
%


%-----------------------------------------------------------------------------------------------------------------------%
\subsection{Hypothesis testing for a mean}

<p>
*  (Last week : confidence interval for a mean) *  Revision:
For large samples ($n > 30$) and/or if the population standard
deviation ($\sigma$) is known, the usual test statistic is given
by: \[Z =\frac{\bar{X} - \mu}{SE(\bar{X})}\]

*  $S.E.(\bar{X}) = { \sigma \over \sqrt{n}} $ or ${s \over \sqrt{n}}$. *  For small samples, use the $t-$distribution
with $n-1$ degrees of freedom.
* 
Critical value from tables.
*  Compare test statistics and critical values.
<p>


\texttt{R} Code:
\begin{verbatim}

> x = c(3, 0, 5, 2, 5, 5, 5, 4, 4, 5)
>qqnorm(x)
>qqline(x)

\end{verbatim}

%


\end{document}
