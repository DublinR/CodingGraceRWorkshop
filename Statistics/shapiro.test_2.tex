






frametitle{Shapiro-Wilk Test(a)}



 * We will often be required to determine whether or not a data set is normally distributed.
 * This assumption underpins many statistical models.
 * The null hypothesis is that the data set is normally distributed.
 * The alternative hypothesis is that the data set is not normally distributed.
 * One procedure for testing these hypotheses is the Shapiro-Wilk test, implemented in texttt{R} using the command texttt{shapiro.test()}.
 * (Remark: You will not be required to compute the test statistic for this test.)



textbf{IMPORTANT}

 * $H_0$ : The sample is drawn from a population that is normally distributed. 
 * $H_1$ : The sample is drawn from a population that is NOT normally distributed. 




For the data set used previously; $x$ and $y$, we use the Shapiro-Wilk test to determine that both data sets are normally distributed.
<code>

> shapiro.test(X)

        Shapiro-Wilk normality test

data:  x
W = 0.9474, p-value = 0.6378

> shapiro.test(y)

        Shapiro-Wilk normality test

data:  y
W = 0.9347, p-value = 0.5273
</code>



 * If we fail to reject $H_0$ we are saying that there is not enough evidence to contradict the null hypothesis. smallskip
 * This is not the same as saying that the data is a sample drawn from a normally distributioned population. smallskip
 * This is a common misconception.
 * Hypothesis Testing is about strength of evidence, not proof.

<p>

frametitle{Graphical Procedures for Assessing Normality}


 * The normal probability (Q-Q) plot is a very useful tool for determining whether or not a data set is normally distributed.
 * textbf{Important} Interpretation is simple. If the points follow the trendline, we can assume that that data set is a sample drawn from a normally distributed population.
%(provided by the second line of texttt{R} code texttt{qqline}).
 * One should expect minor deviations. Numerous major deviations would lead the analyst to conclude that the data set is not normally distributed.
 * The Q-Q plot is best used in conjunction with a formal procedure such as the Shapiro-Wilk test.


<pre>
<code>
>qqnorm(mySample)
>qqline(mySample)
</code>
</pre>
<p>


%-------------------------------------------------%


frametitle{Graphical Procedures for Assessing Normality}

begin{center}
includegraphics[scale=0.32]{10AQQplot}
end{center}




frametitle{Interpretation: Can NOT Assume Normal Distribution}

begin{center}
includegraphics[scale=0.6]{QQplot2}
end{center}




frametitle{Graphical Procedures for Assessing Normality}

begin{center}
includegraphics[scale=0.5]{QQplot1}
end{center}

 * Can assume textbf{Verbal SAT} is a normally distributed variable.
 * Can not assume textbf{GPA} is a normally distributed variable.

