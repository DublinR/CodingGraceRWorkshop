%-----------------------------------------------------------------------------------------------------------%
%--------------------------------------------------------------------------------------DATA FRAMES----------%
\frametitle{Data Frames}


Another way that information is stored is in data frames. This is a way to take many vectors of different types and store them in the same variable. The vectors can be of all different types. For example, a data frame may contain many lists, and each list might be a list of factors, strings, or numbers.

There are different ways to create and manipulate data frames. Most are beyond the scope of this introduction. They are only mentioned here to offer a more complete description.
\frametitle{Data Frames}
Technically, a data frame in R is a very important type of data object;  a type of table where the typical use employs the rows as observations (or cases) and the columns as variables.
Inter alia, a data frame differs from a matrix in that it can contain character values.
Many data sets are stored as data frames.
Let us consider the following two variables; age and height.
\begin{verbatim}
> age=18:29
> age
[1] 18 19 20 21 22 23 24 25 26 27 28 29
\end{verbatim}

In similar fashion, we entered the average heights in a vector called height.
\begin{verbatim}
>height=c(76.1,77,78.1,78.2,78.8,79.7,79.9,81.1,81.2,81.8,82.8,83.5)
> height
[1] 76.1 77.0 78.1 78.2 78.8 79.7 79.9 81.1 81.2 81.8 82.8 83.5
\end{verbatim}

We will now use R's data.frame() command to create our first data frame and store the results in the data frame "village".
\begin{verbatim}
> village=data.frame(age=age,height=height)
\end{verbatim}
\end{frame}
%%%%%%%%%%%%%%%%%%%%%%%%%%%%%%%%%%%%%%%%%%%%%%%%%%%%%%%%%%%%%%%%%%%%%%%%%%%%%%%%%%%%%%%%%%%%%%%%55
\begin{frame}[fragile]
How do we access the data in each column? One way is to state the variable containing the data frame, followed by a dollar sign, then the name of the column we wish to access (as with Lists earlier).
For example, if we wanted to access the data in the "age" column, we would do the following:
\begin{verbatim}
 > village$age
 [1] 18 19 20 21 22 23 24 25 26 27 28 29
\end{verbatim}
The additional typing required by the "dollar sign" notation can quickly become tiresome, so R provides the ability to "attach" the variables in the dataframe to our workspace.
> attach(village)

Let's re-examine our workspace. ( The \texttt{ls()} command lists all data objects in the workspace )
> ls()
[1] "village"

\end{frame}
%%%%%%%%%%%%%%%%%%%%%%%%%%%%%%%%%%%%%%%%%%%%%%%%%%%%%%%%%%%%%%%%%%%%%%%%%%%%%%%%%%%%%%%%%%%%%%%%55
\begin{frame}[fragile]
No evidence of the variables in the workspace. However, R has made copies of the variables in the columns of the data frame, and most importantly, we can access them without the "dollar notation."
\begin{verbatim}
> age
 [1] 18 19 20 21 22 23 24 25 26 27 28 29
> height
 [1] 76.1 77.0 78.1 78.2 78.8 79.7 79.9 81.1 81.2 81.8 82.8
[12] 83.5
\end{verbatim}

Previously we have seen rownames() and colnames() to determine the names from an existing data frame. We can use these commands to create names for a new data frame.

\end{frame}
