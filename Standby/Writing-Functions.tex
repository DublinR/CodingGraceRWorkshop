 \documentclass{beamer}
 
 \usepackage{amsmath}
 \usepackage{graphicx}
 
 \usepackage{framed}
 \usepackage{amssymb}
 
 \begin{document}
%-------------------------------------------------------------------------------%
\begin{frame}[fragile]{Writing Basic Functions}
\begin{itemize}
\item A simple function can be constructed as follows:
\begin{verbatim}

\end{verbatim}
\item \item You decide on the name of the function. The function
command shows R that you are writing a function. Inside the
parenthesis you outline the input objects required and decide what
to call them. The commands occur inside the { }. The name of
whatever output you want goes at the end of the function.
\end{itemize}
\end{frame}
%-------------------------------------------------------------------------------%

\begin{frame}[fragile]{More Complex Functions}
The following function returns several values in the form of a
list:
\begin{verbatim}
myfunc <- function(x)
{ # x is expected to be a numeric vector # function returns the
mean, sd, min, and max of the vector x the.mean <- mean(x) the.sd
<- sd(x)
the.min <- min(x)
the.max <- max(x)
return(list(average=the.mean,stand.dev=the.sd,minimum=the.min,
maximum=the.max)) }
\end{verbatim}


\end{frame}
%-------------------------------------------------------------------------------%

\begin{frame}[fragile]{Argument Matching}

\begin{itemize}
\item How does $R$ know which arguments are which? \item It uses argument
matching.
\item Argument matching is done in a few different ways.
\begin{itemize}
\item The arguments are matched by their positions. The first
supplied argument is matched to the first formal argument
and so on, e.g. when writing sf2 we specified that a1 comes
first, a2 second and a3 third. Using sf2(2, 3, 4) assigns 2
to a1, 3 to a2 and 4 to a3.
\item The arguments are matched by name. A named argument is
matched to the formal argument with the same name, e.g.
sf2 arguments have names a1, a2 and a3.\item Can do things like
sf2(a1=2, a3=3, a2=4), sf2(a3=2, a1=3, a2=4), etc.
\item Name matching happens first, then positional matching is
used for any unmatched arguments.
\end{itemize}
\end{itemize}
\end{frame}

%-------------------------------------------------------------------------------%
\begin{frame}[fragile]{Default values}
We can also give some/all arguments default values.
\begin{verbatim}
mypower <- function(x, pow=2)
{
x^pow
}
\end{verbatim}
If a value for the argument `pow' is not specified in the function call,
a value of 2 is used.
\begin{verbatim}
mypower(4)
[1] 16
\end{verbatim}
If a value for pow is specified, that value is used.
\begin{verbatim}
mypower(4, 3)  #x=4, pow=3
[1] 64
\end{verbatim}
\end{frame}

\end{document}
