%======================================================================================================== %
\subsection*{3. Beginning an analysis}


For the remainder of this tutorial, we will be analyzing the following dataset: 

Concentration
0.3330
0.1670
0.0833
0.0416
0.0208
0.0104
0.0052
Velocity
3.636
3.636
3.236
2.666
2.114
1.466
0.866

These data are measurements of the rate or velocity of a chemical reaction for different concentrations of substrate. We are interested in fitting a model of the relationship between concentration and velocity. The first thing we need to do is enter the data. We can do this from the Commands Window, from a spreadsheet interface, or by importing an existing file. Most of this tutorial will focus on command line input so let's begin there. At the prompt, create two vectors: 
> conc <- c(0.3330, 0.1670, 0.0833, 0.0416, 0.0208, 0.0104, 0.0052)
> vel <- c(3.636, 3.636, 3.236, 2.660, 2.114, 1.466, 0.866)
We use the function c(), which stands for concatenate, to create a vector. The individual elements can be numeric, as in this example, or character, or any other mode. However, all elements in a vector must be the same mode. Note that the elements are separated by commas. The spaces between elements are included for clarity and are not required by R. 
Now, we will combine these two vectors into a single data frame: 
> df <- data.frame(conc, vel)
Let's look at the data frame to be sure we entered the data correctly. To view any object in R, simply type its name: 

\begin{verbatim}
> df
    conc   vel
1 0.3330 3.636
2 0.1670 3.636
3 0.0833 3.236
4 0.0416 2.660
5 0.0208 2.114
6 0.0104 1.466
7 0.0052 0.866
\end{verbatim}
Oops. It looks like there is an error in one of the velocity entries. The fourth one down, 2.660, should be 2.666. You could use the up-arrow to recall the command you used to create vel, make the change, and run it again. Then use the up-arrow again to recall the command you used to create df and run that one again. Let's look at another way to do it. First, we'll change just that one element in the vector vel. Individual elements in a vector are referenced in square brackets. We want to change the fourth element, so we type: 
> vel[4] <- 2.666
Take a look at vel to see that it has been changed. There are a couple ways to change elements in a data frame. Treating the data frame as a matrix, we can reference the element in the fourth row and second column like this: 
> df[4,2] <- 2.666
Note the order that the row and column are referenced: first the row, then the column. Data frames also allow us to reference individual columns by their names. This is done with the name of the data frame, followed by a dollar sign, and the name of the column. So the vel data can be referenced as df$vel. Now we can change the fourth element like this: 
\begin{verbatim}
> df$vel[4] <- 2.666
\end{verbatimm}
Note that df$vel is a vector so we only need one number in the brackets. Let's take another look at df to be sure we have it right this time. 
\begin{framed}
\begin{verbatim}
> df
    conc   vel 
1 0.3330 3.636
2 0.1670 3.636
3 0.0833 3.236
4 0.0416 2.666
5 0.0208 2.114
6 0.0104 1.466
7 0.0052 0.866
\end{verbatim}
\end{framed}
Looks good! We could have created this data frame in other ways. If the data were already entered into a spreadsheet, database, or ASCII file, you could import it using the appropriate commands. Alternatively, you could create a data frame using a spreadsheet-like interface right in R. From the Edit choose Data Editor... and follow the prompts. You can use the fix() command to edit existing data frames. 
Now we're ready to do an analysis.

%======================================================================================================== %
4. Visualizing your data


The first thing we want to do is plot the data. You can easily get a basic plot with the following:

> plot(df$conc, df$vel)

The first vector given is the independent variable to be plotted on the X-axis, the second is the dependent variable for the Y-axis. If you execute a plotting function and there is no active graphsheet, a default graphsheet will be opened for you. If you then do another plot, the first one will be lost and the new plot will be drawn in the 
same graphics window. You can keep your first plot by opening another graphsheet by typing: 

\begin{verbatim}
> win.graph()
\end{verbatim}
The default is a square graphsheet. You can create portrait graphsheets, landscape graphsheets, any shape you want. Check win.graph in the help files to see how. 
We're going to look at 3 different ways to analyze these data: simple linear regression, non-linear regression, and polynomial regression. 

