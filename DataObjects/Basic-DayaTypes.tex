 \documentclass{beamer}
 
 \usepackage{amsmath}
 \usepackage{graphics}
 \usepackage{graphicx}
 \usepackage{framed}
 \usepackage{amssymb}
 
\begin{document}
%%%%%%%%%%%%%%%%%%%%%%%%%%%%%%%%%%%%%%%%%%%%%%%%%%%%%%%%%%%%%%%%%%%%%%%%

#### Introducton to R}

Introduction to R

 *   What is R
 *   History of R
\end{itemize}

Simple R Programming

 *   Important R functions
  *   help()
  *   summary()
  
\end{itemize}

  

%%%%%%%%%%%%%%%%%%%%%%%%%%%%%%%%%%%%%%%%%%%%%%%%%%%%%%%%%%%%%%%%%%%%%%%%

#### Introducton to R}

"summary" is a generic function used to produce result summaries of the results of various model fitting functions. 
The function invokes particular methods which depend on the class of the first argument. 

  *   attach()
  *   data()
   *   Loads specified data sets, or list the available data sets. 
  *   ?
  *   data.entry
  *   c()

\end{itemize}


%%%%%%%%%%%%%%%%%%%%%%%%%%%%%%%%%%%%%%%%%%%%%%%%%%%%%%%%%%%%%%%%%%%%%%%%

#### Introducton to R}


Combine Values into a Vector or List

  *   assignment operator
  *   Sys.time and Sys.Date returns the system's idea of the current date with and without time. 
  *   q()

\end{itemize}



%%%%%%%%%%%%%%%%%%%%%%%%%%%%%%%%%%%%%%%%%%%%%%%%%%%%%%%%%%%%%%%%%%%%%%%%

#### Introducton to R}


mode() *   storage mode of a data object( data structure)

The expression as(object, value) is the way to coerce an object to a particular class. 



%%%%%%%%%%%%%%%%%%%%%%%%%%%%%%%%%%%%%%%%%%%%%%%%%%%%%%%%%%%%%%%%%%%%%%%%

#### Introducton to R}


basic R editor

 *   new script
 *   updating script
 *   running script
 *  
\end{itemize}

 
 

%%%%%%%%%%%%%%%%%%%%%%%%%%%%%%%%%%%%%%%%%%%%%%%%%%%%%%%%%%%%%%%%%%%%%%%%

#### Introducton to R}

"summary" is a generic function used to produce result summaries of the results of various model fitting functions. 
The function invokes particular methods which depend on the class of the first argument. 


  *   attach()
  *   data()
   *   Loads specified data sets, or list the available data sets. 
  *   ?
  *   data.entry
  *   c()
*   Combine Values into a Vector or List
  *   assignment operator
  *   Sys.time and Sys.Date returns the system's idea of the current date with and without time. 
  *   q()
 


%%%%%%%%%%%%%%%%%%%%%%%%%%%%%%%%%%%%%%%%%%%%%%%%%%%%%%%%%%%%%%%%%%%%%%%%

#### Introducton to R : mode and class}

\begin{description}
 * [\texttt{mode()}]  storage mode of a data object( data structure)
* [\texttt{class()}] class
\end{description}


%%%%%%%%%%%%%%%%%%%%%%%%%%%%%%%%%%%%%%%%%%%%%%%%%%%%%%%%%%%%%%%%%%%%%%%%

#### Introducton to R}


vectors

vector produces a vector of the given length and mode.
as.vector, a generic, attempts to coerce its argument into a vector of mode mode (the default is to coerce to whichever mode is most convenient).
is.vector returns TRUE if x is a vector of the specified mode having no attributes other than names. It returns FALSE otherwise.  

%%%%%%%%%%%%%%%%%%%%%%%%%%%%%%%%%%%%%%%%%%%%%%%%%%%%%%%%%%%%%%%%%%%%%%%%

#### Introducton to R}
integer
Creates or tests for objects of type "integer".
integer(length = 0)
as.integer(x, ...)
is.integer(x)


%%%%%%%%%%%%%%%%%%%%%%%%%%%%%%%%%%%%%%%%%%%%%%%%%%%%%%%%%%%%%%%%%%%%%%%%

#### Introducton to R}

 Character
Create or test for objects of type "character".
character(length = 0)
as.character(x, ...)
is.character(x)



%%%%%%%%%%%%%%%%%%%%%%%%%%%%%%%%%%%%%%%%%%%%%%%%%%%%%%%%%%%%%%%%%%%%%%%%

#### Introducton to R}

logical
Create or test for objects of type "logical", and the basic logical constants.
logical(length = 0)
as.logical(x, ...)
is.logical(x)


%%%%%%%%%%%%%%%%%%%%%%%%%%%%%%%%%%%%%%%%%%%%%%%%%%%%%%%%%%%%%%%%%%%%%%%%

#### Introducton to R}

numeric
Creates or coerces objects of type "numeric". 
numeric(length = 0)
as.numeric(x, ...)
is.numeric(x)



%%%%%%%%%%%%%%%%%%%%%%%%%%%%%%%%%%%%%%%%%%%%%%%%%%%%%%%%%%%%%%%%%%%%%%%%

#### Introducton to R :  Lists}



   *   list() Functions to construct, coerce and check for both kinds of R lists. 
 *   vectors
 *   dataframes



%%%%%%%%%%%%%%%%%%%%%%%%%%%%%%%%%%%%%%%%%%%%%%%%%%%%%%%%%%%%%%%%%%%%%%%%

#### Introducton to R}

Packages


 *   What are pacakges
*   extending the program
 *   where are packages to be found
 *   some notable packages
 *   installing packages
 *   loading packages
 *   updating packages



%%%%%%%%%%%%%%%%%%%%%%%%%%%%%%%%%%%%%%%%%%%%%%%%%%%%%%%%%%%%%%%%%%%%%%%%
\end{document}
