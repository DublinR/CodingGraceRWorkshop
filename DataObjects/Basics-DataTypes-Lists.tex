
%%%%%%%%%%%%%%%%%%%%%%%%%%%%%%%%%%%%%%%%%%%%%%%%%%%%%%%%%%%%%%%%%%%%%%%%

#### Introducton to R : Lists}
Lists



*  are an ordered collection of components;
* components may be arbitrary R objects (data frames, vectors,lists, etc. );
*  single bracket notation for sublists;
* double bracket notation for individual components;
* construct using the function list().
\end{itemize}


%%%%%%%%%%%%%%%%%%%%%%%%%%%%%%%%%%%%%%%%%%%%%%%%%%%%%%%%%%%%%%%%%%%%%%%%

#### Introducton to R : Lists}
A simple example of a list is as follows:

begin{framed}
begin{verbatim}

L1 <- list(name="Fred",wife="Mary",no.children=3,
child.ages=c(4,7,9))

end{verbatim}
end{framed}

%%%%%%%%%%%%%%%%%%%%%%%%%%%%%%%%%%%%%%%%%%%%%%%%%%%%%%%%%%%%%%%%%%%%%%%%

#### Introducton to R : Lists}
Each component of the list is given a name (i.e. name, wife, no.children, child.ages).

We can construct a second list omitting the component names:

begin{framed}
begin{verbatim}

L2 <- list("Fred", "Mary", 3, c(4,7,9))

end{verbatim}
end{framed}

%%%%%%%%%%%%%%%%%%%%%%%%%%%%%%%%%%%%%%%%%%%%%%%%%%%%%%%%%%%%%%%%%%%%%%%%

#### Introducton to R : Lists}
What is the difference between the two? Three equivalent ways of accessing the first component.

begin{framed}
begin{verbatim}

L1[["name"]]
L1$name
L1[[1]]


end{verbatim}
end{framed}

%%%%%%%%%%%%%%%%%%%%%%%%%%%%%%%%%%%%%%%%%%%%%%%%%%%%%%%%%%%%%%%%%%%%%%%%

#### Introducton to R : Lists}

A sublist consisting of the first component only
> L1[1]
$name
[1] "Fred"



%%%%%%%%%%%%%%%%%%%%%%%%%%%%%%%%%%%%%%%%%%%%%%%%%%%%%%%%%%%%%%%%%%%%%%%%

#### Introducton to R : Lists}
The names of each component of the list can be accessed using

\begin{framed}
\begin{verbatim}

names(L1)
names(L2)

\end{verbatim}
\end{framed}

%%%%%%%%%%%%%%%%%%%%%%%%%%%%%%%%%%%%%%%%%%%%%%%%%%%%%%%%%%%%%%%%%%%%%%%%

#### Introducton to R : Lists}

We can set the names for the list components after the list has been
created.

\begin{framed}
\begin{verbatim}

names(L2) <- c("name.hus", "name.wife", "no.child", "child.age")
names(L2)

\end{verbatim}
\end{framed}

%%%%%%%%%%%%%%%%%%%%%%%%%%%%%%%%%%%%%%%%%%%%%%%%%%%%%%%%%%%%%%%%%%%%%%%%

#### Introducton to R : Lists}

We can also concatenate lists:

begin{framed}
begin{verbatim}

L3 <- c(L1, L2)

end{verbatim}
end{framed}


%%%%%%%%%%%%%%%%%%%%%%%%%%%%%%%%%%%%%%%%%%%%%%%%%%%%%%%%%%%%%%%%%%%%%%%%
