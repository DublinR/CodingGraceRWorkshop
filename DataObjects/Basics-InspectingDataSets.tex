

\frametitle{Basic R editing – Script Editor and \texttt{edit()}}
R has an inbuilt script editor. We will use it for this class, but there are plenty of top quality “Integrated Development Environments “ out there. (Read up about RStudio for  example.)

To start a new script, or open an existing script simply go to “File” on the menu bar and click the appropriate options.
A new dialogue box will appear. You can write and edit code using this editor.
To pass the code for compiling – press the “run line or selection” option (The third icon on the menu).
Another way to edit code is to use the \texttt{edit()} function – which operates directly from the command line.  To edit the code defining an object X, simply type edit(X).
\end{frame}
%------------------------------------------------------%

\frametitle{Changing GUI options}
We can change the GUI options using the “GUI preferences” option on the Edit menu. (Important when teaching R)
A demonstration will be done in class. 
%------------------------------------------------------%
\end{frame}
%------------------------------------------------------%

\frametitle{Embedded Datasets}
Several data sets – intended as learning tools – are automatically installed when \texttt{R} is installed. Many more are installed within packages to complement learning to use those packages.  One of these is the famous ``\textit{iris}" data set, which is used in many data mining exercises.
\begin{itemize}
\item To see what data sets are available – type \texttt{data()}.
\item To load a data set, simply type in the name of the data set. 
\item To specify that a specific data set is to be used for analysis, use the command \texttt{attach()}.

\end{itemize}
\end{frame}
%------------------------------------------------------%

Some data sets are very large. To just see the first few rows, we use the head() function.
{
\small
\begin{framed}
\begin{verbatim}
> head(iris)
  Sepal.Length Sepal.Width Petal.Length Petal.Width Species
1          5.1         3.5          1.4         0.2  setosa
2          4.9         3.0          1.4         0.2  setosa
3          4.7         3.2          1.3         0.2  setosa
4          4.6         3.1          1.5         0.2  setosa
5          5.0         3.6          1.4         0.2  setosa
6          5.4         3.9          1.7         0.4  setosa
\end{verbatim}
\end{framed}
}
\end{frame}

%------------------------------------------------------%


### Working directories
You can change your working directly by using the appropriate options on the File menu.
To determine the current working directory – you can use the \texttt{getwd()} command:	
\begin{framed}
\begin{verbatim}
> getwd()
[1] "C:/Users/Kevin/Documents"
\end{verbatim}
\end{framed}
\end{frame}

%------------------------------------------------------%

To change the working directory – we would use the \texttt{setwd()} command.
\begin{framed}
\begin{verbatim}
> getwd()
[1] "C:/Users/Kevin"
>
> setwd("C:/Users/Kevin/Documents")
>
> getwd()
[1] "C:/Users/Kevin/Documents"
\end{verbatim}
\end{framed}
\end{frame}

### Removing items

Sometimes it is desirable to save a subset of your workspace instead of the entire workspace. One option is to use the \texttt{rm()} function to remove unwanted objects right before exiting your \texttt{R} session; another possibility is to use the save function. 

%------------------------------------------------------%

The save function accepts multiple arguments to specify the objects you wish to save, or, alternatively, a character vector with the names of the objects can be passed to save through the \texttt{list=} argument. 

Once the objects to be saved are specified, the only other required option is the \texttt{file=option}, specifying the destination of the saved R object. Although there is no requirement to do so, it is common to use a suffix of \texttt{.rda} or \texttt{.RData} for saved \texttt{R} workspace files.

For example, to save the \texttt{R} objects x, y, and z to a file called \texttt{myData.rda} ,the following statements could be used:

\begin{verbatim}
> save(x,y,z,fil= ‘mydata.rda’)
\end{verbatim}
\end{frame}

Whenever \texttt{R} encounters such a file in the working directory at the beginning of a session, it automatically loads it making all your saved objects available again. 
So one method for saving your work is to always save your workspace image at the end of an R session. If you’d like to save your workspace image at some other time during your R session, you can use the \texttt{save.image()} function, which, when called with no arguments, will also save the current workspace to a file called .RData in the working directory.
\end{frame}
%---------------------------------------------------------------------------------------------%

\frametitle{Quitting the R environment}
As the front page text indicates – all you have to do to quite the workspace is to type in \texttt{q()}. You will then be prompted to save your work.
\end{frame}
%---------------------------------------------------------------------------------------------%


\frametitle{Coming unstuck}
If you are having trouble with a piece of code that is currently compiling – all you have to do is press ESC. Just like other computing environments.
\end{frame}
%------------------------------------------------------%

\frametitle{Listing all items in a workspace}
To list all items in an \texttt{R} environment, we use the \texttt{ls()} function. This provides a list of all data objects  accessible.
\begin{framed}
\begin{verbatim}
> ls()
 [1] "a"          "A"          "authors"    "b"          "books"     
 [6] "C"          "D"          "ex1"        "Gerb"       "Lst"       
[11] "m"          "m1"         "op"         "presidents" "r"         
[16] "showSmooth" "sm"         "sm.3RS"     "sm2"        "sm3"       
[21] "Trig"       "Vec1"       "x"          "X"          "x.at"      
[26] "x1"         "x2"         "x3R"        "y"          "Y"         
[31] "y.at"      
\end{verbatim}
\end{framed}
\end{frame}


