\documentclass[a4paper,12pt]{article}
%%%%%%%%%%%%%%%%%%%%%%%%%%%%%%%%%%%%%%%%%%%%%%%%%%%%%%%%%%%%%%%%%%%%%%%%%%%%%%%%%%%%%%%%%%%%%%%%%%%%%%%%%%%%%%%%%%%%%%%%%%%%%%%%%%%%%%%%%%%%%%%%%%%%%%%%%%%%%%%%%%%%%%%%%%%%%%%%%%%%%%%%%%%%%%%%%%%%%%%%%%%%%%%%%%%%%%%%%%%%%%%%%%%%%%%%%%%%%%%%%%%%%%%%%%%%
\usepackage{eurosym}
\usepackage{vmargin}
\usepackage{amsmath}
\usepackage{graphics}
\usepackage{epsfig}
\usepackage{subfigure}
\usepackage{fancyhdr}
\usepackage{framed}

\setcounter{MaxMatrixCols}{10}
%TCIDATA{OutputFilter=LATEX.DLL}
%TCIDATA{Version=5.00.0.2570}
%TCIDATA{<META NAME="SaveForMode"CONTENT="1">}
%TCIDATA{LastRevised=Wednesday, February 23, 201113:24:34}
%TCIDATA{<META NAME="GraphicsSave" CONTENT="32">}
%TCIDATA{Language=American English}

\pagestyle{fancy}
\setmarginsrb{20mm}{0mm}{20mm}{25mm}{12mm}{11mm}{0mm}{11mm}

%
%
\lhead{Dublin R} \rhead{R Commands} \chead{R Computing} %\input{tcilatex}
%
%
\begin{document}

\tableofcontents
%http://www.calvin.edu/~scofield/courses/m143/materials/RcmdsFromClass.pdf
%http://www-stat.stanford.edu/~susan/courses/s141/RNotes.pdf
%--------------------------------------------------%
\newpage
\section{\texttt{R} Commands}
\subsection{\texttt{addmargins()}}
\begin{framed}
\begin{verbatim}
> cbind(c(10,140,300),c(30,24,12))->Xdat
> Xdat
     [,1] [,2]
[1,]   10   30
[2,]  140   24
[3,]  300   12
> addmargins(Xdat)
     [,1] [,2] [,3]
[1,]   10   30   40
[2,]  140   24  164
[3,]  300   12  312
[4,]  450   66  516
\end{verbatim}
\end{framed}
\subsection{\texttt{ecdf()}}
\begin{framed}
\begin{verbatim}
plot(ecdf(iris[,1]),verticals=TRUE, col.points='red',cex=0.4, col.hor='red', col.vert='bisque')
\end{verbatim}
\end{framed}
%--------------------------------------------------%
\subsection{\texttt{edit()}}
\subsection{\texttt{edit()}}
\subsection{\texttt{jitter()}}
\subsection{\texttt{merge()}}
\subsection{\texttt{scan()}}
\subsection{\texttt{scale()}}
% scale() 
% converts a data frame to standardized scores
\subsection{\texttt{sweep()}}
\subsection{\texttt{Sys.Time()} and \texttt{Sys.Date()} }
\subsection{\texttt{table()}}
\subsection{\texttt{View()}}
\subsection{\texttt{which()}}
%--------------------------------------------------%
\newpage



\end{document}
%--------------------------------------------------%



