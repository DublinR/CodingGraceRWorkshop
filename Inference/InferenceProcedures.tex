	\documentclass[a4paper,12pt]{article}
%%%%%%%%%%%%%%%%%%%%%%%%%%%%%%%%%%%%%%%%%%%%%%%%%%%%%%%%%%%%%%%%%%%%%%%%%%%%%%%%%%%%%%%%%%%%%%%%%%%%%%%%%%%%%%%%%%%%%%%%%%%%%%%%%%%%%%%%%%%%%%%%%%%%%%%%%%%%%%%%%%%%%%%%%%%%%%%%%%%%%%%%%%%%%%%%%%%%%%%%%%%%%%%%%%%%%%%%%%%%%%%%%%%%%%%%%%%%%%%%%%%%%%%%%%%%
\usepackage{eurosym}
\usepackage{vmargin}
\usepackage{amsmath}
\usepackage{framed}
\usepackage{graphics}
\usepackage{epsfig}
\usepackage{subfigure}
\usepackage{enumerate}
\usepackage{fancyhdr}

\setcounter{MaxMatrixCols}{10}
%TCIDATA{OutputFilter=LATEX.DLL}
%TCIDATA{Version=5.00.0.2570}
%TCIDATA{<META NAME="SaveForMode"CONTENT="1">}
%TCIDATA{LastRevised=Wednesday, February 23, 201113:24:34}
%TCIDATA{<META NAME="GraphicsSave" CONTENT="32">}
%TCIDATA{Language=American English}

\pagestyle{fancy}
\setmarginsrb{20mm}{0mm}{20mm}{25mm}{12mm}{11mm}{0mm}{11mm}
\lhead{Statistics with \texttt{r}} \rhead{Kevin O'Brien} \chead{Chi-Square Test} %\input{tcilatex}

\begin{document}



\section{Test for Equality of Variance and Means}

\begin{itemize}
\item Test for Equality of Test (\texttt{var.test()})
\item Welch Two Sample \emph{t-}test (\texttt{t.test()})
\item Independent Two Sample \emph{t-}test (\texttt{t.test(var.equal=TRUE)})

\end{itemize}


\newpage
\section{Outliers}
\subsection{Grubb's Test for Outliers}
\begin{framed}
	\begin{verbatim}
	library(outliers)
	grubbs.test(X)
	\end{verbatim}
\end{framed}
\subsection{Dixon Test for Outliers}
\subsection{Outliers on Boxplots}
Boxplots can used to determine an outlie (we will refer to them as ``boxplot outliers")
\begin{framed}
	\begin{verbatim}
	boxplot(X, horizontal = TRUE)
	\end{verbatim}
\end{framed}

\newpage

\section{Inference Procedures}
\subsection{Confidence Interval }
A confidence interval gives an estimated range of values which is likely to include an unknown population parameter, the estimated range being calculated from a given set of sample data. If independent samples are taken repeatedly from the same population, and a confidence interval calculated for each sample, then a certain percentage (confidence level) of the intervals will include the unknown population parameter. 

Confidence intervals are usually calculated so that this percentage is $95\%$, but we can produce $90\%$, $99\%$, $99.9\%$ (or whatever) confidence intervals for the unknown parameter. The width of the confidence interval gives us some idea about how uncertain we are about the unknown parameter. A very wide interval may indicate that more data should be collected before anything very definite can be said about the parameter.
\subsection{Power }
The power of a statistical hypothesis test measures the test's ability to reject the null hypothesis when it is actually false - that is, to make a correct decision. In other words, the power of a hypothesis test is the probability of not committing a type II error. It is calculated by subtracting the probability of a type II error from 1, usually expressed as: 
\[\mbox{Power} = 1 - \mbox{P(type II error) } = 1- \beta \]The maximum power a test can have is 1, the minimum is 0. Ideally we want a test to have high power, close to 1.

\section{Single Sample Inference Procedures}
If we have a single sample we might want to answer several
questions:
\begin{itemize}
	\item What is the mean value? \item Is the mean value
	significantly different from current theory? (Hypothesis test)
	\item What is the level of uncertainty associated with our
	estimate of the mean value? (Confidence interval)
\end{itemize}

\begin{itemize}
	\item (Last week : confidence interval for a mean) \item Revision:
	For large samples ($n > 30$) and/or if the population standard
	deviation ($\sigma$) is known, the usual test statistic is given
	by: \[Z =\frac{\bar{X} - \mu}{SE(\bar{X})}\]
	
	\item $S.E.(\bar{X}) = { \sigma \over \sqrt{n}} $ or ${s \over \sqrt{n}}$. 
	\item For small samples, use the $t-$distribution with $n-1$ degrees of freedom.
	\item Critical value from tables.
	\item Compare test statistics and critical values.
\end{itemize}

To ensure that our analysis is correct we need to check for
outliers in the data (i.e. boxplots) and we also need to check
whether the data are normally distributed or not.

\begin{framed}
	\begin{verbatim}
	> t.test(X,mu=10)
	
	One Sample t-test
	
	data:  X 
	t = 14.1421, df = 4, p-value = 0.0001451
	alternative hypothesis: true mean is not equal to 10 
	95 percent confidence interval:
	10.08037 10.11963 
	sample estimates:
	mean of x 
	10.1 
	\end{verbatim}
\end{framed}


%--------------------------------------------------------------------------------------------------%


%--------------------------------------------------------------------------%
\newpage
%section 9 Inference Procedures



%------------------------------------------------------%
\subsubsection{Two Sample Tests}


All of the previous hypothesis tests and confidence intervals can be
extended to the two-sample case.

The same assumptions apply, i.e. data are normally distributed in
each population and we may want to test if the mean in one
population is the same as the mean in the other population, etc.

Normality can be checked using histograms, boxplots and Q-Q
plots as before. The Anderson-Darling test can be used on
each group of data also.


%------------------------------------------------------%
\subsubsection{Implementation}

This can be carried out in R by hand:

\footnotesize \begin{verbatim}
>obs.vals <- matrix(c(43,9,44,4), nrow=2, byrow=T)
>row.tots <- apply(obs.vals, 1, sum)
>col.tots <- apply(obs.vals, 2, sum)
>exp.vals <- row.tots%o%col.tots/sum(obs.vals)
>TS <- sum((obs.vals-exp.vals)^2/exp.vals)
>TS
>[1] 1.777415
\end{verbatim}\normalsize



\end{document}
