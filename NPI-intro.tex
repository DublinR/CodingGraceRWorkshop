


\section{Non-Parametric Tests}
Many statistical tests assume that you have sampled data from populations that follow a Normal distribution. 
Biological data never follow a Gaussian distribution precisely, because a Gaussian distribution extends infinitely in both directions, and so it includes both infinitely low negative numbers and infinitely high positive numbers. But many kinds of biological data follow a bell-shaped distribution that is approximately Gaussian. 

Because statistical tests work well even if the distribution is only approximately Gaussian (especially with large samples), these tests are used routinely in many fields of science.

An alternative approach does not assume that data follow a Gaussian distribution. These tests, called nonparametric tests, are appealing because they require fewer assumptions about the distribution of the data. In this approach, values are ranked from low to high, and the analyses are based on the distribution of ranks.
Often, the analysis will be one of a series of experiments. Since you want to analyze all the experiments the same way, you cannot rely on the results from a single normality test.
