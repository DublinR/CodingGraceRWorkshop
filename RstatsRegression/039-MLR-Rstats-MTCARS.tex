
<p>
#### {Example: MTCars}
The data was extracted from the 1974 Motor Trend US magazine, and comprises fuel consumption and 10 aspects of automobile design and performance for 32 automobiles (1973�74 models).

A data frame with 32 observations on 11 variables. Let
us assume that the variable mpg is the response variable, with the other ten being predictor variables.

begin{ }
          * textbf{mpg}  Miles/(US) gallon
          * textbf{cyl}  Number of cylinders
          * textbf{disp}  Displacement (cu.in.)
          * textbf{hp}  Gross horsepower
          * textbf{drat}  Rear axle ratio
          * textbf{wt}  Weight (lb/1000)
          * textbf{qsec}  1/4 mile time
          * textbf{vs}  V/S
          * textbf{am}  Transmission (0 = automatic, 1 = manual)
          * textbf{gear}  Number of forward gears
          * textbf{carb}  Number of carburetors
end{ }

<p>
#### {Model specification and output}
Specification of a multiple regression analysis is done by setting up a
model formula with + between the explanatory variables:
<code>
lm(mpg~cyl+disp+hp+drat+wt+qsec+vs+am+gear+carb, data=mtcars)
</code>

which is meant to be read as "mpg is described using a model that
is additive in cyl, disp, and so forth.� The output is as follows:

<code>
Coefficients:
            Estimate Std. Error t value Pr(>|t|)
(Intercept) 12.30337   18.71788   0.657   0.5181
cyl         -0.11144    1.04502  -0.107   0.9161
disp         0.01334    0.01786   0.747   0.4635
hp          -0.02148    0.02177  -0.987   0.3350
drat         0.78711    1.63537   0.481   0.6353
wt          -3.71530    1.89441  -1.961   0.0633 .
qsec         0.82104    0.73084   1.123   0.2739
vs           0.31776    2.10451   0.151   0.8814
am           2.52023    2.05665   1.225   0.2340
gear         0.65541    1.49326   0.439   0.6652
carb        -0.19942    0.82875  -0.241   0.8122
---
Signif. codes:  0 �                   � 0.001 �             *  * � 0.01 �       * � 0.05 �.� 0.1 � � 1

Residual standard error: 2.65 on 21 degrees of freedom
Multiple R-squared: 0.869,      Adjusted R-squared: 0.8066
F-statistic: 13.93 on 10 and 21 DF,  p-value: 3.793e-07
</code>
Notice that none of the predictor variables is significant. The only one that comes even close is ``wt".

small
<code>
       mpg   cyl  disp    hp  drat    wt  qsec    vs    am  gear  carb
mpg   1.00 -0.85 -0.85 -0.78  0.68 -0.87  0.42  0.66  0.60  0.48 -0.55
cyl  -0.85  1.00  0.90  0.83 -0.70  0.78 -0.59 -0.81 -0.52 -0.49  0.53
disp -0.85  0.90  1.00  0.79 -0.71  0.89 -0.43 -0.71 -0.59 -0.56  0.39
hp   -0.78  0.83  0.79  1.00 -0.45  0.66 -0.71 -0.72 -0.24 -0.13  0.75
drat  0.68 -0.70 -0.71 -0.45  1.00 -0.71  0.09  0.44  0.71  0.70 -0.09
wt   -0.87  0.78  0.89  0.66 -0.71  1.00 -0.17 -0.55 -0.69 -0.58  0.43
qsec  0.42 -0.59 -0.43 -0.71  0.09 -0.17  1.00  0.74 -0.23 -0.21 -0.66
vs    0.66 -0.81 -0.71 -0.72  0.44 -0.55  0.74  1.00  0.17  0.21 -0.57
am    0.60 -0.52 -0.59 -0.24  0.71 -0.69 -0.23  0.17  1.00  0.79  0.06
gear  0.48 -0.49 -0.56 -0.13  0.70 -0.58 -0.21  0.21  0.79  1.00  0.27
carb -0.55  0.53  0.39  0.75 -0.09  0.43 -0.66 -0.57  0.06  0.27  1.00
</code>
normalsize

In many cases there is a high degree of correlation between two predictor variables. The variable ``disp" has correlation coefficients of $-0.85$, $0.79$ and $0.89$ and with ``cyl", ``hp" and ``wt" respectively.

end{document}
