
frametitle{Shapiro-Wilk Test(a)}


begin{ }
         * We will often be required to determine whether or not a data set is normally distributed.
         * Again, this assumption underpins many statistical models.
         * The null hypothesis is that the data set is normally distributed.
         * The alternative hypothesis is that the data set is not normally distributed.
         * One procedure for testing these hypotheses is the Shapiro-Wilk test, implemented in texttt{R} using the command texttt{shapiro.test()}.
         * (Remark: You will not be required to compute the test statistic for this test.)
end{ }
end{frame}
%----------------------------------------%
begin{frame}[fragile]
frametitle{Shapiro Wilk Test(b)}
For the data set used previously; $x$ and $y$, we use the Shapiro-Wilk test to determine that both data sets are normally distributed.
<code>

> shapiro.test(x)

        Shapiro-Wilk normality test

data:  x
W = 0.9474, p-value = 0.6378

> shapiro.test(y)

        Shapiro-Wilk normality test

data:  y
W = 0.9347, p-value = 0.5273
</code>

end{frame}


%-------------------------------------------------%
begin{frame}[fragile]
frametitle{Graphical Procedures for assessing Normality}

begin{ }
         * The normal probability (Q-Q) plot is a very useful tool for determining whether or not a data set is normally distributed.
         * Interpretation is simple. If the points follow the trendline (provided by the second line of texttt{R} code texttt{qqline}).
         * One should expect minor deviations. Numerous major deviations would lead the analyst to conclude that the data set is not normally distributed.
         * The Q-Q plot is best used in conjunction with a formal procedure such as the Shapiro-Wilk test.
end{ }

<code>
>qqnorm(CWdiff)
>qqline(CWdiff)
</code>

end{frame}

%-------------------------------------------------%

begin{frame}
frametitle{Graphical Procedures for Assessing Normality}

begin{center}
includegraphics[scale=0.32]{10AQQplot}
end{center}
end{frame}

end{document}
