documentclass[a4]{beamer}
usepackage{amssymb}
usepackage{graphicx}
usepackage{subfigure}
usepackage{newlfont}
usepackage{amsmath,amsthm,amsfonts}
%usepackage{beamerthemesplit}
usepackage{pgf,pgfarrows,pgfnodes,pgfautomata,pgfheaps,pgfshade}
usepackage{mathptmx}  % Font Family
usepackage{helvet}   % Font Family
usepackage{color}

mode<presentation> {
 usetheme{Default} % was
 useinnertheme{rounded}
 useoutertheme{infolines}
 usefonttheme{serif}
 %usecolortheme{wolverine}
% usecolortheme{rose}
usefonttheme{structurebold}
}

setbeamercovered{dynamic}

title[MA4413]{Statistics for Computing  {normalsize Lecture 10A}}
author[Kevin O'Brien]{Kevin O'Brien  {scriptsize Kevin.obrien@ul.ie}}
date{Autumn Semester 2012}
institute[Maths & Stats]{Dept. of Mathematics & Statistics,  University textit{of} Limerick}

renewcommand{arraystretch}{1.5}

begin{document}


begin{frame}
titlepage
end{frame}

%---------%
begin{frame}
frametitle{Using texttt{R} for Inference Procedures}
begin{ }
        * [1] Paired t-test
        * [2] Single Sample Tests for Proportions
        * [3] Two Sample Test for Proportions
        * [4] Test for the equality of variances for two samples
        * [5] Shapiro-Wilk Test for Normality
        * [6] Graphical procedures for assessing normality
end{ }
end{frame}
%---------%
begin{frame}
frametitle{p-values using texttt{R}}
begin{ }
         * In every inference procedure performed using texttt{R}, a p-value is presented to the screen for the user to interpret.

         * If the p-value is larger than a specified threshold $alpha/k$ then the appropriate conclusion is a
failure to reject the null hypothesis.

         * Conversely, if the p-value is less than threshold, the appropriate conclusion is to reject the null hypothesis.

         * In this module, we will use a significance level$alpha=0.05$ and almost always the procedures will be two tailed ($k=2$). Therefore the threshold $alpha/k$ will be $0.025$.
end{ }
end{frame}

%---------%
begin{frame}
frametitle{Using Confidence Limits}
begin{ }
         * Alternatively, we can use the confidence interval to make a decision on whether or not we should reject or fail to reject the null hypothesis.
         * If the null value is within the range of the confidence limits, we fail to reject the null hypothesis.
         * If the null value is outside the range of the confidence limits, we reject the null hypothesis.
         * Occasionally a conclusion based on this approach may differ from a conclusion based on the p-value. In such a case, remark upon this discrepancy.
end{ }
end{frame}
%------------------------------------------%

begin{frame}
frametitle{The paired t-test (a)}

begin{ }
         * Previously we have seen the paired t-test. It is used to determine whether or
not there is a significant difference between paired measurements. Equivalently whether or not
the case-wise differences are zero.
         * The mean and standard deviation of the case-wise differences are used to determine the test statistic.
         * Under the null hypothesis, the expected value of the case-wise differences is zero (i.e $H_0 : mu_d = 0$).
[ TS = frac{bar{d} - mu_d}{frac{s_d}{sqrt{n}}} ]
end{ }
end{frame}


%------------------------------------------%
begin{frame}[fragile]
frametitle{The paired t-test (b)}
begin{ }
         * The calculation is dependent on the case-wise differences
         * Here the case-wise differences between paired measurements (e.g. ``before" and ``after")
         * The test statistic can be computed using the following texttt{R} code, where texttt{CWdiff} is the case-wise differences.          * The mean, standard deviation and sample size are presented in the next slide.
end{ }
end{frame}

%------------------------------------------%
begin{frame}[fragile]
frametitle{The paired t-test (c)}
<code>
> CWdiff = Before - After
> mean(CWdiff)
[1] 8.214286
> sd(CWdiff)
[1] 7.904999
> length(CWdiff)
[1] 14
</code>

[ TS = frac{8.21 - 0}{frac{7.90}{sqrt{14}}} = 3.881 ]
In an exam situation, the candidate will be expected to compute this value. It will be omitted from texttt{R} code output.
end{frame}

%------------------------------------------%
begin{frame}[fragile]
frametitle{The paired t-test (d)}
Implementing the paired t-test using texttt{R}.
<code>
> t.test(Before,After,paired=TRUE)

        Paired t-test

data:  Before and After
t = 3.8881, df = 13, p-value = 0.001868
alternative hypothesis: true difference in means is not 0
95 percent confidence interval:
  3.650075 12.778496
sample estimates:
mean of the differences
               8.214286

</code>
end{frame}

%-------------------------------------------------%
begin{frame}[fragile]
frametitle{The paired t-test (e)}
begin{ }
         * The p-value ($0.001868$) is less than the threshold is less than the threshold $0.0250$
         * We reject the null hypothesis (mean of case-wise differences being zero, i.e. expect no difference between ``before" and ``after")
         * We conclude that there is a difference between `before' and `after'.
         * That is to say, we can expected a difference between two paired measurements.
end{ }
end{frame}
%-------------------------------------------------%
begin{frame}[fragile]
frametitle{The paired t-test (f)}
begin{ }
         * We consider the confidence interval. We are $95%$ confident that the expected value of the case-wise difference is at least 3.65.
         * Here the null value (i.e. 0) is not within the range of the confidence limits.
         * Therefore we reject the null hypothesis.
end{ }
<code>
> t.test(Before,After,paired=TRUE)
...
...
95 percent confidence interval:
  3.650075 12.778496
...
</code>
end{frame}
%-------------------------------------------------%
begin{frame}[fragile]
frametitle{Single Sample Proportion Test (a)}

begin{ }
         * In this procedure, we determine whether or not we are are justified in assuming that the population proportion takes a certain value.
         * For example, suppose we believed that the population proportion of students with iphones or androids was $80%$.
         * We would write the null and alternative accordingly.
[H_0 : pi = 80% ]
[H_1 : pi neq 80% ]
         * The  appropriate texttt{R} command is texttt{prop.test(x,n,p)}
         * $x$ is the number of successes, $n$ is the sample size and $p$ is the population proportion assumed under the null hypothesis.
         * Suppose we survey 65 students, with 50 replying that they had an iphone or android.
end{ }

end{frame}
%-------------------------------------------------%

begin{frame}[fragile]
frametitle{Single Sample Proportion Test (b)}

<code>
> prop.test(50,65,0.80)

        1-sample proportions test 

data:  50 out of 65, null probability 0.8
X-squared = 0.2163, df = 1, p-value = 0.6418
alternative hypothesis: true p is not equal to 0.8
95 percent confidence interval:
 0.6452269 0.8610191
sample estimates:
        p
0.7692308
</code>

end{frame}
%-------------------------------------------------%

begin{frame}[fragile]
frametitle{Single Sample Proportion Test (c)}

begin{ }
         * The p-value is above the threshold. Therefore we fail to reject the null hypothesis that the population proportion ($pi$) is $80%$.

         * The observed proportion is a very straightforward calculation:

[ hat{p} = frac{50}{65} = 0.76923= 76.92%]
         * Nonetheless, you would be required to show how it was calculated.
end{ }

end{frame}

%-------------------------------------------------%
begin{frame}[fragile]
frametitle{Test of Equality for Two Sample Proportions (a)}
The null hypothesis is that two populations have the same proportions for a particular characteristic.
[H_0 : pi_1 = pi_2 ]
[H_1 : pi_1 neq pi_2 ]
begin{ }
         * The command is texttt{prop.test(c(x1,x2),c(n1,n2))}
         * $x1$ and $x2$ are the number of successes from both samples.
         * $n2$ and $n2$ are the sample sizes for both groups.
         * The difference in population proportions assumed under the null hypothesis is zero.
         * (It is possible to specify a different null value, but we will not consider this in this module.)
end{ }
end{frame}

%-------------------------------------------------%
begin{frame}
frametitle{Test of Equality for Two Sample Proportions (b)}
begin{ }
         * Consider a study where the proportion of Irish students who owned mobile devices, such as iphones and androids was compared to the corresponding proportion of French student.
         * As before, $65$ Irish students were interviewed, with $50$ replying that they owned mobile devices.
         * $90$ french students were interview, with 60 responding that they owned mobile devices.
         * The test of equality of proportions is implemented on the next slide.
end{ }
end{frame}

%-------------------------------------------------%


begin{frame}[fragile]
frametitle{Test of Equality for Two Sample Proportions (c)}
Based on the p-value, we fail to reject the null hypothesis. There is not enough evidence to assume a difference in proportions. Also the expected difference assumed under the null hypothesis, i.e. 0, is contained in the confidence interval.
<code>
> prop.test(c(50,60),c(65,90))

        2-sample test for equality of proportions

data:  c(50, 60) out of c(65, 90)
X-squared = 1.4613, df = 1, p-value = 0.2267
alternative hypothesis: two.sided
95 percent confidence interval:
 -0.05202058  0.25714878
sample estimates:
   prop 1    prop 2
0.7692308 0.6666667
</code>
end{frame}
begin{frame}[fragile]
frametitle{Test of Equality for Two Sample Proportions (d)}
begin{ }
         * You would be required to compute the differences in observed proportions.
         * Additionally you will given the texttt{R} code for one sample procedures. This may or may not be relevant for answering the question.
end{ }
<code>
> prop.test(60,90,0.80)
...
...
X-squared = 9.184, df = 1, p-value = 0.002441
alternative hypothesis: true p is not equal to 0.8
95 percent confidence interval:
 0.5585219 0.7604058
sample estimates:
        p
0.6666667
</code>

end{frame}



%-------------------------------------------------%


begin{frame}
frametitle{Shapiro-Wilk Test(a)}


begin{ }
         * We will often be required to determine whether or not a data set is normally distributed.
         * Again, this assumption underpins many statistical models.
         * The null hypothesis is that the data set is normally distributed.
         * The alternative hypothesis is that the data set is not normally distributed.
         * One procedure for testing these hypotheses is the Shapiro-Wilk test, implemented in texttt{R} using the command texttt{shapiro.test()}.
         * (Remark: You will not be required to compute the test statistic for this test.)
end{ }
end{frame}
%----------------------------------------%
begin{frame}[fragile]
frametitle{Shapiro Wilk Test(b)}
For the data set used previously; $x$ and $y$, we use the Shapiro-Wilk test to determine that both data sets are normally distributed.
<code>

> shapiro.test(x)

        Shapiro-Wilk normality test

data:  x
W = 0.9474, p-value = 0.6378

> shapiro.test(y)

        Shapiro-Wilk normality test

data:  y
W = 0.9347, p-value = 0.5273
</code>

end{frame}


%-------------------------------------------------%
begin{frame}[fragile]
frametitle{Graphical Procedures for assessing Normality}

begin{ }
         * The Q-Q plot is a very useful tool for determining whether or not a data set is normally distributed
         * Interpretation is simple. If the points follow the trendline (provided by the second line of texttt{R} code texttt{qqline}).
         * One should expect minor deviations. Numerous major deviations would lead the analyst to conclude that the data set is not normally distributed.
         * The Q-Q plot is best used in conjunction with a formal procedure such as the Shapiro-Wilk test.
end{ }

<code>
>qqnorm(CWdiff)
>qqline(CWdiff)
</code>

end{frame}

%-------------------------------------------------%

begin{frame}
frametitle{Graphical Procedures for Assessing Normality}

begin{center}
includegraphics[scale=0.32]{10AQQplot}
end{center}
end{frame}
end{document}

%-------------------------------------------------%

Kolmogorov Smirnov Tests
Non-parametric procedures
Compute the test statistic
Same distribution
