\documentclass[11pt]{article} % use larger type; default would be 10pt

\usepackage[utf8]{inputenc} 
\usepackage{geometry} % to change the page dimensions
\geometry{a4paper} 

\usepackage{graphicx} 
\usepackage{booktabs} % for much better looking tables
\usepackage{array} % for better arrays (eg matrices) in maths
\usepackage{paralist} % very flexible & customisable lists (eg. enumerate/itemize, etc.)
\usepackage{verbatim} 
\usepackage{framed}
\usepackage{subfig}
\usepackage{fancyhdr} % This should be set AFTER setting up the page geometry
\pagestyle{fancy} % options: empty , plain , fancy
\renewcommand{\headrulewidth}{0pt} 
\lhead{}\chead{}\rhead{}
\lfoot{}\cfoot{\thepage}\rfoot{}

%%% SECTION TITLE APPEARANCE
\usepackage{sectsty}
\allsectionsfont{\sffamily\mdseries\upshape} 
\usepackage[nottoc,notlof,notlot]{tocbibind} % Put the bibliography in the ToC
\usepackage[titles,subfigure]{tocloft} % Alter the style of the Table of Contents
\renewcommand{\cftsecfont}{\rmfamily\mdseries\upshape}
\renewcommand{\cftsecpagefont}{\rmfamily\mdseries\upshape}

\title{Working with Dates and Times with \texttt{R}}
\author{The Author}
\begin{document}
\maketitle


\section{System Time and Date}

\begin{framed}
\begin{verbatim}
> Sys.time()
[1] "2014-06-17 11:04:46 BST"
> ## print with possibly greater accuracy:
> op <- options(digits.secs=6)
> Sys.time()
[1] "2014-06-17 11:04:46.641917 BST"
> options(op)
> 
> ## locale-specific version of date()
> format(Sys.time(), "%a %b %d %X %Y")
[1] "Tue Jun 17 11:04:46 2014"
> 
> Sys.Date()
[1] "2014-06-17"
> Sys.time()
[1] "2014-06-17 11:05:06 BST"
\end{verbatim}
\end{framed}
%----------------------------------------------------------------------------------------------%
\newpage
\section{Date-Time Classes}


Description of the classes \texttt{POSIXlt} and \texttt{POSIXct} representing calendar dates and times (to the nearest second).

\subsection{A subsection}

\begin{framed}
\begin{verbatim}
> (z <- Sys.time())             # the current date, as class "POSIXct"
[1] "2014-06-17 10:54:24 BST"
> 
> Sys.time() - 3600             # an hour ago
[1] "2014-06-17 09:54:25 BST"
> 
> as.POSIXlt(Sys.time(), "GMT") # the current time in GMT
[1] "2014-06-17 09:54:25 GMT"
\end{verbatim}
\end{framed}
\newpage
\begin{framed}
\begin{verbatim}
> format(.leap.seconds)         # all 24 leap seconds in your timezone
 [1] "1972-07-01 01:00:00" "1973-01-01 00:00:00" "1974-01-01 00:00:00"
 [4] "1975-01-01 00:00:00" "1976-01-01 00:00:00" "1977-01-01 00:00:00"
 [7] "1978-01-01 00:00:00" "1979-01-01 00:00:00" "1980-01-01 00:00:00"
[10] "1981-07-01 01:00:00" "1982-07-01 01:00:00" "1983-07-01 01:00:00"
[13] "1985-07-01 01:00:00" "1988-01-01 00:00:00" "1990-01-01 00:00:00"
[16] "1991-01-01 00:00:00" "1992-07-01 01:00:00" "1993-07-01 01:00:00"
[19] "1994-07-01 01:00:00" "1996-01-01 00:00:00" "1997-07-01 01:00:00"
[22] "1999-01-01 00:00:00" "2006-01-01 00:00:00" "2009-01-01 00:00:00"
\end{verbatim}
\end{framed}

\begin{framed}
\begin{verbatim}
> print(.leap.seconds, tz="PST8PDT")  # and in Seattle's
 [1] "1972-07-01 01:00:00 BST" "1973-01-01 00:00:00 GMT"
 [3] "1974-01-01 00:00:00 GMT" "1975-01-01 00:00:00 GMT"
 [5] "1976-01-01 00:00:00 GMT" "1977-01-01 00:00:00 GMT"
 [7] "1978-01-01 00:00:00 GMT" "1979-01-01 00:00:00 GMT"
 [9] "1980-01-01 00:00:00 GMT" "1981-07-01 01:00:00 BST"
[11] "1982-07-01 01:00:00 BST" "1983-07-01 01:00:00 BST"
[13] "1985-07-01 01:00:00 BST" "1988-01-01 00:00:00 GMT"
[15] "1990-01-01 00:00:00 GMT" "1991-01-01 00:00:00 GMT"
[17] "1992-07-01 01:00:00 BST" "1993-07-01 01:00:00 BST"
[19] "1994-07-01 01:00:00 BST" "1996-01-01 00:00:00 GMT"
[21] "1997-07-01 01:00:00 BST" "1999-01-01 00:00:00 GMT"
[23] "2006-01-01 00:00:00 GMT" "2009-01-01 00:00:00 GMT"
> 
> ## look at *internal* representation of "POSIXlt" :
> leapS <- as.POSIXlt(.leap.seconds)
> names(leapS) ; is.list(leapS)
NULL
[1] TRUE
> utils::str(unclass(leapS), vec.len = 7)
\end{verbatim}
\end{framed}


\end{document}
