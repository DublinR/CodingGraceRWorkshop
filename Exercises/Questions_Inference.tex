\chapter{Statistical Inference}

\section{Confidence Interval examples}

\subsection{Example}
A random sample of 15 observations is taken from a normally distributed population
of values. The sample mean is 94.2 and the sample variance is 24.86.
Calculate a 99\% confidence interval for the population mean.


\subsubsection{Solution}
$t_(14,0.005) = 2.977$
99\% CI is $94.2 \pm 2.977 \sqrt{24.86/15}$ \\i.e. $94.2 \pm 3.83$ \\i.e. $(90.37,98.03)$


\subsection{Example 1: paired T test}


\begin{tabular}{|c|c|c|c|c|c|c|}
  \hline
X & 5.20 & 5.15 & 5.17 & 5.16 & 5.19 & 5.15\\
Y & 5.20 & 5.15 & 5.17 & 5.16 & 5.19 & 5.15\\
  \hline
\end{tabular}


\subsection{Example 2}

Seven measurements of the pH of a buffer solution gave the
following results:

\begin{tabular}{|c|c|c|c|c|c|c|}
  \hline
5.12 & 5.20 & 5.15 & 5.17 & 5.16 & 5.19 & 5.15\\
  \hline
\end{tabular}



\subsection{Example}
Suppose that 9 bags of salt granules are selected from the supermarket
shelf at random and weighed. The weights in grams are 812.0, 786.7, 794.1,
791.6, 811.1, 797.4, 797.8, 800.8 and 793.2. Give a 95\% confidence interval for the
mean of all the bags on the shelf. Assume the population is normal.


Here we have a random sample of size n = 9. The mean is 798.30. The sample
variance is $s^2 = 72.76$, which gives a sample standard deviation $s = 8.53$.

The upper 2.5\% point of the Student's $t$ distribution with n-1 (= 9-1 = 8) degrees of freedom is 2.306.

The 95\% confidence interval is therefore from \\
$(798.30 - 2.306 \times (8.53/\sqrt{9}), 798.30 + 2.306 \times (8.53/\sqrt{9})$\\
which is\\
$(798.30 - 6.56, 798.30 + 6.56) = (791.74, 804.86)$\\
It is sometimes more useful to write this as $798.30 \pm 6.56$.

Note that even if we do not assume the population is normal (that assumption is
never really true) the Central Limit Theorem might suggest that the confidence interval
is nearly right. A larger confidence would increase the length of the interval, so we
trade off increased certainty of coverage against a longer interval.

\subsection{Example}
Ten soldiers visit the rifle range on two different weeks. The first
week their scores are:
67 24 57 55 63 54 56 68 33 43
The second week they score
70 38 58 58 56 67 68 77 42 38
Give a 95\% confidence interval for the improvement in scores from week one to
week two.


\subsubsection{Answer}


This is a case of paired samples, for the scores are repeated observations for each
soldier, and there is good reason to think that the soldiers will differ from each other
in their shooting skill. So we work with the individual differences between the scores.
We shall have to assume that the pairwise differences are a random sample from a
normal distribution.

The differences are:

3 14 1 3 -7 13 12 9 9 -5


Effectively we now have a single sample of size 10, and want a 95\% confidence
interval for the mean of the population from which these differences are drawn. For
this we use a Student's $t$ interval. The sample mean of the differences is 5.2, and
$s^2$ = 54.84. So $s = 7.41$, and the 95\% $t$ interval for the difference in the means is
$5.2 - 2.26(7.41)/\sqrt{10},  5.2 + 2.26(7.41)/\sqrt{10} = (.0.1, 10.5)$.

\subsection{Example} A sample of 50 households in one community
shows that 10 of them are watching a TV special on the national
economy. In a second community, 15 of a random sample of 50
households are watching the TV special. We test the hypothesis
that the overall proportion of viewers in the two communities does
not differ, using the 1 percent level of significance, as follows:

\subsection{2 sided test}
A two-sided test is used when we are concerned about a possible
deviation in either direction from the hypothesized value of the
mean. The formula used to establish the critical values of the
sample mean is similar to the formula for determining confidence
limits for estimating the population mean, except that the
hypothesized value of the population mean m0 is the reference
point rather than the sample mean.

