	\documentclass[a4paper,12pt]{article}
%%%%%%%%%%%%%%%%%%%%%%%%%%%%%%%%%%%%%%%%%%%%%%%%%%%%%%%%%%%%%%%%%%%%%%%%%%%%%%%%%%%%%%%%%%%%%%%%%%%%%%%%%%%%%%%%%%%%%%%%%%%%%%%%%%%%%%%%%%%%%%%%%%%%%%%%%%%%%%%%%%%%%%%%%%%%%%%%%%%%%%%%%%%%%%%%%%%%%%%%%%%%%%%%%%%%%%%%%%%%%%%%%%%%%%%%%%%%%%%%%%%%%%%%%%%%
\usepackage{eurosym}
\usepackage{vmargin}
\usepackage{amsmath}
\usepackage{graphics}
\usepackage{framed}
\usepackage{epsfig}
\usepackage{subfigure}
\usepackage{enumerate}
\usepackage{fancyhdr}

\setcounter{MaxMatrixCols}{10}
%TCIDATA{OutputFilter=LATEX.DLL}
%TCIDATA{Version=5.00.0.2570}
%TCIDATA{<META NAME="SaveForMode"CONTENT="1">}
%TCIDATA{LastRevised=Wednesday, February 23, 201113:24:34}
%TCIDATA{<META NAME="GraphicsSave" CONTENT="32">}
%TCIDATA{Language=American English}

\pagestyle{fancy}
\setmarginsrb{20mm}{0mm}{20mm}{25mm}{12mm}{11mm}{0mm}{11mm}
\lhead{Introduction to R} \rhead{Kevin O'Brien} \chead{Creating Data} %\input{tcilatex}

\begin{document}


%---------------------------------------------------%
\section{Data Creation and Transfer}

%------------------------------------------------------%
\subsection{The Assignment operator}
The assignment operator is used to assign names to particular values. Historically the assign-
ment operator was ) a "<-". 
The assignment operator can also be =. This is valid as of R
version 1.4.0.
Both will be used, although, you should learn one and stick with it. Many long term R
users prefer the arrow approach. You can also use -> as an assignment operator, reversing the
usual assignment positions. (This is actually really useful). Commands are separated either by
a semi colon or by a newline.
\begin{framed}
\begin{verbatim}
> a <- 6
> b = 5
> a + b -> d
> # dont use "c" as a variable name.
> d
[1] 11
>e=7;f<-4  # two commands on one line.
\end{verbatim}
\end{framed}


%------------------------------------------------------%
Before we continue, try using the up and down keys, and see what happens. Previously
typed commands are re-presented, and can be re-executed.
R stores both data and output from data analysis (as well as everything else) in objects.
The variables we have created so far are objects. A list of all objects in the current session can
be obtained with ls().
%==========================================================================%
\subsection*{Reserved Words}
Some names are used by the system, e.g.T, F,q,c etc . Avoid using these as names when you create objects..

\subsection*{ The \texttt{c()} command}
To create a simple data set, known as a vector, we use the c() command to create the vector.
\begin{verbatim}
> Y=c(1,2,4,8,16 ) #create a data vector with specified elements
> Y
[1] 1 2 4 8 16
\end{verbatim}
\subsection*{ Vector of Numeric Values}
Numvec = c(10,13,15,19,25);

\subsection*{ Vector of Character Values}
Charvec = c("LouLou","Oscar","Rasher");


\subsection*{Vector of Logical Values}
Charvec = c(TRUE,TRUE,FALSE,TRUE);
Vectors can be bound together either by row or by column.
\begin{verbatim}
> X=1:3; Y=4:6
> cbind(X,Y)
X Y
[1,] 1 4
[2,] 2 5
[3,] 3 6
>
> rbind(X,Y)
[,1] [,2] [,3]
X 1 2 3
Y 4 5 6
\end{verbatim}
%==============================================================================%
\subsection{The scan() command}
The scan() function is a useful method of inputting data quickly. You can use to quickly copy
and paste values into the R environment. It is best used in the manner as described in the
following example. Create a variable "X" and use the scan() function to populate it with
values. Type in a value, and then press return. Once you have entered all the values, press
return again to return to normal operation.
\begin{framed}
\begin{verbatim}
> X=scan()
1: 4
2: 5
3: 5
4: 6
5:
Read 4 items
\end{verbatim}
\end{framed}

Remark: Try the \texttt{edit()} command on object X.
%==========================================================================%
\subsection*{Characters with the \texttt{scan()} command}
The scan() command can also be used forinputting character data.
\begin{verbatim}
> Y=scan(what=" ")
1: LouLou
2: Oscar
3: Charlie
4:
Read 3 items
> Y
[1] "LouLou" "Oscar" "Charlie"
\end{verbatim}
%==============================================================================%
\subsection*{Using the data editor}
%==============================================================================%
\subsection*{Spreadsheet Interface}
R provides a spreadsheet interface for editing the values of existing data sets. We use the
command \texttt{data.entry()}, and name of the data object as the argument. (Also try out the
fix() command)
\begin{framed}
\begin{verbatim}
> data.entry(X) # Edit the data set and exit interface
> X

\end{verbatim}
\end{framed}



\subsection*{Creating a Data Frame}
\begin{framed}
\begin{verbatim}
A <- data.frame(
         name=c("a","b","c"),
         ownership=c("Case 1","Case 1","Case 2"),
         listed.at=c("NSE",NA,"BSE"),
              # Firm "b" is unlisted.
         is.listed=c(TRUE,FALSE,TRUE),
              # R convention
              # boolean variables are named "is.something"
         x=c(2.2,3.3,4.4),
         date=as.Date(c("2004-04-04","2005-05-05","2006-06-06"))
         )
\end{verbatim}
\end{framed}
%====================================================================================%

\subsection*{Using the \texttt{scan()} command to input character data}
Previously we have seen the \texttt{scan()} command used to quickly input numeric data. The command can also be used to input character data. The addition argument , \texttt{what=" "}, must be used.  
Create a variable “grouping” that comprises, in order,  five “A”s and then six “B”s.

\begin{verbatim}
# inputting character data

grouping = scan(what=" ")

# Enter values
# Hit return again when you have finished.

\end{verbatim}


\subsection*{Spreadsheet Interface}
\texttt{R} provides a spreadsheet interface for editing the values of existing data sets.
We use the command \texttt{data.entry()} , and name of the data object as the argument.

\begin{verbatim}
> data.entry(X) # Edit the data set and exit interface
> X
\end{verbatim}

\end{document}
