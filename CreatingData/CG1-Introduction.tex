%===============================%
Contents
1 Introduction to R 3
1.1 Installing R...... . . 3
1.2 Command Line Interface..... . 3
1.3 The Assignment operator..... . 4
1.3.1 Reserved Words..... . . 4
1.4 Commenting...... . . 4
1.5 Defining Variables.....5
1.6 Help Functions...... 5
1.7 The help.start() command.... . 5
1.8 Basic Maths Operations..... . 5
1.9 Basic R Editor...... 6
1.10 Built-In Data Sets..... . 6
1.11 The summary() command..... 7
1.12 Working directories..... . 7
1.13 Coming Unstuck.....7
1.14 Quitting the R environment....8
1.15 Data Objects...... . 8
1.16 Listing all items in a workspace.... 8
1.17 Removing items...... 8
1.18 Saving and Loading R Data Objects.... 8
%======================================================================================================%
\section{1 Introduction to R}
Source: R project website (http://www.r-project.org)
R is a language and environment for statistical computing and graphics. It is a GNU project
which is similar to the S language and environment which was developed at Bell Laboratories
(formerly AT&T, now Lucent Technologies) by John Chambers and colleagues. R can be con-
sidered as a different implementation of S. There are some important differences, but much
code written for S runs unaltered under R.

%------------------------------------------------------%
R provides a wide variety of statistical (linear and nonlinear modelling, classical statistical tests,
time-series analysis, classification, clustering, ...) and graphical techniques, and is highly ex-
tensible. The S language is often the vehicle of choice for research in statistical methodology,
and R provides an Open Source route to participation in that activity.
One of R's strengths is the ease with which well-designed publication-quality plots can be
produced, including mathematical symbols and formulae where needed. Great care has been
taken over the defaults for the minor design choices in graphics, but the user retains full control.

%------------------------------------------------------%
R is available as Free Software under the terms of the Free Software Foundation's GNU General
Public License in source code form. It compiles and runs on a wide variety of UNIX platforms
and similar systems (including FreeBSD and Linux), Windows and MacOS.
R is a programming environment
\item uses a well-developed but simple programming language
\item allows for rapid development of new tools according to user demand
\item these tools are distributed as packages, which any user can download to customize the R
environment.
Base R and most R packages are available for download from the Comprehensive R Archive Net-
work (CRAN) cran.r-project.org. Base R comes with a number of basic data management,
analysis, and graphical tools R's power and 
exibility, however, lie in its array of packages
(currently more than 4,000!)

%----------------------------------------------------------------------%
1.1 Installing R
R is very easily installed by downloading it from the CRAN website. Installation usually takes
about 2 minutes. When Installation of R is complete, the distinctive R Icon will appear on your
desktop. To start R, simply click this Icon. It is common to re-install R once a year or so. The
current version of R, version 3.0.0. was released quite recently.

%------------------------------------------------------%
1.2 Command Line Interface
When you start R, the command line interface window will appear on screen. This is one
of many windows in the R environment, others including graphical output windows, or script
editors. R code can be entered into the command line directly. Alternatively code can be saved
to a script, which can be run inside a session using the source() function.


1.4 Commenting
For the sake of readability, it is good practice to comment code. The # character at the
beginning of a line signifies a comment, which is not executed. Lines of hashtags can be used
to identify the beginning and end of code segments
# This is a comment
#####################
# Start of Segment 1
#####################
4
%===============================%
1.5 Defining Variables
A convention is to use define a variable name with a capital letter (R is case sensitive). This
reduces the chance of overwriting in-build R functions, which are usually written in lowercase
letters. Commonly used variable names such as x,y and z (in lower case letters) are not \re-
served".
