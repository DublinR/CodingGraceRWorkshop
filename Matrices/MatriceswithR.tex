

<p>
### {Matrices}
<p>
#### {Creating a matrix}
Matrices can be created using the texttt{matrix()} command. The arguments to be supplied are 1) vector of values to be entered
2) Dimensions of the matrix, specifying either the numbers of rows or columns.

Additionally you can specify if the values are to be allocated by row or column. By default they are allocated by column.
<code>
Vec1=c(1,4,5,6,4,5,5,7,9)		# 9 elements
A=matrix(Vec,nrow=3)		#3 by 3 matrix. Values assigned by column.
A
B= matrix(c(1,6,7,0.6,0.5,0.3,1,2,1),ncol=3,byrow =TRUE)
B				          #3 by 3 matrix. Values assigned by row.
</code>	
If you have assigned values by column, but require that they are assigned by row, you can use the transpose function
<code>
t().
t(A)				# Transpose
A=t(A)	
</code>

Another methods of creating a matrix is to "bind" a number of vectors together, either by row or by column. The commands are rbind() and cbind() respectively.
<code>
x1 =c(1,2) ; x2 = c(3,6)
rbind(x1,x2)
cbind(x1,x2)
</code>




Particular rows and columns can be accessed by specifying the row number or column number, leaving the other value blank.
<code>
A[1,]	  # access first row of A
B[,2]   # access first column of B
</code>
Addition and subtractions
For matrices, addition and subtraction works on an element- wise basis. The first elements of the respective matrices are added, and so on.
A+B
A-B

<p>
#### {Matrix Multiplication}
To multiply matrices, we require a special operator for matrices; $"%     %"$.
If we just used the normal multiplication, we would get an element-wise multiplication.
<code>
A       B  		#Element-wise multiplication
A %     % B  	#Matrix multiplication
</code>

We can compute crossproducts using the crossprod () command. If only one matrix is used it
<code>
crossprod(A,B) 		# A'B
crossprod(A) 			# A'A
</code>
Diagonals
The diag() command is a very versatile function for using matrices.
It can be used to create a diagonal matrix with elements of a vector in the principal diagonal. For an existing matrix, it can be used to return a vector containing the elements of the principal diagonal.


Most importantly, if k is a scalar, diag() will create a k x k identity matrix.
<code>
Vec2=c(1,2,3)
diag(Vec2)	#	Constructs a diagonal matrix based on values of Vec2
diag(A)	#        Returns diagonal elements of A as a vector
diag(3)	#	creates a 3 x 3 identity matrix
diag(diag(A)) #  	Diagonal matrix D of matrix A ( Jacobi Method)
</code>
Determinants, Inverse Matrices and solving Linear systems
To compute the determinant of a square matrix, we simply use the det() command
det(A)
det(B)
To find the inverse of a square matrix, we use the solve() command, specifying only the matrix in question
solve(A)

To solve a system of linear equations in the form Ax=b , where A is a square matrix, and b is a column vector of known values, we use the solve() command to determine the values of the unknown vector x.
<code>
b=vec2  # from before
solve(A, b)
</code>
Row and Column Statistics.
Statistic on the rows and columns can easily be computed if required.
<code>
rowMeans(A)  # Returns vector of row means.
rowSums(A)  # Returns vector of row sums.
colMeans(A)  # Returns vector of column means.
colSums(A)  # Returns vector of coumn means.
</code>
Eigenvalues and Eigenvectors
The eigenvalues and eigenvectors can be computed using the eigen() function.  A data object is created.
This is a very important type of matrix analysis, and many will encounter it again in future modules.
<code>
Y = eigen(A)
names(Y)
"	y$val are the eigenvalues of A
"	y$vec are the eigenvectors of A
</code>


<p>
### {Matrices}
A matrix refers to a numeric array of rows and columns.

One of the easiest ways to create a matrix is to combine vectors of equal
length using texttt{cbind()}, meaning "column bind". Alternatively one can use  texttt{rbind()}, meaning ``row bind".


sub<p>
#### {Matrices Inversion}
sub<p>
#### {Matrices Multiplication}
