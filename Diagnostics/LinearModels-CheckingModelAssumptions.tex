
Large
		<p>
####        * {Analysis of residuals}
		begin{ }
		        * 	Perform an analysis of regression residuals ( you can pick the best regression model from last <p>
### ).
		        * 	Are the residuals normally distributed?
		begin{ }
			         * Histogram /  Boxplot / QQ plot / Shapiro Wilk Test
			         * Also you can plot the residuals to check that there is constant variance.
		end{ }	
		end{ }
	
	<pre>
		<code>
		y <- rnorm(10)
		x <- rnorm(10)
		fit1 <- lm(y~x)
		res.fit1 <-  resid(fit1)
		plot(res.fit1)
		</code>
     </pre>
<p>		
newpage
<p>
### {Assessing a Linear Regression :  Model Diagnostics}
begin{ }
         * The ease of implementation fosters the impression that linear regression is easy: just use the texttt{textbf{lm()}} function. Yet fitting
this is only the beginning.

         * After a linear regression analysis has been performed. It is good practice to verify the model’s quality
by running diagnostic checks.

         * The approach we will take is to create diagnostic plots, by plotting the model object. Rather than producing a scatterplot, this method will produce several diagnostic plots:
end{ }


<pre>
<code>

Fit1 <- lm(Sepal.length ~ Sepal.Width, data=iris)
plot(Fit1)
</code> 
</pre>
<p>

begin{ }
         * Next, identify possible outliers either by looking at the diagnostic plot of the residuals
%=================================================================================== %
newpage
         * Another approach is to use the texttt{textbf{outlierTest()}} function of the textit{textbf{car}} package:

end{ }

<pre>
<code>
#If package not installed, uncomment next line.
#install.packages("car")

library(car)
outlierTest(Fit1)
</code> 
</pre>
<p>
<code>
> outlierTest(Fit1)

No Studentized residuals with Bonferonni p < 0.05
Largest |rstudent|:
    rstudent unadjusted p-value Bonferonni p
132  2.79155          0.0059429      0.89143

</code>
%=================================================================================== %
newpage
<p>
#### {Influence Measures}
begin{ }
         * Finally, identify any overly influential observations by using the texttt{textbf{influence.measures()}} 
function.
         * If an observation is considered influential, it will be indicated with an asterisk on the right hand side.          * Interpretation of the individual statistics, such as textit{DFFITS} and textit{DFBETA} are beyond the scope of this workshop.
end{ }


<pre>
<code>
influence.measures(Fit1)
</code> 
</pre>
<p>

end{document}
