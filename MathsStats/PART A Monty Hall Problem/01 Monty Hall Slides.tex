documentclass{beamer}

usepackage{graphicx}
usepackage{framed}

begin{document}
%================================================================================= %
begin{frame}
begin{figure}
centering
includegraphics[width=1.1linewidth]{MontyHall/Slide1}


end{figure}

end{frame}
%============================================================================= %
%================================================================================= %
begin{frame}
	frametitle{Let's Make a Deal with Monty Hall}
	begin{figure}
		centering
		includegraphics[width=0.99linewidth]{MontyHall/Slide2}
		
	end{figure}
	
end{frame}
%============================================================================= %
begin{frame}
	
	frametitle{Monty Hall Problem}LARGE
	begin{ }
	         * Suppose that there are three closed doors on the set of the  textbf{emph{Let's Make a Deal}}, a TV game show presented by Monty Hall. 
	         * Behind one of these doors is a car; behind the other two are goats. 
	         * The contestant does not know where the car is, but Monty Hall does.
	
	         * 
	The contestant selects a door, but not the outcome is not immediately evident. 
	end{ }
	
end{frame}
%============================================================================ %
begin{frame}
	
	frametitle{Monty Hall Problem}LARGE
	begin{ }
	         * Monty opens one of the remaining ``wrong" doors - revealing a goat.
end{ }

begin{figure}
centering
includegraphics[width=0.7linewidth]{MontyHall/goat}
caption{}
label{fig:goat}
end{figure}

	{ Large textit{Remember this for later : If the contestant has already chosen the correct door, Monty is equally likely to open either of the two remaining doors.}
}
end{frame}
%============================================================================ %
begin{frame}
	frametitle{Monty Hall Problem}LARGE
	begin{ }
		         * textbf{Stick or Switch}
	After Monty has shown a goat behind the door that he opens, the contestant is always given the option to switch doors: from the one they have already to select to the other closed door. 
	         * textbf{Question} -  What is the probability of winning the car if she stays with her first choice? What if she decides to switch?
	end{ }
end{frame}

%================================================================================= %
begin{frame}
	frametitle{Ask Marilyn}
	begin{figure}
		centering
		includegraphics[width=1.17linewidth]{MontyHall/Slide3}

	end{figure}
	
end{frame}
%============================================================================= %

	
	%================================================================================= %
	begin{frame}
		frametitle{The Game Show Problem}
		begin{figure}
			centering
			includegraphics[width=1.1linewidth]{MontyHall/Slide4}
	
		end{figure}
		
	end{frame}
	%============================================================================= %

begin{frame}
Large

textbf{Is it to your advantage to swith your choice of door?} bigskip
Three possible answers 
begin{ }
         * Yes - Switch Doors bigskip

         * No - Stick with Your original choice bigskip

         * Doesn't Matter either way
end{ }
end{frame}
%================================================================================= %
begin{frame}
frametitle{Marilyn Responds}
		begin{figure}
		centering
		includegraphics[width=1.0linewidth]{MontyHall/Slide5}
		
	end{figure}
		
end{frame}
%============================================================================= %
begin{frame}
	LARGE
	[ mbox{and that was the end of that....} ]
	bigskip
		LARGE
		[ mbox{except....} ]
end{frame}
%================================================================================= %
begin{frame}
	begin{figure}
		centering
		includegraphics[width=1.1linewidth]{MontyHall/Slide6}
		
	end{figure}
	
end{frame}
%============================================================================= %

%================================================================================= %
begin{frame}
	
	Let's read some letters!!!!
	begin{figure}
		centering
		includegraphics[width=0.9linewidth]{MontyHall/Slide7}
		end{figure}
	
end{frame}

%================================================================================= %
begin{frame}
	begin{figure}

		includegraphics[width=1.20linewidth]{MontyHall/Slide8}
		
	end{figure}
	
end{frame}
%============================================================================= %

begin{frame}
	begin{figure}
		centering
		includegraphics[width=0.7linewidth]{MontyHall/687}
	end{figure}
	
end{frame}
%================================================================================= %
begin{frame}
	begin{figure}
		centering
		includegraphics[width=1.1linewidth]{MontyHall/Slide9}

	end{figure}
	
end{frame}
%============================================================================= %


%================================================================================= %
begin{frame}
	begin{figure}
		centering
		includegraphics[width=1.16linewidth]{MontyHall/Slide10}
		
	end{figure}
	
end{frame}
%============================================================================= %

%================================================================================= %
begin{frame}
	begin{figure}
		centering
		includegraphics[width=1.17linewidth]{MontyHall/Slide11}
		
	end{figure}
	
end{frame}
%============================================================================= %

%================================================================================= %
begin{frame}
	begin{figure}
		centering
		includegraphics[width=1.18linewidth]{MontyHall/Slide12}

	end{figure}
	
end{frame}
%============================================================================= %

%================================================================================= %
begin{frame}
	begin{figure}
		centering
		includegraphics[width=1.16linewidth]{MontyHall/Slide13}
	end{figure}
	
end{frame}
%============================================================================= %
%================================================================================= %
begin{frame}
	begin{figure}
		centering
		includegraphics[width=1.17linewidth]{MontyHall/Slide14}

	end{figure}
	
end{frame}
%============================================================================= %

%================================================================================= %
begin{frame}
	begin{figure}
		centering
		includegraphics[width=1.17linewidth]{MontyHall/Slide15}
		
	end{figure}
	
end{frame}
%============================================================================= %
%============================================================================ %
begin{frame}
	
	frametitle{Implementation with texttt{R} (part 1)}
	
	begin{ }
         * e have 3 doors to choose from, so we will define a sequence A,B and C. 
         * The command texttt{textbf{sample(,n)}} takes a sample of size n from a specified set of values. 
         * Here we just want to select one door to be our "correct door" and another to be ``selected" door (i.e. the contestant selects)
end{ }
	
	
end{frame}

%============================================================================ %
begin{frame}[fragile]
	frametitle{Implementation (part 1)}
	These events are independent. We will perform the selection for both doors separately, but this can be implemented in one command.
	
	
	<pre>
		<code>
		Doors = c("A","B","C")
		Correct = sample(Doors,1)
		Choice = sample(Doors,1)
		</code> 
	</pre>
<p>
end{frame}
%============================================================================ %
begin{frame}[fragile]
	A wrong door must be selected to be opened. The door selected by the contestant can't be chosen. First let us select the doors that must stay closed, then find the ones we can choose from to open
	
	<pre>
		<code>
		StayClosed = union(Correct, Choice)
		CanOpen = setdiff(Doors, StayClosed)
		</code> 
	</pre>
<p>
end{frame}
%============================================================================ %
%============================================================================ %
begin{frame}[fragile]
	
	<pre>
		<code>
		NotOpen = setdiff(Doors,Open); Stick = Choice        
		#The previous statement is to aid the narrative. 
		
		Switch = setdiff(NotOpen,Choice)
		</code> 
	</pre>
<p>
end{frame}
%============================================================================ %
%============================================================================ %
begin{frame}[fragile]
	
	<pre>
		<code>	
		# Was sticking the right decision? Or was switching?
		# The following logicial statements  will tell us.
		
		Stick==Choice ## 1 for Yes 0 for No 
		
		Switch==Choice ## 1 for Yes 0 for No 
		</code> 
	</pre>
<p>
end{frame}
end{document}
%============================================================================ %
begin{frame}[fragile]
	
	frametitle{Writing a Function}
	We are going to create a function texttt{textbf{MHfunc()}} to help us simulate a solution for the Monty Hall Problem. The function is constructed using texttt{R} code we have seen already. The output of the function is returned as a data frame, with three columns:
	 
		        * [Correct] : The number of the correct door.
		        * [Choice] :  The door that the contestant chose originally, and the door selected if the contestant decided to ``stick".
		        * [Switch] : The door selected if the contestant chose to ``switch".
	

<p>
end{frame}

end{document}
