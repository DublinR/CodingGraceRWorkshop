



\newpage
\chapter{Advanced R code}
\section{Data frame}
A Data frame is
\subsection{Merging Data frames}

\section{Functions}
Syntax to define functions

\begin{framed}
\begin{verbatim}
myfct <- function(arg1, arg2, ...) { function_body }
\end{verbatim}
\end{framed}
The value returned by a function is the value of the function body, which is usually an unassigned final expression, e.g.: return()

Syntax to call functions
\begin{framed}
\begin{verbatim}
myfct(arg1=..., arg2=...)
\end{verbatim}
\end{framed}


\subsection{Time and Date}
It is useful . The length of time a program takes is interesting.


\begin{framed}
\begin{verbatim}
date() # returns the current system date and time
\end{verbatim}
\end{framed}


\section{The Apply family}

Sometimes want to apply a function to each element of a
vector/data frame/list/array.
\\
Four members: lapply, sapply, tapply, apply
\\
lapply: takes any structure and gives a list of results (hence
the `l')
\\
sapply: like lapply, but tries to simplify the result to a
vector or matrix if possible (hence the `s')
\\
apply: only used for arrays/matrices
\\
tapply: allows you to create tables (hence the `t') of values
from subgroups defined by one or more factors.
\newpage

\section{Plots}
This section is an introduction for producing simple graphs with
the R Programming Language.
\begin{itemize}
*  Line Charts  *  Bar Charts *  Histograms *  Pie
Charts *  Dotcharts
\end{itemize}


\begin{itemize}
* 
* 
\end{itemize}
\large \begin{verbatim}
> code here
\end{verbatim}\large


\subsection{ Charts}

\begin{framed}
\begin{verbatim}
# Define 2 vectors cars <- c(1, 3, 6, 4, 9) trucks <- c(2, 5, 4,
5, 12)

# Calculate range from 0 to max value of cars and trucks g_range
<- range(0, cars, trucks)

# Graph autos using y axis that ranges from 0 to max # value in
cars or trucks vector.  Turn off axes and # annotations (axis
labels) so we can specify them ourself plot(cars, type="o",
col="blue", ylim=g_range,
axes=FALSE, ann=FALSE)

# Make x axis using Mon-Fri labels axis(1, at=1:5,
lab=c("Mon","Tue","Wed","Thu","Fri"))

# Make y axis with horizontal labels that display ticks at # every
4 marks. 4*0:g_range[2] is equivalent to c(0,4,8,12). axis(2,
las=1, at=4*0:g_range[2])

# Create box around plot box()

# Graph trucks with red dashed line and square points
lines(trucks, type="o", pch=22, lty=2, col="red")

# Create a title with a red, bold/italic font title(main="Autos",
col.main="red", font.main=4)

# Label the x and y axes with dark green text title(xlab="Days",
col.lab=rgb(0,0.5,0)) title(ylab="Total", col.lab=rgb(0,0.5,0))

# Create a legend at (1, g_range[2]) that is slightly smaller #
(cex) and uses the same line colors and points used by # the
actual plots legend(1, g_range[2], c("cars","trucks"), cex=0.8,
col=c("blue","red"), pch=21:22, lty=1:2);

\end{verbatim}
\end{framed}
\subsection{Bar charts}
\begin{framed}
\begin{verbatim}
# Define the cars vector with 7 values
cars <- c(1, 3, 6, 4, 9, 5, 7)
# Graph cars
barplot(cars)
\end{verbatim}
\end{framed}
\subsection{Boxplots}
\subsection{Setting graphical parameters}
\subsection{Miscellaneous}
The following code can be used to make variations of the plots.

\begin{framed}
\large \begin{verbatim}
# Make an empty chart
plot(1, 1, xlim=c(1,5.5), ylim=c(0,7), type="n", ann=FALSE)

# Plot digits 0-4 with increasing size and color
text(1:5, rep(6,5), labels=c(0:4), cex=1:5, col=1:5)

# Plot symbols 0-4 with increasing size and color
points(1:5, rep(5,5), cex=1:5, col=1:5, pch=0:4)
text((1:5)+0.4, rep(5,5), cex=0.6, (0:4))

# Plot symbols 5-9 with labels
points(1:5, rep(4,5), cex=2, pch=(5:9))
text((1:5)+0.4, rep(4,5), cex=0.6, (5:9))

# Plot symbols 10-14 with labels
points(1:5, rep(3,5), cex=2, pch=(10:14))
text((1:5)+0.4, rep(3,5), cex=0.6, (10:14))

# Plot symbols 15-19 with labels
points(1:5, rep(2,5), cex=2, pch=(15:19))
text((1:5)+0.4, rep(2,5), cex=0.6, (15:19))

# Plot symbols 20-25 with labels
points((1:6)*0.8+0.2, rep(1,6), cex=2, pch=(20:25))
text((1:6)*0.8+0.5, rep(1,6), cex=0.6, (20:25))
\end{verbatim}\large
\end{framed}

\subsection{Lattice Graphs}
\subsection{setting up}
Execute the following command:
\begin{framed}
\begin{verbatim}
library(lattice)
\end{verbatim}
\end{framed}
For information on lattice, type:
\begin{framed}
\begin{verbatim}
help(package = lattice)
\end{verbatim}
\end{framed}
The examples in this section are generally drawn from the R documentation and Murrell (2006).

Murrell gives three reasons for using Lattice Graphics:

They usually look better.
They can be extended in powerful ways.
The resulting output can be annotated, edited, and saved

\subsection{3 Dimensional Graphs}
How to do a 3-d graph

\newpage
\chapter{Statistical Analysis using R}
\section{Confidence Intervals}
\subsection{Confidence Intervals for Large Samples}
\subsection{Confidence Intervals for Small Samples}

\section{Linear Models}

The Slope and Intercept
\begin{framed}
\begin{verbatim}

\end{verbatim}
\end{framed}

\section{ANOVA}








\subsection{Subsetting datasets by rows}

Suppose we wish to divide a data frame into two different section. The simplest approach we can take is to create two new data sets, each assigned data from the relevant rows of the original data set.

Suppose our dataset ``Info" has the dimensions of 200 rows and 4 columns. We wish to separate "Info" into two subsets , with the first and second 100 rows respectively. ( We call these new subsets "Info.1" and "Info.2".)
\begin{verbatim}
Info.1 = Info[1:100,]#assigning "info" rows 1 to 100
Info.2 = Info[101:200,]#assigning "info" rows 101 to 200
\end{verbatim}

More useful commands such as rbind() and cbind()  can be used to manipulate vectors.

Part 2 Strategies for Data project
\begin{itemize}
*  Exploratory Data Analysis

The first part of your report should contain some descriptive statistics and summary values. Also include some tests for normality.

* {Regression}
You should have a data set with multiple columns, suitable for regression analysis.
Familiarize yourself with the data, and decide which variable is the dependent variable.

Also determine the independent variables that you will use as part of your analysis.

* {Correlation Analysis}
Compute the Pearson correlation for the dependent variable with the respective independent variables.  As part of your report, mention the confidence interval for the correlation estimate
Choose the independent variables with the highest correlation as your candidate variables.
For these independent variables, perform a series of simple linear regression procedures.
\begin{verbatim}
lm(y~x1)
lm(y~x2)
\end{verbatim}
Comment on the slope and intercept estimates and their respective p-values. Also comment on the coefficient of determination (multiple R squared). Remember to write the regression equations.
Perform a series of multiple linear regressions, using pairs of candidate independent variables.
\begin{verbatim}
lm(y~x1 +x2)
lm(y~x2 +x3)
\end{verbatim}
Again, comment on the slope and intercept estimates, and their respective p-values.
In this instance, compare each of the models using the coefficient of determinations. Which model explains the data best?
\subsection{Analysis of residuals}
Perform an analysis of regression residuals ( you can pick the best regression model from last section).
Are the residuals normally distributed?
Histogram /  Boxplot / QQ plot / Shapiro Wilk Test
Also you can plot the residuals to check that there is constant variance.
\begin{verbatim}
y=rnorm(10)
x=rnorm(10)
fit1=lm(y~x)
res.fit1 = resid(fit1)
plot(res.fit1)
\end{verbatim}




%---------------------------------------------------------------------------Probability Distributions ----%
\newpage
\chapter{Probability Distributions}
\section{Generating a set of random numbers}

%\begin{framed}
\large \begin{verbatim}
rnorm(10)
\end{verbatim}\large
%\end{framed}

%----------------------------------------------------------------------------Graphical Methods--%
\newpage
\chapter{Graphical methods}

\section{Scatterplots}
%\begin{figure}
%  % Requires \usepackage{graphicx}
%  \includegraphics[scale=0.40]{MTCARSmpgwt.png}\\
%  \caption{Scatterplot}\label{mpgwt}
%\end{figure}


\section{Adding titles, lines, points to plots}


\large \begin{verbatim}
library(MASS)
# Colour points and choose plotting symbols according to a levels of a factor
plot(Cars93$Weight, Cars93$EngineSize, col=as.numeric(Cars93$Type),
pch=as.numeric(Cars93$Type))

# Adds x and y axes labels and a title.
plot(Cars93$Weight, Cars93$EngineSize, ylab="Engine Size",
xlab="Weight", main="My plot")
# Add lines to the plot.
lines(x=c(min(Cars93$Weight), max(Cars93$Weight)), y=c(min(Cars93$EngineSize),
max(Cars93$EngineSize)), lwd=4, lty=3, col="green")
abline(h=3, lty=2)
abline(v=1999, lty=4)
# Add points to the plot.
\end{verbatim}\large

\newpage
\chapter{Programming}







%----------------------------------------------------%
\subsubsection{Two Sample t test}

The two-sample t test is used to test the hypothesis that two samples may
be assumed to come from distributions with the same mean.

The theory for the two-sample t test is not very different in principle from
that of the one-sample test. Data are now from two groups, $x_{11}, . . . , x_{1n1}$
and $x_{21}, . . . , x_{2n2}$ , which we assume are sampled from the normal distributions
$N(µ_{1}, \sigma^{1}_{2} )$ and
$N(µ_{2}, \sigma^{2}_{2} )$, and it is desired to test the null hypothesis
$\mu_{1} = \mu_{2}$. You then calculate

\[
t = \frac{\bar{X}_{1}-\bar{X}_{2}}{S.E.(\bar{X}_{1}-\bar{X}_{2})}
\]




%---------------------------------------------------%
\subsubsection{slide234}
The TS are <equation here>  
The p-values for both of these tests are 0 and so there is enough evidence to reject $H_0$ and conclude that both 0 and 1 are not 0, i.e. there is a significant linear relationship between x and y. 
Also given are the $R^2$ and $R^2$ adjusted values. Here $R^2 = SSR/SST = 0.8813$ and so $88.13\%$ of the variation in y is being explained by x. 
The final line gives the result of using the ANOVA table to assess the model t.

%----------------------------------------------------%

\subsubsection{slide235}

In SLR, the ANOVA table tests <EQN>The TS is the F value and the critical value and p-values are found
in the F tables with (p - 1) and (n - p) degrees of freedom.

This output gives the p-value = 0, therefore there is enough evidence to reject H0 and conclude that there is a signicant linear relationship between y and x. The full ANOVA table can be accessed using :

<TABLE HERE>



\subsubsection{slide236}
Once the model has been tted, must then check the residuals.
The residuals should be independent and normally distributed with
mean of 0 and constant variance.
A Q-Q plot checks the assumption of normality (can also use a
histogram as in MINITAB) while a, plot of the residuals versus fitted values gives an indication as to whether the assumption of constant variance holds.

<HISTOGRAM>




\section{Introduction to ``R}}
``R} consists of a base package and many additional packages
``R} was originally designed as a command language.  
Commands were typed into a text-based input area on the computer screen and the program responded with a response to each command.
The ``R} console opens with information and then a prompt mark  ``>"  it is ready to accept commands
``R}  is an open source software package, meaning that the code written to implement the various functions can be freely examined and modified.
``R} can be installed free of charge from the ``R}-project website.

%----------------------------------------------------%
\subsubsection{slidename}

\large \begin{verbatim}
> xbar <- 83
> sigma <- 12
> n <- 5
> sem <- sigma/sqrt(n)
> sem
[1] 5.366563
> xbar + sem * qnorm(0.025)
[1] 72.48173
> xbar + sem * qnorm(0.975)
[1] 93.51827
\end{verbatim}\large


\subsubsection{Testing the slope (II)}

You can compute a
t test for that hypothesis simply by dividing the estimate by its standard
error
\begin{equation}
t = \frac{\hat{\beta}}{S.E.(\hat{\beta})}
\end{equation}
which follows a t distribution on n - 2 degrees of freedom if the true $\beta$ is
zero.


%----------------------------------------------------%
\begin{itemize}
*  The standard $\chi^{2}$ test  in chisq.test works with data in matrix form, like fisher.test does.
*  For a 2 by 2 table, the test is exactly equivalent to prop.test.
\end{itemize}


\large \begin{verbatim}
> chisq.test(lewitt.machin)
\end{verbatim}\large







\chapter { R Graphics}
\section Enhancing your scatter plots
\subsection{Adding lines}
Previously we have used scatter plots to plot bivariate data. They were constructed using the plot() command.
Recall that we can use the arguments ``xlim} and ``ylim} to control the vertical and horizontal range of the plots, by specifying a two element vector (min and max) for each.

Using the ``abline()`` command, we can add lines to our scatter plots. We specify the argument according to the type of line required. A demonstration of three types of line is provided below.
Additionally we change the colour of the added lines, by specifying a colour in the ``col} argument. We can also change the line type to one of four possible types, using the ``lty} argument.

The line types are follows
\begin{itemize}
* ``lty =1}   Normal full line (default)
* ``lty =2}   Dashed line
* ``lty =3}   Dotted line
* ``lty =4}   Dash-dot line
\end{itemize}
\large \begin{verbatim}
x=rnorm(10)
y=rnorm(10)
plot(x,y)
plot(x,y,xlim=c(-4,4),ylim=c(-4,4))
abline(v =0 , lty =2 )    # add a vertical dotted line (here the y-axis) to the plot
abline(h=0  ,lty =3)    # add a horizontal dotted line (here the x-axis) to the plot
abline(a=0,b=1,col="green") # add a line to your plot with intercept "a" and slope "b"
\end{verbatim}\large

\subsection{Changing your plot character}

To change the plot character (the symbol for each covariate, we supply an additional argument to the plot() function.  This argument is formulated as pch=n where n is some number.
Additionally we change the colour of the characters, by specifying a colour in the col argument.
\large \begin{verbatim}
plot(x,y,pch=15,col="red")#Square plot symbols
plot(x,y,pch=16,col="green")#Orb plot symbols
plot(x,y,pch=17,col="mauve")#Triangular plot symbols
plot(x,y,pch=36,col="amber")#Dollar sign plot symbols
\end{verbatim}\large
Recall that we can add new variates to an existing scatterplot using the points() function. Remember to set the vertical and horizontal limits accordingly.
\large \begin{verbatim}
y1 = rnorm(10); y2 = rnorm(10)
plot(x,y1, pch=8,col="purple" ,xlim=c(-5,5),ylim=c(-5,5))
points(x,y2,pch=12,col="green")
\end{verbatim}\large
\subsection{Adding the regression model line}

The ``abline()`` function can be used to add a regression model line  by supplying as an argument the ``coef()`` values for intercept and slope estimates .These estimates can be inputted directly by using both functions in conjunction.

\large \begin{verbatim}
Fit1 =lm(y1~x);  coef(Fit1)
abline(coef(Fit1))
\end{verbatim}\large

\subsection{Adding a title }

It is good practice to label your scatterplots properly. You can specify the following argument
\begin{itemize}
* main="Scatterplot Example", This provides the plot with a title
* sub="Subtitle",                 This adds a subtitle
* xlab="X variable ",This command labels the x axis 
*    ylab="y variable ",This command labels the y-axis
\end{itemize}
We can also add text to each margin, using the ``mtext()`` command.  
We simply require the number of the side. (1 = bottom, 2=left,3=top,4=right). 
We can change the colour using the col argument.
\large \begin{verbatim}
plot(x,y,main="Scatterplot Example",   sub="subtitle",    xlab="X variable ", ylab="y variable ")
mtext("Enhanced Scatterplot", side=4,col="red ")
\end{verbatim}\large
Alternatively , we can also use the command title() to add a title to an existing scatterplot.
\large \begin{verbatim}
title(main="Scatterplot Example)
\end{verbatim}\large


\section{Combining plots}
It is possible to combine two plots. We used the graphical parameters command ``par()`` to create an array. 
Often we just require two plots side by side or above and below. We simply specify the numbers of rows and columns of this array using the ``mfrow} argument, passed as a vector.

\begin{verbatim}
par(mfrow=c(1,2))
plot(x,y1)# draw first plot
plot(x,y2)# draw second plot
par(mfrow=c(1,1))# reset to default setting.
\end{verbatim}

\section{Plot of single vectors}
If only one vector is specified i.e. ``plot(x)},  the plot created will simply be a scatter-plot of the values of x against their indices.

$plot(x)$
Suppose we wish to examine a trend that these points represent. We can connect each covariate using a line.

$plot(x, type = "l")$
If we wish to have both lines and points, we would input the following code. This is quite useful if we wish to see how a trend develops over time.
$plot(x, type = "b")$


















