#### {Useful Statistical Commands}

* ``mean()`` mean of a data set
* ``median()`` median of a data set
* ``length()`` Sample Size
* ``IQR()`` Inter-Quartile Range of a sample
* ``var()`` Variance of a sample
* ``sd()`` Standard Deviation  of a sample
* ``range()`` Range of a data set
* ``fivenum()`` Tukey's five number summary


#### {Set Theory Operations}

* ``union()`` union of sets A and B
* ``intersect()`` intersection of sets A and B
* ``setdiff()`` set difference A-B (order is important)


<pre>
<code>
x = 5:10
y = 8:12
union(x,y)
intersect(x,y)
setdiff(x,y)
setdiff(y,x)
<\code>
</pre>
%=============================================================================%


% R Class : Set Theory and Sampling

%# Sampling
%# - Sampling With Replacement
%# - Sampling Without Replacement

%=============================================================================%
#### {Set Theory with ``R} }


* Union
* Intersection
* Set Diffference

<pre>
<code>
X = 5:10
Y = 8:12
<\code>
</pre>

<pre>
<code>

union(X,Y)
# [1]  5  6  7  8  9 10 11 12
intersect(X,Y)
# [1]  8  9 10

<\code>
</pre>
%=============================================================================%


\section{R Class : Set Theory and Sampling}

%=============================================================================%
%# Sampling
%# - Sampling With Replacement
%# - Sampling Without Replacement

### Set Theory with ``R``


* Union
* Intersection
* Set Diffference

<pre>
<code>
X = 5:10
Y = 8:12
<\code>
</pre>
\end{frame}
%=============================================================================%

\frametitle{Set Theory with ``R} }
<pre>
<code>

union(X,Y)
# [1]  5  6  7  8  9 10 11 12
intersect(X,Y)
# [1]  8  9 10

<\code>
</pre>
\end{frame}
%=============================================================================%
\end{document}

#### {Set Theory Operations}

* ``union()`` union of sets A and B
* ``intersect()`` intersection of sets A and B
* ``setdiff()`` set difference A-B (order is important)


<pre>
<code>
x = 5:10
y = 8:12
union(x,y)
intersect(x,y)
setdiff(x,y)
setdiff(y,x)
<\code>
</pre>
%---------------------------------------------------------%
#### {The Birthday function}
The R command ``pbirthday()`` computes the probability of a coincidence of a number of randomly chosen people sharing a birthday, given that there are n people to choose from.
Suppose there are four people in a room. The probability of two of them sharing a birthday is computed as about 1.6 \%
<code>
> pbirthday(4)
[1] 0.01635591
<\code>

How many people do you need for a greater than 50\% chance of a shared birthday? (choose from 23,43,63,83)?
%---------------------------%



#### {Other Mathematical Functions}
%-----------------------------------------------------------------------%
Complex numbers
<pre>
<code>
x = -1 ;  sqrt(x)  ;  str(x) ; # variable is defined as numeric, not complex.
y = -1 +0i ;  sqrt(y)  ;  str(y) ;    #variable is defined as complex .

<\code>
</pre>
%-----------------------------------------------------------------------%
Trigonometric  Functions

<pre>
<code>
pi#returns the value of pi to six decimal places
sin(3.5*pi)# correct answer is -1
cos(3.5*pi)# correct answer is zero

<\code>
</pre>
%-----------------------------------------------------------------------%

%============================================================================================================== %
#### {Generating Random Numbers}
R is very useful for performing simulations.

<pre>
<code>
#generate a random number between 0 and 1
runif(1)
#generate four random numbers between 0 and 6 
runif(4,min=0,max=6)   
<\code>
</pre>
Random numbers can be discretized using the "floor()" or "Ceiling()" functions. Suppose we wish to simulate four throws of a dice.

X = ceiling (runif(4,min=0,max=6))
Y = floor (runif(4,min=1,max=7))
X+Y

%================================================================================================================= %
#### {Section 2: Basic Mathematical operations}
Trigonometric and power functions

Integration

<pre>
<code>
integrate(sin, lower =0, upper = 3)
integrate(dnorm, -1.96, 1.96) # standard normal distribution
integrate(dnorm, 0, Inf)   # standard normal distribution
<\code>
</pre>
Complex numbers 

``R} has a small number of built-in constants.
\begin{description}
\item[``LETTERS``]: the 26 upper-case letters of the Roman alphabet;

\item[``letters``]: the 26 lower-case letters of the Roman alphabet;

\item[``month.abb``]: the three-letter abbreviations for the English month names;

\item[``month.name``]: the English names for the months of the year;

\item[``pi``]: the ratio of the circumference of a circle to its diameter.

