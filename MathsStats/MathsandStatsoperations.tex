\documentclass[a4paper,12pt]{article}
%%%%%%%%%%%%%%%%%%%%%%%%%%%%%%%%%%%%%%%%%%%%%%%%%%%%%%%%%%%%%%%%%%%%%%%%%%%%%%%%%%%%%%%%%%%%%%%%%%%%%%%%%%%%%%%%%%%%%%%%%%%%%%%%%%%%%%%%%%%%%%%%%%%%%%%%%%%%%%%%%%%%%%%%%%%%%%%%%%%%%%%%%%%%%%%%%%%%%%%%%%%%%%%%%%%%%%%%%%%%%%%%%%%%%%%%%%%%%%%%%%%%%%%%%%%%
\usepackage{eurosym}
\usepackage{vmargin}
\usepackage{amsmath}
\usepackage{graphics}
\usepackage{epsfig}
\usepackage{subfigure}
\usepackage{fancyhdr}
%\usepackage{listings}
\usepackage{framed}
\usepackage{graphicx}

\setcounter{MaxMatrixCols}{10}
%TCIDATA{OutputFilter=LATEX.DLL}
%TCIDATA{Version=5.00.0.2570}
%TCIDATA{<META NAME="SaveForMode" CONTENT="1">}
%TCIDATA{LastRevised=Wednesday, February 23, 2011 13:24:34}
%TCIDATA{<META NAME="GraphicsSave" CONTENT="32">}
%TCIDATA{Language=American English}

\pagestyle{fancy}
\setmarginsrb{20mm}{0mm}{20mm}{25mm}{12mm}{11mm}{0mm}{11mm}
\lhead{Dublin ``R}} \rhead{10 April 2013}
\chead{Introduction to ``R} (Module A)}
%\input{tcilatex}


% http://www.norusis.com/pdf/SPC_v13.pdf
\begin{document}

\tableofcontents
\section{Mathematical and Statistical Commands}
%------------------------------------------------------%
1.8 Basic Maths Operations
The most commonly used mathematical operators are all supported by R. Here are a few
examples:
5 + 3 * 5 # Note the order of operations.
log (10) # Natural logarithm with base e=2.718282
log(8,2) # Log to the base 2
4^2 # 4 raised to the second power
7/2 # Division
factorial(4) #Factorial of Four
sqrt (25) # Square root
abs (3-7) # Absolute value of 3-7
pi # The mysterious number \\\
exp(2) # exponential function

%------------------------------------------------------%
R can be used for many mathematical operations, including
* Set Theory
* Trigonometry
* Complex Numbers
* Binomial Coecients
We will not go into any of these in great detail today.
%---------------------------------------------------------------------------------------%

#### {Basic Mathematical Calculations}
Basic operations
addition  +      
subtraction    	-
division   /     
multiplication   $\ast$
power     


#### *{Mathematical functions}
\begin{itemize}
* abs()       - Absolute value 
* exp()	   - The exponential
* log(,b)     - logarithm to the base "b". The default setting is the natural log.
\end{itemize}
\subsubsection*{trigonometrics functions}

pi            -  to 6 decimal places.

tan() ,cos(), sin()   - Common trigonometric functions

 
 Precision
 floor() 	  The floor function 		x
 ceiling()    The ceiling function       x

\begin{itemize} 
* round()      Round to the nearest integer
* round( ,2)  Round to two decimal places
\end{itemize} 
 
\begin{itemize}
* Exercise:  write $\pi$   with 4 decimal places.  
\end{itemize}

 
%=====================================================================================================================%
<pre>
<code>
	Complex numbers
	x = -1 ;  sqrt(x)  ;  str(x) ; 	# variable is defined as numeric, not complex.
	y = -1 +0i ;  sqrt(y)  ;  str(y) ;    	#variable is defined as complex .
	Trigonometric  Functions
	pi				#returns the value of pi to six decimal places
	sin(3.5*pi)			# correct answer is -1
	cos(3.5*pi)			# correct answer is zero
<\code>
</pre>
%=====================================================================================================================%

#### {Basic Calculations}
<code>
We will briefly look at how R accomplished basic calculations.


x*y			# multiplication
x/z			# division

x^2			# powers
sqrt(x)		# square root

exp(z)		 # exponentials   
log(y)		 # logarithms

pi             # returns the value of pi to six decimal places

<\code>

Complex numbers , Trigonometric  Functions and Binomial Coefficients


Binomial coefficients are computed using the choose() command.



<pre>
<code>
J = -1 ;  sqrt(J)  ;  str(J) ;      # variable is defined as numeric, not complex.
K = -1 +0i ;  sqrt(K)  ;  str(K) ;  # variable is defined as complex .


<\code>
</pre>
% sin(3.5*pi)             # correct answer is -1
% cos(3.5*pi)             # correct answer is zero
% choose(6,2)             # From 6 how many ways of choosing items.


%=========================================================================================================== %






#### {Useful Mathematical Operators}

\begin{itemize}
	* Factorials
	$n! = n \times n-1 \times \ldots \times 2 \times 1 $
	* Binomial Coefficients
	\[ { n \choose k }  = \frac{n!}{(n-k)! \times k!}\]
\end{itemize}
The ``R} commands are ``factorial()`` and ``choose()`` respectively.

Matrices and Linear Algebra
Factorials and permutations
\textbf{The Choose Function}

\[ { 6 \choose 3} =\frac{654}{321}= 20\]


<pre>
<code>
> factorial(6)
[1] 720
> choose(6,3)
[1] 20
> 
<\code>
</pre>

%--------------------%
#### {Mathematical Precision Functions}
Three commonly used mathematical precision functions are:
\begin{itemize}
	* Absolute Value Function $| x |$ - distance on the number line from zero.
	* Ceiling Function $\lceil x \rceil$ - rounds a value up to the nearest integer.
	* Floor Function  $\lfloor x \rfloor $ - rounds a value down to the nearest integer.
\end{itemize}
\begin{itemize}
	* ``floor()`` Floor function of x, $\lfloor x \rfloor$.
	* ``ceiling()`` Ceiling function of x, $\lceil x \rceil$.
	* ``round()`` Rounding a number to a specified number of decimal places.
\end{itemize}
%--------------------------------------------%

<pre>
	<code>
	pi
	floor(pi)
	ceiling(pi)
	<\code>
</pre>
<code>
> pi
[1] 3.141593
>
> floor(pi)
[1] 3
>
> ceiling(pi)
[1] 4
>
<\code>
We can also round numbers to a specified number of decimal places, using the ``round()`` command.
<pre>
	<code>
	round(pi,3)
	round(pi,2)
	<\code>
</pre>
<code>
> round(pi,3)
[1] 3.142
> round(pi,2)
[1] 3.14
<\code>


#### {Truncation and discretization}

The functions "floor" and "ceiling" can be used to discretize outcomes. In this instance we should use "ceiling".
\begin{itemize}
* ``ceiling(X)}
* ``floor(X)}
* ``round(X,2)}
\end{itemize}
X=ceiling(X)



The expected value is 3.5.

The variance from first principles we can calculate the variance using

#### {Managing Precision}

\begin{itemize}
	* ``floor()`` Floor function of x, $\lfloor x \rfloor$.
	* ``ceiling()`` Ceiling function of x, $\lceil x \rceil$.
	* ``round()`` Rounding a number to a specified number of decimal places.
\end{itemize}

%--------------------%
#### {Sequences}


#### {Sampling}

Types of Sampling
\begin{itemize}
	* Sampling With Replacement
	* Sampling Without Replacement
\end{itemize}

The R command we use to perform sampling is sample().

All elements in either X or Y

<pre>
	<code>
	
	
	> X=c(4,5)
	>
	> sample(X,2)
	[1] 4 5
	>
	> sample(X,1);sample(X,1);sample(X,1);
	[1] 4
	[1] 5
	[1] 5
	
	<\code>
</pre>
When x is a single value, the function sample() behaves differently.

<pre>
	<code>
	
	> Y=c(4)
	>
	> sample(Y,1)
	[1] 2
	> 
	> sample(Y,2)
	[1] 3 1
	<\code>
</pre>


%-----------------------------------------------------%
Generate a quick pick : pick 6 numbers from 1 to 42. ( Same number cant be selected more than once)

Generate five values from a die ( Same number can be selected more than once).

<code>
> Lotto = 1:42
> Dice = 1:6
> 
> sample(Lotto,6)
[1] 38 25 34 30 22 29
> 
> sample(Dice,5,replace = TRUE)
[1] 4 3 2 3 3
>
<\code>
%---------------------------------------------------------------------------%
\section{Sampling}
# 
# The R command we use to perform sampling is sample().
# 
# All elements in either X or Y



<code>
> X=c(4,5)
>
> sample(X,2)
[1] 4 5
>
> sample(X,1);sample(X,1);sample(X,1);
[1] 4
[1] 5
[1] 5

When x is a single value, the function sample() behaves differently.

> Y=c(4)
>
> sample(Y,1)
[1] 2
> 
> sample(Y,2)
[1] 3 1

<\code>

\begin{itemize}
	* Sampling With Replacement
	* Sampling Without Replacement
\end{itemize}
%-----------------------------------------------------%
Generate a quick pick : pick 6 numbers from 1 to 42. ( Same number cant be selected more than once)

Generate five values from a die ( Same number can be selected more than once).

<code>
> Lotto = 1:42
> Dice = 1:6
> 
> sample(Lotto,6)
[1] 38 25 34 30 22 29
> 
> sample(Dice,5,replace = TRUE)
[1] 4 3 2 3 3
>
<\code>
<p>
#### {Useful Statistical Commands}
\begin{itemize}
	* ``mean()`` mean of a data set
	* ``median()`` median of a data set
	* ``length()`` Sample Size
	* ``IQR()`` Inter-Quartile Range of a sample
	* ``var()`` Variance of a sample
	* ``sd()`` Standard Deviation  of a sample
	* ``range()`` Range of a data set
	* ``fivenum()`` Tukey's five number summary
\end{itemize}

#### {Set Theory Operations}
\begin{itemize}
	* ``union()`` union of sets A and B
	* ``intersect()`` intersection of sets A and B
	* ``setdiff()`` set difference A-B (order is important)
\end{itemize}

<pre>
	<code>
	x = 5:10
	y = 8:12
	union(x,y)
	intersect(x,y)
	setdiff(x,y)
	setdiff(y,x)
	<\code>
</pre>
%=============================================================================%


% R Class : Set Theory and Sampling

%# Sampling
%# - Sampling With Replacement
%# - Sampling Without Replacement

%=============================================================================%
#### {Set Theory with ``R} }
	
	\begin{itemize}
		* Union
		* Intersection
		* Set Diffference
	\end{itemize}
	<pre>
		<code>
		X = 5:10
		Y = 8:12
		<\code>
	</pre>

	<pre>
		<code>
		
		union(X,Y)
		# [1]  5  6  7  8  9 10 11 12
		intersect(X,Y)
		# [1]  8  9 10
		
		<\code>
	</pre>
%=============================================================================%


\section{R Class : Set Theory and Sampling}

%=============================================================================%
%# Sampling
%# - Sampling With Replacement
%# - Sampling Without Replacement

%=============================================================================%
\begin{frame}[fragile] 
	\frametitle{Set Theory with ``R} }
	
	\begin{itemize}
		* Union
		* Intersection
		* Set Diffference
	\end{itemize}
	<pre>
		<code>
		X = 5:10
		Y = 8:12
		<\code>
	</pre>
\end{frame}
%=============================================================================%
\begin{frame}[fragile] 
	\frametitle{Set Theory with ``R} }
	<pre>
		<code>
		
		union(X,Y)
		# [1]  5  6  7  8  9 10 11 12
		intersect(X,Y)
		# [1]  8  9 10
		
		<\code>
	</pre>
\end{frame}
%=============================================================================%
\end{document}

#### {Set Theory Operations}
\begin{itemize}
	* ``union()`` union of sets A and B
	* ``intersect()`` intersection of sets A and B
	* ``setdiff()`` set difference A-B (order is important)
\end{itemize}

<pre>
	<code>
	x = 5:10
	y = 8:12
	union(x,y)
	intersect(x,y)
	setdiff(x,y)
	setdiff(y,x)
	<\code>
</pre>
%---------------------------------------------------------%
#### {The Birthday function}
The R command ``pbirthday()`` computes the probability of a coincidence of a number of randomly chosen people sharing a birthday, given that there are n people to choose from.
Suppose there are four people in a room. The probability of two of them sharing a birthday is computed as about 1.6 \%
<code>
> pbirthday(4)
[1] 0.01635591
<\code>

How many people do you need for a greater than 50\% chance of a shared birthday? (choose from 23,43,63,83)?
%---------------------------%



#### {Other Mathematical Functions}
%-----------------------------------------------------------------------%
Complex numbers
<pre>
	<code>
	x = -1 ;  sqrt(x)  ;  str(x) ; 	# variable is defined as numeric, not complex.
y = -1 +0i ;  sqrt(y)  ;  str(y) ;    	#variable is defined as complex .

<\code>
</pre>
%-----------------------------------------------------------------------%
Trigonometric  Functions

<pre>
<code>
pi				#returns the value of pi to six decimal places
sin(3.5*pi)			# correct answer is -1
cos(3.5*pi)			# correct answer is zero

<\code>
</pre>
%-----------------------------------------------------------------------%

%============================================================================================================== %
#### {Generating Random Numbers}
R is very useful for performing simulations.

<pre>
<code>
#generate a random number between 0 and 1
runif(1)						
#generate four random numbers between 0 and 6				 
runif(4,min=0,max=6) 			  
<\code>
</pre>
Random numbers can be discretized using the "floor()" or "Ceiling()" functions. Suppose we wish to simulate four throws of a dice.

X = ceiling (runif(4,min=0,max=6))
Y = floor (runif(4,min=1,max=7))
X+Y

%================================================================================================================= %
#### {Section 2: Basic Mathematical operations}
Trigonometric and power functions

Integration

<pre>
<code>
integrate(sin, lower =0, upper = 3)
integrate(dnorm, -1.96, 1.96)					 # standard normal distribution
integrate(dnorm, 0, Inf)							   # standard normal distribution
<\code>
</pre>
Complex numbers 

``R} has a small number of built-in constants.
\begin{description}
	\item[``LETTERS}]: the 26 upper-case letters of the Roman alphabet;

	\item[``letters}]: the 26 lower-case letters of the Roman alphabet;

	\item[``month.abb}]: the three-letter abbreviations for the English month names;

	\item[``month.name}]: the English names for the months of the year;
	
\item[``pi}]: the ratio of the circumference of a circle to its diameter.

