
\maketitle

\section{Probability}
%-------------------------------------------------------------------------%
#### {Sequence of Integers}
A sequence of integers can be created using the “:” operator, specifying the upper and lower bounds on either side. This is very useful for Dice experiments. Importantly this sequence is constructed as a numeric vector.
<pre>
<code>
1:10
5:1
Dice = 1:6
Dice
class(Dice)
mode(Dice)
<\code>
</pre>

<code>
> 1:10
 [1]  1  2  3  4  5  6  7  8  9 10
> 5:1
[1] 5 4 3 2 1
> Dice = 1:6
> Dice
[1] 1 2 3 4 5 6
> class(Dice)
[1] "integer"
> mode(Dice)
[1] "numeric"
<\code>

%-------------------------------------------------------------------------%
#### {Playing with Dice}


Another important command is the ``sample()`` command. This command causes ``R} to select $n$ items from the specified vector. We can use it to simulate one roll of a die.
<pre>
<code>
Dice = 1:6
sample(Dice,1)
<\code>
</pre>
<code>
> sample(Dice,1)
[1] 2 
> 
> sample(Dice,1)
[1] 4 
>
> sample(Dice,1)
[1] 1
> 
> sample(Dice,1)
[1] 2
<\code>

%--------------------------------------------------------------------------------------%
#### {Sampling Without Replacement and With Replacement}
\noindent Suppose we wish to simulate two rolls of a die. Surely we just specify 2 in the argument of the ``sample()`` command. In fact we do, but this is not enough.
The ``sample()`` command works on the default basis of \textbf{\textit{sampling without replacement}}. That is to say: once a number has been selected, it can not be selected again.
<code>
> sample(Dice,6)
[1] 4 3 6 1 2 5
> sample(Dice,7)
Error in sample(Dice, 7) : 
  cannot take a sample larger than the population when 'replace = FALSE'
<\code>
Sampling without replacement is appropriate if we are selecting lottery numbers. You can only pick a particular number once.
<code>
> Lotto=1:42
> sample(Lotto,6)
[1]  2 38 19 10  3 27
> 
<\code>

Going back to multiple dice rolls, what we do here is to additionally specify ``replace=TRUE}. This specifies an experiment where there is \textbf{\textit{sampling with replacement}}. Let us simulate 20 dice rolls. We can also use the ``table()`` function to study the outcomes of the simulation. This command tells use how many times each outcome (number) came up. (This is called a \textbf{frequency table}). Out of interest, we will compute the mean value of the dice rolls using the ``mean()`` command.
<code>
> sample(Dice,20,replace=TRUE)
 [1] 4 1 3 2 5 4 4 4 6 5 6 6 5 6 6 5 1 1 1 3
>
> X= sample(Dice,20,replace=TRUE)
> table(X)
X
1 2 3 4 5 6 
4 4 2 3 3 4 
> mean(X)
[1] 3.45
<\code>


Let us roll dice 300 times.  Each of the six outcome is equally likely,  so we will expect 50 realisations of each number.  But that rarely happens, as \textbf{Sampling fluctation} comes into play.
In the table below, we see that the number 2 comes up thirteen more times than the number 3. Is there something wrong? A hidden bias in our code? No. It is just a random occurence.

%-----------------------------------%
<code>
> X=sample(Dice,300,replace=TRUE)
> table(X)
X
 1  2  3  4  5  6 
48 57 44 55 51 45 
> mean(X)
[1] 3.463333
 hist(X,breaks=c(0:6 + 0.5),col=c("red","blue","green"))
<\code>
 
A bar chart known as a \textbf{\textit{histogram}} to accompany this frequency table can be constructed using the following code. The command ``hist()`` is used to construct the histogram. The argument ``breaks} is used to specify the lower and upper bounds of each interval (the sides of each of the columns).
<pre>
<code>
 hist(X,breaks=c(0:6 + 0.5),col=c("red","blue","green"))
</pre>
<\code>

%--------------------------------------%
#### {Rolling Two Dice}
So far we have simulated the roll of one die. Many games would use two dice rolled together, with an outcome between 2 and 12. A first guess would simply be to specify a certain numbers sampled with replacement from a sequence of integers 2 to 12. Wrong! 
For one die, each outcome is equally likely. For a two dice experiment, there are differences in the probability of each outcome occuring, with 2 and 12  being the least likely outcomes, while 7 is the most likely outcome.
Let roll two dice $X$ and $Y$ seperately, and then add the two numbers from each together (calling this addition $Z$.)


<pre>
<code>
X=sample(Dice,300,replace=TRUE)
Y=sample(Dice,300,replace=TRUE)
Z=X+Y
table(Z)
hist(Z,breaks=c(1:12 + 0.5),col=c("red","blue","green","yellow"))
<\code>
</pre>
<code>
> X=sample(Dice,300,replace=TRUE)
> Y=sample(Dice,300,replace=TRUE)
> Z=X+Y
> table(Z)
Z
 2  3  4  5  6  7  8  9 10 11 12 
 8 18 19 38 41 56 32 28 26 23 11
\end{veratim}




%---------------------------------------------------------------------------------------------%
#### {Dice Rolls Sums Experiment}
Let work on the basis of 100 dice rolls, and for the sake of simplicity let us consider the \textbf{sum} of those 100 single die-rolls. Importantly, we are simulating a \textbf{fair} die. The lowest mathematially possible value is 100, while the highest possible value is 600. We expect to get a value of approximately 350, halfway between the minimum and maximum, but each time we perform the experiment there will be slight fluctuations.  Try the last line of the code below a few times. How many times do you get a figure less than 340 or greater than 360?
<pre>
<code>
X= sample(Dice,100,replace=TRUE)
sum(X)

#Equivalently
sum(sample(Dice,100,replace=TRUE))
<\code>
</pre>
<p>
#### {Simulation Study of Dice Rolls Sums Experiment}
Lets consider this Dice Roll Summation experiment. We will perform the experiment 100000 times, and see what sort of distribution of summations we get.
We will save the results in a vector called ``sums}.
<pre>
<code>
Sums=numeric()    # Initialize An Empty Vector
M=100000  # Number of Iterations

for (i in 1:M)
    {
     NewSum=sum(sample(Dice,100,replace=TRUE))
     Sums = c(Sums, NewSum)
     }
<\code>
</pre>

We can perform some basic statistical operations to study this vector. 
In particular we are interested in the extremes: How many times was there a summation less than 300, and how many times was there a summation greated than 400? (around 1.5\% probability in each case)

<code>
> length(Sums[Sums<300])
[1] 144
> length(Sums[Sums>400])
[1] 160  
<\code>

Lets us look at a histogram (a type of bar chart) of the ``Sums} vector ( Use around breaks =100 to specify more intervals). What sort of shape is this histogram?
<pre>
<code>
hist(Sums, breaks=100)
<\code>
</pre>

This is a ver crude introduction to the Central Limit Theorem. 
Even though the Dice Rolls are not normally distributed, the distribution of summations, are described in this experiment, are from a normally distributed sampling population. 
Also consider the probability of getting a sum more than 400. 
Recalling that dice simulation is for fair dice, the probability of getting a score more extreme than 400 is 1.5\% approximately. 
This provides (again crudely) an introduction to the idea of p-values , which are used a lot in statistical inference procedures. 

Suppose it was not certain whether a die was fair or crooked favouring higher values such as 4,5 and 6. The 100 roll experiment was performed and the score turned out to be 400.  
It would be a very unusual outcome for a fair die, but not impossible. 
For crooked dice, larger summations would be expected and a score of approximately 400 would be common. Would you think the die was fair or crooked.

Footnote: Arbitrarily, let us consider a crooked dice, where 4,5 and 6 are twice a likely to appear. Try out the following code.
<code>
CrookedDice=c(1,2,3,4,4,5,5,6,6)
sum(sample(CrookedDice,100,replace=TRUE))
<\code>

%----------------------------------------------------------------------------------------------------------------------------------------%
<p>
#### {The Birthday function}

The R command ``pbirthday()`` computes the probability of a coincidence of a number of randomly chosen people sharing a birthday, given that there are $n$ people to choose from.
Suppose there are four people in a room. The probability of two of them sharing a birthday can be computed as follows: (Answer:  about 1.6 \%)

<code>
> pbirthday(4)
[1] 0.01635591
<\code>



\noindent How many people do you need for a greater than 50\% chance of a shared birthday for n people? Many would make the guess 183 people.  Let us use the ``sapply()`` command. The first arguments is the data set (here a sequence of integers from 2 to 60) and then the command or function (here ``pbirthday()``, but specified without the brackets.
(starting from 2 – so the 5th element corresponds to 6 people, etc)
<code>
 > sapply(2:60,pbirthday)
 [1] 0.002739726 0.008204166 0.016355912 0.027135574
 [5] 0.040462484 0.056235703 0.074335292 0.094623834
 [9] 0.116948178 0.141141378 0.167024789 0.194410275
[13] 0.223102512 0.252901320 0.283604005 0.315007665
[17] 0.346911418 0.379118526 0.411438384 0.443688335
[21] 0.475695308 0.507297234 0.538344258 0.568699704
[25] 0.598240820 0.626859282 0.654461472 0.680968537
[29] 0.706316243 0.730454634 0.753347528 0.774971854
[33] 0.795316865 0.814383239 0.832182106 0.848734008
[37] 0.864067821 0.878219664 0.891231810 0.903151611
[41] 0.914030472 0.923922856 0.932885369 0.940975899
[45] 0.948252843 0.954774403 0.960597973 0.965779609
[49] 0.970373580 0.974431993 0.978004509 0.981138113
[53] 0.983876963 0.986262289 0.988332355 0.990122459
[57] 0.991664979 0.992989448 0.994122661
<\code>
The answer is 23 people (see entry 22)- probably much less than you thought!!




\end{document}
