
\section{The R Programming Language}

The R Programming Language is a statistical , data analysis , etc

R is a free software environment for statistical computing and graphics.


\section{Vector types}
``R} operates on named data structures. The simplest such structure is the
vector, which is a single entity consisting of an ordered collection of
Numbers or characters.

\begin{itemize}
*  Numeric vectors
*  Character vectors
*  Logical vectors
*  (also complex number vectors and colour vectors)
\end{itemize}

To create a vector, use the assignment operator and the concatenate function.
For numeric vectors, the values are simply numbers.

\begin{verbatim}
># week8.r
>NumVec<-c(10.4,5.6,3.1,6.4)
\end{verbatim}

Alternatively we can use the ``assign()`` command

For character vectors, the values are simply characters, specified with
quotation marks.A logical vectors is a vector whose elements are TRUE, FALSE or NA

\begin{verbatim}
>CharVec<-c(``blue", ``green", ``yellow")
>LogVec<-c(TRUE, FALSE)
\end{verbatim}

\section{Graphical data entry interface}

``Data.entry()`` is a useful  command for inputting or editing data sets. Any
changes are saved automatically (i.e. don’t need to use the assignment
operator). We can also used the ``edit()`` command, which calls the ``R Editor}.

\begin{verbatim}
>data.entry(NumVec)
>NumVec <- edit(NumVec)
\end{verbatim}

Another method of creating vectors is to use the following
\begin{verbatim}
numeric (length = n)
character (length = n)
logical (length = n)
\end{verbatim}
These commands create empty vectors, of the appropriate kind, of length $n$. You can then use the graphical data entry interface to populate your data sets.

\subsubsection{Accessing specified elements of a vector}

The $n$th element of vector ``Vec" can be accessed by specifying its index when
calling ``Vec".
\begin{verbatim}>Vec[n]
\end{verbatim}
A sequence of  elements of vector ``Vec" can be accessed by specifying its index
when calling ``Vec".
\begin{verbatim}>Vec[l:u]
\end{verbatim}
Omitting and deleting the $n$th element of vector ``Vec"
\begin{verbatim}
>Vec[-n]
>Vec <- Vec[-n]
\end{verbatim}

%\subsection{Reading data}


\subsection{inputting data}
Concatenation

\subsection{using help}

\begin{framed}
\begin{verbatim}
?mean
\end{verbatim}
\end{framed}

%\subsection{Adding comments}


\newpage
\section{Managing Precision}
\begin{itemize}
*  ``floor()`` - 
*  ``ceiling()`` - 
*  ``round()`` - 
*  ``as.integer()`` -
\end{itemize}

\begin{framed}
\begin{verbatim}
> pi
[1] 3.141593
> floor(pi)
[1] 3
> ceiling(pi)
[1] 4
> round(pi,3)
[1] 3.142
> as.integer(pi)
[1] 3
\end{verbatim}
\end{framed}

\subsection{exponentials, powers and logarithms}
\large
\begin{framed}
\begin{verbatim}
>x^y
>exp(x)
>log(x)
>log(y)
#determining the square root of x
>sqrt(x)
\end{verbatim}
\end{framed}

%============================================================================================== %

\subsection{Packages}
The capabilities of ``R} are extended through user-submitted packages, which allow specialized statistical techniques, graphical devices, as well as and
import/export capabilities to many external data formats.



\large
\subsection{vectors}
\large
\begin{framed}
\begin{verbatim}
R handles vector objects quite easily and intuitively.

> x<-c(1,3,2,10,5)    #create a vector x with 5 components
> x
[1]  1  3  2 10  5
> y<-1:5              #create a vector of consecutive integers
> y
[1] 1 2 3 4 5
> y+2                 #scalar addition
[1] 3 4 5 6 7
> 2*y                 #scalar multiplication
[1]  2  4  6  8 10
> y^2                 #raise each component to the second power
[1]  1  4  9 16 25
> 2^y                 #raise 2 to the first through fifth power
[1]  2  4  8 16 32
> y                   #y itself has not been unchanged
[1] 1 2 3 4 5
> y<-y*2
> y                   #it is now changed
[1]  2  4  6  8 10
\end{verbatim}
\end{framed}
\large

\subsubsection{Misc}
``seq()`` and ``rep()`` are useful commands for constructing vectors with a certain pattern.

%\end{document}

\subsection{Matrices}
A matrix refers to a numeric array of rows and columns.

One of the easiest ways to create a matrix is to combine vectors of equal
length using cbind(), meaning "column bind". Alternatively one can use rbind(), meaning ``row bind".


\large
\begin{framed}
\begin{verbatim}
> X=c(1,4,5,7,8,9,5,8,9)
> mean(X);median(X)       #mean and median of vector
[1] 6.222222
[2] 7
> sd(X)                   #standard deviation of Vector
[1] 2.682246
> length(X)               #sample size of vector
[1] 9
> sum(X)
[1] 56
> X^2
[1]  1 16 25 49 64 81 25 64 81
> rev(X)
[1] 9 8 5 9 8 7 5 4 1
> sort(X)                 #items in ascending order
[1] 1 4 5 5 7 8 8 9 9
> X[1:5]
[1] 1 4 5 7 8
\end{verbatim}
\end{framed}
\large


\section{Summary Statistics}
The ``R} command ``summary()`` returns a summary statistics for a simple dataset.
The ``R} command ``fivenum()`` returns a summary statistics for a simple dataset, but without the mean.
Also, the quartiles are computed a different way.

\large
\begin{framed}
\begin{verbatim}
> summary(mtcars$mpg)
Min. 1st Qu.  Median    Mean 3rd Qu.    Max.
10.40   15.43   19.20   20.09   22.80   33.90 
>
> fivenum(mtcars$mpg)
[1] 10.40 15.35 19.20 22.80 33.90
\end{verbatim}
\end{framed}
\large




\section{Bivariate Data}
\large \begin{framed}
\begin{verbatim}
> Y=mtcars$mpg
> X=mtcars$wt
>
> cor(X,Y)          #Correlation
[1] -0.8676594
>
> cov(X,Y)          #Covariance
[1] -5.116685
\end{verbatim}
\end{framed}\large


\section{Histograms}
Histograms can be created using the ``hist()`` command.
To create a histogram of the car weights from the Cars93 data set
\large
\begin{framed}
\begin{verbatim}
hist(mtcars$mpg, main="Histogram of MPG (Data: MTCARS) ")
\end{verbatim}
\end{framed}\large
``R} automatically chooses the number and width of the bars. We can
change this by specifying the location of the break points.
\large
\begin{framed}
\begin{verbatim}hist(Cars93$Weight, breaks=c(1500, 2050, 2300, 2350, 2400,
2500, 3000, 3500, 3570, 4000, 4500), xlab="Weight",
main="Histogram of Weight")
\end{verbatim}
\end{framed}\large



\section{Boxplot}
Boxplots can be used to identify outliers.

By default, the ``boxplot()`` command sets the orientation as vertical. By adding the argument ``horizontal=TRUE}, the orientation can be changed to horizontal.
\large
\begin{framed}
\begin{verbatim}
boxplot(mtcars$mpg, horizontal=TRUE, xlab="Miles Per Gallon",
main="Boxplot of MPG")
\end{verbatim}
\end{framed}\large

\begin{figure}
% Requires \usepackage{graphicx}
\includegraphics[scale=0.4]{MTCARSboxplot.png}\\
\caption{Boxplot}\label{boxplot}
\end{figure}



\newpage
\chapter{Advanced R code}
\section{Data frame}
A Data frame is
\subsection{Merging Data frames}

\section{Functions}
Syntax to define functions

\begin{framed}
\begin{verbatim}
myfct <- function(arg1, arg2, ...) { function_body }
\end{verbatim}
\end{framed}
The value returned by a function is the value of the function body, which is usually an unassigned final expression, e.g.: return()

Syntax to call functions
\begin{framed}
\begin{verbatim}
myfct(arg1=..., arg2=...)
\end{verbatim}
\end{framed}


\section{Time and Date}
It is useful . The length of time a program takes is interesting.


\begin{framed}
\begin{verbatim}
date() # returns the current system date and time
\end{verbatim}
\end{framed}


\section{The Apply family}

Sometimes want to apply a function to each element of a
vector/data frame/list/array.
\\
Four members: lapply, sapply, tapply, apply
\\
lapply: takes any structure and gives a list of results (hence
the `l')
\\
sapply: like lapply, but tries to simplify the result to a
vector or matrix if possible (hence the `s')
\\
apply: only used for arrays/matrices
\\
tapply: allows you to create tables (hence the `t') of values
from subgroups defined by one or more factors.
\newpage
\chapter{Data Visualization}
\section{Plots}
This section is an introduction for producing simple graphs with
the R Programming Language.
\begin{itemize}
*  Line Charts  *  Bar Charts *  Histograms *  Pie
Charts *  Dotcharts
\end{itemize}

\section{The R Programming Language}

The R Programming Language is a statistical , data analysis , etc

R is a free software environment for statistical computing and graphics.

\section{Writing R scripts}
Editing your R script ``R Editor".
\begin{itemize}
*  On the menu of the R console, click on ‘file’.
*  Select ‘open script’ or ‘new script’ as appropriate.
*  Navigate to your working directory and select your ‘``.R}’ file
*  A new dialogue box ````the R editor}" will open up.
*  Input or select code you wish to compile.
*  To compile this code, highlight it. Click the ‘edit button’ on the menu.
*  Select either ``Run Line" or ``Run Selection or All".
*  Your code should now compile.
*  To save your code, clink on ``file" and then ````save as}".
*  Save the file with the ````.R}" extension to your working directory.
\end{itemize}

\section{Vector types}
``R} operates on named data structures. The simplest such structure is the
vector, which is a single entity consisting of an ordered collection of
Numbers or characters.

\begin{itemize}
*  Numeric vectors
*  Character vectors
*  Logical vectors
*  (also complex number vectors and colour vectors)
\end{itemize}

To create a vector, use the assignment operator and the concatenate function.
For numeric vectors, the values are simply numbers.

\begin{verbatim}
># week8.r
>NumVec<-c(10.4,5.6,3.1,6.4)
\end{verbatim}

Alternatively we can use the ``assign()`` command

For character vectors, the values are simply characters, specified with
quotation marks.A logical vectors is a vector whose elements are TRUE, FALSE or NA

\begin{verbatim}
>CharVec<-c(``blue", ``green", ``yellow")
>LogVec<-c(TRUE, FALSE)
\end{verbatim}

\section{Graphical data entry interface}

``Data.entry()`` is a useful  command for inputting or editing data sets. Any
changes are saved automatically (i.e. don’t need to use the assignment
operator). We can also used the ``edit()`` command, which calls the ``R Editor}.

\begin{verbatim}
>data.entry(NumVec)
>NumVec <- edit(NumVec)
\end{verbatim}

Another method of creating vectors is to use the following
\begin{verbatim}
numeric (length = n)
character (length = n)
logical (length = n)
\end{verbatim}
These commands create empty vectors, of the appropriate kind, of length $n$. You can then use the graphical data entry interface to populate your data sets.

\subsubsection{Accessing specified elements of a vector}

The $n$th element of vector ``Vec" can be accessed by specifying its index when
calling ``Vec".
\begin{verbatim}>Vec[n]
\end{verbatim}
A sequence of  elements of vector ``Vec" can be accessed by specifying its index
when calling ``Vec".
\begin{verbatim}>Vec[l:u]
\end{verbatim}
Omitting and deleting the $n$th element of vector ``Vec"
\begin{verbatim}
>Vec[-n]
>Vec <- Vec[-n]
\end{verbatim}

\section{Reading data}


\subsection{inputting data}
Concatenation

\subsection{using help}

?mean

\subsection{Adding comments}

\subsection{Packages}
The capabilities of R are extended through user-submitted packages, which allow specialized statistical techniques, graphical devices, as well as and
import/export capabilities to many external data formats.

\section{Managing Precision}
\begin{itemize}
*  ``floor()`` - 
*  ``ceiling()`` - 
*  ``round()`` - 
*  ``as.integer()`` -
\end{itemize}

\begin{framed}
\begin{verbatim}
> pi
[1] 3.141593
> floor(pi)
[1] 3
> ceiling(pi)
[1] 4
> round(pi,3)
[1] 3.142
> as.integer(pi)
[1] 3
\end{verbatim}
\end{framed}

\section{Basic Operations}
\subsection{Complex numbers}
\subsection{Trigonometric functions}
\section{Matrices}

%\end{document}


\subsubsection{exponentials, powers and logarithms}
\begin{framed}
\begin{verbatim}
>x^y
>exp(x)
>log(x)
>log(y)
#determining the square root of x
>sqrt(x)
\end{verbatim}
\end{framed}

\subsection{vectors}
\begin{framed}
\begin{verbatim}
R handles vector objects quite easily and intuitively.

> x<-c(1,3,2,10,5)    #create a vector x with 5 components
> x
[1]  1  3  2 10  5
> y<-1:5              #create a vector of consecutive integers
> y
[1] 1 2 3 4 5
> y+2                 #scalar addition
[1] 3 4 5 6 7
> 2*y                 #scalar multiplication
[1]  2  4  6  8 10
> y^2                 #raise each component to the second power
[1]  1  4  9 16 25
> 2^y                 #raise 2 to the first through fifth power
[1]  2  4  8 16 32
> y                   #y itself has not been unchanged
[1] 1 2 3 4 5
> y<-y*2
> y                   #it is now changed
[1]  2  4  6  8 10
\end{verbatim}
\end{framed}

\subsubsection{Misc}
``seq()`` and ``rep()`` are useful commands for constructing vectors with a certain pattern.

%\end{document}

\subsection{Matrices}
A matrix refers to a numeric array of rows and columns.

One of the easiest ways to create a matrix is to combine vectors of equal
length using cbind(), meaning "column bind". Alternatively one can use rbind(), meaning ``row bind".


\subsubsection{Matrices Inversion}
\subsubsection{Matrices Multiplication}


\subsection{Data frame}
A Data frame is
\newpage

%------------------------------------------------------------------------------------------------%

\chapter{Descriptive Statistics}

\section{Basic Statistics}

\large
\begin{framed}
\begin{verbatim}
> X=c(1,4,5,7,8,9,5,8,9)
> mean(X);median(X)       #mean and median of vector
[1] 6.222222
[2] 7
> sd(X)                   #standard deviation of Vector
[1] 2.682246
> length(X)               #sample size of vector
[1] 9
> sum(X)
[1] 56
> X^2
[1]  1 16 25 49 64 81 25 64 81
> rev(X)
[1] 9 8 5 9 8 7 5 4 1
> sort(X)                 #items in ascending order
[1] 1 4 5 5 7 8 8 9 9
> X[1:5]
[1] 1 4 5 7 8
\end{verbatim}
\end{framed}
\large


\section{Summary Statistics}
The ``R} command ``summary()`` returns a summary statistics for a simple dataset.
The ``R} command ``fivenum()`` returns a summary statistics for a simple dataset, but without the mean.
Also, the quartiles are computed a different way.

\large
\begin{framed}
\begin{verbatim}
> summary(mtcars$mpg)
Min. 1st Qu.  Median    Mean 3rd Qu.    Max.
10.40   15.43   19.20   20.09   22.80   33.90 
>
> fivenum(mtcars$mpg)
[1] 10.40 15.35 19.20 22.80 33.90
\end{verbatim}
\end{framed}
\large




\section{Bivariate Data}
\large \begin{framed}
\begin{verbatim}
> Y=mtcars$mpg
> X=mtcars$wt
>
> cor(X,Y)          #Correlation
[1] -0.8676594
>
> cov(X,Y)          #Covariance
[1] -5.116685
\end{verbatim}
\end{framed}\large


\section{Histograms}
Histograms can be created using the ``hist()`` command.
To create a histogram of the car weights from the Cars93 data set
\large
\begin{framed}
\begin{verbatim}
hist(mtcars$mpg, main="Histogram of MPG (Data: MTCARS) ")
\end{verbatim}
\end{framed}\large
``R} automatically chooses the number and width of the bars. We can
change this by specifying the location of the break points.
\large

\begin{verbatim}hist(Cars93$Weight, breaks=c(1500, 2050, 2300, 2350, 2400,
2500, 3000, 3500, 3570, 4000, 4500), xlab="Weight",
main="Histogram of Weight")
\end{verbatim}



\section{Boxplot}
Boxplots can be used to identify outliers.

By default, the ``boxplot()`` command sets the orientation as vertical. By adding the argument ``horizontal=TRUE}, the orientation can be changed to horizontal.
\large
\begin{framed}
\begin{verbatim}
boxplot(mtcars$mpg, horizontal=TRUE, xlab="Miles Per Gallon",
main="Boxplot of MPG")
\end{verbatim}
\end{framed}

%\begin{figure}
%  Requires \usepackage{graphicx}
%  \includegraphics[scale=0.4]{MTCARSboxplot.png}\\
%  \caption{Boxplot}\label{boxplot}
%\end{figure}
