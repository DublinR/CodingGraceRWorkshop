
documentclass{beamer}

usepackage{amsmath}
usepackage{amssymb}
usepackage{framed}
usepackage{subfiles}

begin{document}
	
%ESRItalk

The American Community Survey distributes downloadable data about United States communities. 
Download the 2006 microdata survey about housing for the state of Idaho using texttt{download.file()} from here: 

<code>
https://dl.dropbox.com/u/7710864/data/csv_hid/ss06hid.csv

or here:

https://spark-public.s3.amazonaws.com/dataanalysis/ss06hid.csv 
</code>
and load the data into texttt{R}. 

%=========================================================================== %

% You will use this data for the next several questions. 

noindent textbf{textit{Code Book}}
The code book, describing the variable names is here: 

<code>
https://dl.dropbox.com/u/7710864/data/PUMSDataDict06.pdf
</code>

%=========================================================================== %


How many housing units in this survey were worth more than $1,000,000?
%              *  * 53             *  * 

%<pre>
%	<code>
%	# Download 2006 microdata survey 
%	# re: housing for Idaho using download.file()
%	# setwd("~/DA")
%	download.file(
%	'https://spark-public.s3.amazonaws.com/dataanalysis/ss06hid.csv',
%	"ss06hid.csv", method="curl")
%	
%	# Download the code book:
%	# download.file(
%	'https://spark-public.s3.amazonaws.com/dataanalysis/PUMSDataDict06.pdf',
%	"PUMSDataDict06.pdf", method="curl")
%	</code>
%</pre>
<p>



%=========================================================================================== %



<pre>
	<code>	
	# load the data into R
	idahoData <- read.csv("ss06hid.csv", header=TRUE)
	
	# Is it just Idaho data?
	table(idahoData$ST)
	#Check the PDF - what does 16 mean?
	
	#any missing data?
	summary(idahoData$ST)
	
	# How many housing units are worth
	# more than $1,000,000?
	table(idahoData$TYPE,idahoData$VAL)
	</code>
</pre>
<p>



%=========================================================================================== %



<pre>
	<code>
	#from local files
	idahoData <- read.csv("daquiz2.csv", header=TRUE)
	
	</code>
</pre>
<p>



%=========================================================================================== %


frametitle{Question 4}


	         * Use the data you loaded from Question 3. 
	         * Consider the variable FES. 
	         * Which of the "tidy data" principles does this variable violate?


%%READY
%textbf{textit{Revision}}
%What are the three characteristics of tidy data?
%
%
%	         * ``textit{textbf{Tidy data}}" by Hadley Wickham (RStudio)
%	         * Submission to Journal of Statistical Software
%	         * (http://vita.had.co.nz/papers/tidy-data.pdf)
%
%Three Principles from Hadley Wickham's paper
%
%	        * [1.] Each variable forms a column, 
%	        * [2.] Each observation forms a row, 
%	        * [3.] Each table/file stores data about one kind of observation.
%

<pre> 
	<code>
	# let's look!
	unique(idahoData$FES)
	</code>
</pre>
<p> 


%-----------------------------------------------------------------%

%<p>
####        * { Question 5 }



textbf{Options}

	        * [(i)]  Each tidy data table contains information about only one type of observation.
	(Not so)
	
	        * [(ii)]  Each variable in a tidy data set has been transformed to be interpretable.
	(No)
	
	        * [(iii)]  Tidy data has no missing values.
	
	        * [(iv)]  Tidy data has one variable per column.


%-----------------------------------------------------------------%

%<p>
####        * { Question 5 }

Use the data you loaded from Question 3. 


	         * How many households have 3 bedrooms and and 4 total rooms? 
	         * How many households have 2 bedrooms and 5 total rooms? 
	         * How many households have 2 bedrooms and 7 total rooms?

<pre>
	<code>
	#USING TABLE
	#Rooms on Rows , Bedrooms on Columns
	#dnn adds dimension names
	
	table(idahoData$RMS,idahoData$BDS,
	    dnn=list("RMS","BDS"))
	
	</code>
</pre>
<p>


%============================================================= %

Another Way of Doing it
<pre>
	<code>
	# How many households have 3 bedrooms and 4 total rooms?
	nrow(idahoData[!is.na(idahoData$BDS) & idahoData$BDS==3 &
	!is.na(idahoData$BDS) & idahoData$RMS==4,])
	
	# How many households have 2 bedrooms and 5 total rooms?
	nrow(idahoData[!is.na(idahoData$BDS) & idahoData$BDS==2 &
	!is.na(idahoData$BDS) & idahoData$RMS==5,])
	
	# How many households have 2 bedrooms and 7 total rooms?
	nrow(idahoData[!is.na(idahoData$BDS) & idahoData$BDS==2 &
	!is.na(idahoData$BDS) & idahoData$RMS==7,])
	
	</code>
</pre>
<p>
%              *  * 148, 386, 49             *  * 


%-----------------------------------------------------------------%

%<p>
####        * {Question 6}


	         * Use the data from Question 3. 
	         * Create a logical vector that identifies the households on greater than 10 acres who sold more than $10,000 worth of agriculture products. 
	         * Assign that logical vector to the variable `texttt{agricultureLogical}`. 
	         * Apply the `texttt{which()} function like this to identify the rows of the data frame where the logical vector is `TRUE`.



%====================================================== %

<pre> 
	<code>
	# Like this (this wont run yet)
	which(agricultureLogical) 
	</code>
</pre>
<p> 

%====================================================== %

What are the first 3 values that result?

<pre>
<code>
	# Showing off a bit
	q6cols <- c("ACR", "AGS")
	which(names(idahoData) %in% q6cols)  
	
	# logical vector
	agricultureLogical <- idahoData$ACR==3 & idahoData$AGS==6
	
	# and:
	which(agricultureLogical) 
	</code>
</pre>
<p> 

%             *  * 125, 238, 262             *  * 

%====================================================== %

	
frametitle{Question 7}


	         * Use the data from Question 3. 
	         * Create a logical vector that identifies the households on greater than 10 acres who
	sold more than $10,000 worth of agriculture products. 
	         * Assign that logical vector to the variable texttt{agricultureLogical}. 
	         * Apply the texttt{which()} function like this to identify the rows of the 
	data frame where the logical vector is TRUE and assign it to the variable indexes. 



%====================================================== %

	
<pre> 
<code>
	indexes =  which(agricultureLogical) 
</code>
</pre>
<p> 

If your data frame for the complete data is called texttt{dataFrame} you can create a data frame 
with only the above subset with the command: 


%====================================================== %

	
<pre> 
	<code>
	subsetDataFrame  = dataFrame[indexes,] 
	</code>
</pre>
<p> 

noindent Note that we are subsetting this way because the NA values in the variables 
will cause problems if you subset directly with the logical statement. 



%====================================================== %

	
noindent How many households in the subsetDataFrame have a missing value for the mortgage status 
(MRGX) variable?

<pre> 
	<code>
	indexes <- which(agricultureLogical)
	subsetIdahoData <- idahoData[indexes,]
	
	# And then:
	nrow(subsetIdahoData[is.na(subsetIdahoData$MRGX),])
	</code>
</pre>
<p> 




%====================================================== %

	
frametitle{Question 8}

	         * Use the data from Question 3.
	         * Apply `texttt{strsplit()}` to split all the names of the data frame on the characters "wgtp". 
	         * What is the value of the 123 element of the resulting list?


<pre> 
<code>
	List <- strsplit(names(idahoData), "wgtp")
	List[123]
</code>
</pre>
<p> 

%             *  * "" "15"             *  * 

%====================================================== %

frametitle{Question 9}

What are the 0% and 100% quantiles of the variable texttt{YBL}? Is there anything wrong with these values?
textit{ Hint: you may need to use the texttt{na.rm} parameter.}

<pre> 
	<code>
	quantile(idahoData$YBL, na.rm=TRUE)
	#  0%  25%  50%  75% 100% 
	#  -1    3    5    7   25 
	</code>
</pre>
<p> 



%====================================================== %
	
frametitle{Question 10}

In addition to the data from Question 3, the American Community Survey also collects data about populations. 
Using `texttt{download.file()}`, download the population record data from: 

<code>
https://dl.dropbox.com/u/7710864/data/csv_hid/ss06pid.csv 

or here:

https://spark-public.s3.amazonaws.com/dataanalysis/ss06pid.csv
</code>


%====================================================== %

	

	         * Load the data into texttt{R}. Assign the housing data from Question 3 to a data frame `texttt{housingData}` and the population data from above to a data frame `populationData`.
	
	         * Use the merge command to merge these data sets based only on the common identifier "SERIALNO". 
	
	         * What is the dimension of the resulting data set? 

%[OPTIONAL] For fun, you might look at the data and see what happened when they merged.


%========================================================================================== %

	
%	download.file(
%	'https://spark-public.s3.amazonaws.com/dataanalysis/ss06pid.csv',
%	'ss06pid.csv', method='curl')
%	
textbf{Merging Data Sets}
<pre> 
	<code>

	housingData <- read.csv("ss06hid.csv", header=TRUE)
	popuData <- read.csv("ss06pid.csv", header=TRUE)
	
	dim(merge(housingData, 
	popuData, by="SERIALNO", all=TRUE))
	</code>
</pre>
<p> 



end{document}
