\documentclass{beamer}

\usepackage{amsmath}
\usepackage{graphics}
\usepackage{framed}
\usepackage{amssymb}

\begin{document}
%==============================================================================================%
\begin{frame}
	% % SLIDE 1 - COVER SLIDE
	\begin{figure}
		\centering
		\includegraphics[width=0.7\linewidth]{Rlogo}
	\end{figure}
	\LARGE
	\[ \mbox{Using R for Economic and Social Research} \]	
	
	
	
\end{frame}



	%=================================================================== %
	\begin{frame}[fragile]
		
		
		\begin{figure}
			\centering
			\includegraphics[width=0.97\linewidth]{CRAN}
			\caption{Comprehensive R Archive Network}
			
		\end{figure}
		
		
	\end{frame}
	

	%=================================================================== %
	\begin{frame}	
		\frametitle{R Packages}
		
		\begin{itemize}
			\item ``10 R packages I wish I knew about earlier" - Drew Conway (Yhat.com, February 2013)
			\bigskip \item ``The HadleyVerse" - Hadley Wickham
			\begin{itemize}
				
				\item  ggplot2, dplyr, reshape, lubridate, stringr
				
				\item  With Romain Francois, Dianne Cook and Garret Grolemund.
			\end{itemize}
			\bigskip
			\item Dr Brendan Haplin (UL): lme4 ,TraMineR, Gelman's arm, MASS, foreign. 
			\bigskip
			\item Shiny - Web Applications with \texttt{R}
		\end{itemize}
	\end{frame}
	
	
	
	
	%=================================================================== %
	\begin{frame}[fragile]
		\Large
		\begin{framed}
			\begin{verbatim}
			"As usual, NZ is ahead of the pack and 
			@statisticsnz is the first govt 
			statistics agency to deploy 
			rstudio server pro! #rstats"
			
			Hadley Wickham, 
			(11th December).
			\end{verbatim}
		\end{framed}
		
	\end{frame}
	
	% **number of rows = 15451, number of columns = 426**
	
	
	
	%============================================================================= %
	\begin{frame}1 Introduction to R 3
		\begin{itemize}
			\item[1.1] Installing R      
			\item[1.2] Command Line Interface     
			\item[1.3] The Assignment operator     
			%1.3.1 Reserved Words  
			\item[1.4] Commenting      
			\item[1.5] Defining Variables     
			\item[1.6] Help Functions      
			\item[1.7] The \texttt{help.start()} command     
			\item[1.8] Basic Maths Operations     
			\item[1.9] Basic R Editor      
		\end{itemize}
	\end{frame}
	%============================================================================= %
	\begin{frame}
		1.10 Built-In Data Sets      
		1.11 The summary() command     
		1.12 Working directories      
		1.13 Coming Unstuck    
		1.14 Quitting the R environment   
		1.15 Data Objects  
		1.16 Listing all items in a workspace     
		1.17 Removing items   
		1.18 Saving and Loading R Data Objects    
	\end{frame}
	%============================================================================= %
	\begin{frame}
		\frametitle{Introduction to R (Continued) }
		\begin{itemize}
			\item[2.1] Three particularly useful commands    
			\item[2.2] Changing GUI options     
			\item[2.3] Colours      
			\item[2.4] Use of the Semi-Colon Operator     
			\item[2.5] The apropos() Function     
			\item[2.6] History       
			\item[2.7] The sessionInfo() Function     
			\item[2.8] Time and date functions     
			\item[2.9] Logical States      
			\item[2.10] Missing Data      
			\item[2.11] Files in the Working Directory     
		\end{itemize}
	\end{frame}
	%============================================================================= %
	\begin{frame}
		3 Inspecting a Data Set 11
		3.1 Dimensions of a data set . . . . . . . . . . . . . . . . . . . . . . . . . . . . . . . 11
		3.2 The summary() command . . . . . . . . . . . . . . . . . . . . . . . . . . . . . . 12
		3.3 Structure of a Data Object . . . . . . . . . . . . . . . . . . . . . . . . . . . . . . 12
	\end{frame}
	%============================================================================= %
	\begin{frame}
		4 Packages 13
		4.1 Packages . . . . . . . . . . . . . . . . . . . . . . . . . . . . . . . . . . . . . . . . 13
		4.2 Using and Installing packages . . . . . . . . . . . . . . . . . . . . . . . . . . . . 13
		4.2.1 Version of R . . . . . . . . . . . . . . . . . . . . . . . . . . . . . . . . . . 13
	\end{frame}
	%============================================================================= %
	\begin{frame}
		5 Data Creation, Data Entry, Data Import and Export 14
		5.1 The c() command . . . . . . . . . . . . . . . . . . . . . . . . . . . . . . . . . . 14
		5.1.1 Vector of Numeric Values . . . . . . . . . . . . . . . . . . . . . . . . . . 14
		5.1.2 Vector of Character Values . . . . . . . . . . . . . . . . . . . . . . . . . . 14
		5.1.3 Vector of Logical Values . . . . . . . . . . . . . . . . . . . . . . . . . . . 14
		5.2 The scan() command . . . . . . . . . . . . . . . . . . . . . . . . . . . . . . . . 14
		5.2.1 Characters with the scan() command . . . . . . . . . . . . . . . . . . . 15
		5.3 Using the data editor . . . . . . . . . . . . . . . . . . . . . . . . . . . . . . . . . 15
		5.4 Spreadsheet Interface . . . . . . . . . . . . . . . . . . . . . . . . . . . . . . . . . 15
		
	\end{frame}
	%==============================================================================================%
	\begin{frame}
		
		\textbf{Introduction to R}
		Source: R project website (http://www.r-project.org)
		\begin{itemize}
			\item R is a language and environment for statistical computing and graphics. It is a GNU project
			which is similar to the S language and environment which was developed at Bell Laboratories
			(formerly AT\&T, now Lucent Technologies) by John Chambers and colleagues. 
			\item R can be considered
			as a different implementation of S. There are some important differences, but much
			code written for S runs unaltered under R.
		\end{itemize}
		
	\end{frame}
	%==============================================================================================%
	\begin{frame}
		\begin{itemize}
			\item R provides a wide variety of statistical (linear and nonlinear modelling, classical statistical tests,
			time-series analysis, classification, clustering, ...) and graphical techniques, and is highly extensible.
			The S language is often the vehicle of choice for research in statistical methodology,
			and R provides an Open Source route to participation in that activity.
			\item One of R’s strengths is the ease with which well-designed publication-quality plots can be
			produced, including mathematical symbols and formulae where needed. 
			\item Great care has been
			taken over the defaults for the minor design choices in graphics, but the user retains full control.
			\item R is available as Free Software under the terms of the Free Software Foundation’s GNU General
			Public License in source code form. It compiles and runs on a wide variety of UNIX platforms
			and similar systems (including FreeBSD and Linux), Windows and MacOS.
		\end{itemize}
	\end{frame}
	%==============================================================================================%
	\begin{frame}
		
		\texttt{R} is a programming environment that
		\begin{itemize}
			\item uses a well-developed but simple programming language
			\item allows for rapid development of new tools according to user demand
			\item these tools are distributed as packages, which any user can download to customize the R
			environment.
			
		\end{itemize}
	\end{frame}
	%==============================================================================================%
	\begin{frame}
		
		Base R and most R packages are available for download from the Comprehensive R Archive Network
		(CRAN) cran.r-project.org. Base R comes with a number of basic data management,
		analysis, and graphical tools R’s power and flexibility, however, lie in its array of packages
		(currently more than 4,000!)
	\end{frame}
	%==============================================================================================%
	\begin{frame}
		\frametitle{1.1 Installing R}
		R is very easily installed by downloading it from the CRAN website. Installation usually takes
		about 2 minutes. When Installation of R is complete, the distinctive R Icon will appear on your
		desktop. To start R, simply click this Icon. It is common to re-install R once a year or so. The
		current version of R, version 3.0.0. was released quite recently.
	\end{frame}
	%==============================================================================================%
	\begin{frame}
		
		\frametitle{1.2 Command Line Interface}
		When you start R, the command line interface window will appear on screen. This is one
		of many windows in the R environment, others including graphical output windows, or script
		editors. R code can be entered into the command line directly. Alternatively code can be saved
		to a script, which can be run inside a session using the source() function.
	\end{frame}
	%==============================================================================================%
	\begin{frame}
		\frametitle{1.3 The Assignment operator}
		The assignment operator is used to assign names to particular values. Historically the assignment
		operator was ) a ”<-”. The assignment operator can also be =. This is valid as of R
		version 1.4.0.
		
		Both will be used, although, you should learn one and stick with it. Many long term R
		users prefer the arrow approach. 
	\end{frame}
	%==============================================================================================%
	\begin{frame}[fragile]
		\frametitle{1.3 The Assignment operator}
		
		You can also use -> as an assignment operator, reversing the
		usual assignment positions. (This is actually really useful). Commands are separated either by
		a semi colon or by a newline.
		\begin{framed}
		\begin{verbatim}
		> a <- 6
		> b = 5
		> a + b ->c
		> c
		[1] 11
		>e=7;f<-4
		\end{verbatim}
		\end{framed}
	\end{frame}
	%==============================================================================================%
	\begin{frame}[fragile]
		\frametitle{1.3 The Assignment operator}
		\begin{itemize}
		\item	Before we continue, try using the up and down keys, and see what happens. 
		\item Previously
			typed commands are re-presented, and can be re-executed.
		\end{itemize}
	
	\end{frame}
	%==============================================================================================%
	\begin{frame}[fragile]
		\frametitle{1.3 The Assignment operator }
		R stores both data and output from data analysis (as well as everything else) in objects.
		The variables we have created so far are objects. A list of all objects in the current session can
		be obtained with ls().
	\end{frame}
	
	%==============================================================================================%
	\begin{frame}
		
		1.3.1 Reserved Words
		Some names are used by the system, e.g.T, F,q,c etc . Avoid using these.
	\end{frame}
	%==============================================================================================%
	\begin{frame}[fragile]
		\frametitle{1.4 Commenting}
		For the sake of readability, it is good practice to comment code. The \# character at the
		beginning of a line signifies a comment, which is not executed. Lines of hashtags can be used
		to identify the beginning and end of code segments
		\begin{framed}
			\begin{verbatim}
			# This is a comment
			#####################
			# Start of Segment 1
			#####################
			\end{verbatim}
		\end{framed}
	\end{frame}
	%==============================================================================================%
	\begin{frame}
		\frametitle{1.5 Defining Variables}
		A convention is to use define a variable name with a capital letter (R is case sensitive). This
		reduces the chance of overwriting in-build R functions, which are usually written in lowercase
		letters. Commonly used variable names such as x,y and z (in lower case letters) are not “reserved”.
	\end{frame}
	%==============================================================================================%
	\begin{frame}
		
		\frametitle{1.6 Help Functions}
		Help files for R functions are accessed by preceding the name of the function with ? (e.g. ?sort
		). Alternatively you can use the command help() (e.g. help(sqrt) )
		A HTML document appears on screen with information on the function typed in. As well
		as the list of arguments that the particular function accepts, and how to specify them, there is
		example code at the bottom of the file. These code segments are often invaluable in learning
		how to master the function.
		
	\end{frame}
	%==============================================================================================%
	\begin{frame}
		
		
		\frametitle{1.7 The help.start() command}
		As mentioned by the text on the main console, the help.start() command can be usd to
		access detailed help documentation that comes as part of the installation.
	\end{frame}
	%==============================================================================================%
	\begin{frame}[fragile]
		
		\frametitle{1.8 Basic Maths Operations}
		The most commonly used mathematical operators are all supported by R. Here are a few
		examples:
		
		\begin{verbatim}
		5 + 3 * 5 # Note the order of operations.
		log (10) # Natural logarithm with base e=2.718282
		log(8,2) # Log to the base 2
		4^2 # 4 raised to the second power
		7/2 # Division
		factorial(4) #Factorial of Four
		sqrt (25) # Square root
		abs (3-7) # Absolute value of 3-7
		pi # The mysterious number \\\
		exp(2) # exponential function
		\end{verbatim}
	\end{frame}
	%==============================================================================================%
	\begin{frame}[fragile]
		
		\texttt{R} can be used for many mathematical operations, including
		
		\begin{itemize}
			\item Set Theory
			\item Trigonometry
			\item Complex Numbers
			\item Binomial Coefficients
		\end{itemize}
		Set Theory is always useful to know. We will not go into any of the others in great detail today.
	\end{frame}
	%==============================================================================================%
	\begin{frame}
		
		1.9 Basic R Editor
		R has an inbuilt script editor. We will use it for this class, but there are plenty of top quality
		Integrated Development Environments out there. (Read up about RStudio for example).
		For a while, we will use the in-built script editor. Although it is advisable to start using Rstudio or something similar.
		
	\end{frame}
	%==============================================================================================%
	\begin{frame}
		
		
		To start a new script, or open an existing script simply go to File and click the appropriate
		options. A new dialogue box will appear. You can write and edit code using this editor.
		To pass the code for compiling , press the run line or selection option (The third icon
		on the menu).
		
	\end{frame}
	%==============================================================================================%
	\begin{frame}
		
		Another way to read code is to use the edit() function , which operates directly from the
		command line. To see how the code defining an object X was written (or more importantly,
		could have been written) simply type \texttt{edit(X)}. This command has some useful applications
		that we will see later on.
		
	\end{frame}
	%==============================================================================================%
	\begin{frame}
		
		
		Scripts are saved as .R files. These scripts can be called directly using the \texttt{source()} command.
		
	\end{frame}
	%==============================================================================================%
	\begin{frame}
		\frametitle{1.10 Built-In Data Sets}
		Several data sets , intended as learning tools, are automatically installed when R is installed.
		Many more are installed within packages to complement learning to use those packages. One
		of these is the famous Iris data set, which is used in many data mining exercises.
		
		\begin{itemize}
			\item iris
			\item mtcars
			\item Nile
		\end{itemize}
		
		
	\end{frame}
	
	%==============================================================================================%
	\begin{frame}
		
		
		To see what data sets are available, simply type data(). To load a data set, simply type in the
		name of the data set. Some data sets are very large. To just see the first few (or last) rows, we
		use the head() function or alternatively the tail() function. The default number of rows of
		these commands is 6. Other numbers can be specified.
		
	\end{frame}
	%==============================================================================================%
	\begin{frame}[fragile]
		\begin{verbatim}
		> head(iris)
		Sepal.Length Sepal.Width Petal.Length Petal.Width Species
		1 5.1 3.5 1.4 0.2 setosa
		2 4.9 3.0 1.4 0.2 setosa
		3 4.7 3.2 1.3 0.2 setosa
		4 4.6 3.1 1.5 0.2 setosa
		5 5.0 3.6 1.4 0.2 setosa
		6 5.4 3.9 1.7 0.4 setosa
		>
		\end{verbatim}
		
	\end{frame}
	%==============================================================================================%
	\begin{frame}[fragile]
		\begin{verbatim}
		> tail(iris,4)
		Sepal.Length Sepal.Width Petal.Length Petal.Width Species
		147 6.3 2.5 5.0 1.9 virginica
		148 6.5 3.0 5.2 2.0 virginica
		149 6.2 3.4 5.4 2.3 virginica
		150 5.9 3.0 5.1 1.8 virginic
		\end{verbatim}
		
	\end{frame}
	%==============================================================================================%
	\begin{frame}
		In many situations, it is useful to call a particular data set using the attach() command. This
		will save having to specify the data sets over repeated operations. The file can then be detached
		using the detach() command.
		
		
	\end{frame}
	%==============================================================================================%
	\begin{frame}[fragile]
		\frametitle{1.11 The summary() command}
		The R command summary() is a generic function used to produce result “summaries” of the
		results of various objects, from simple vectors to the output of complex model fitting functions.
		The function invokes particular methods which depend on the class of the first argument.
		> summary(Nile)
		Min. 1st Qu. Median Mean 3rd Qu. Max.
		456.0 798.5 893.5 919.4 1032.0 1370.0
	\end{frame}
	%==============================================================================================%
	\begin{frame}[fragile]
		> summary(Indometh)
		Subject time conc
		1:11 Min. :0.250 Min. :0.0500
		4:11 1st Qu.:0.750 1st Qu.:0.1100
		2:11 Median :2.000 Median :0.3400
		5:11 Mean :2.886 Mean :0.5918
		6:11 3rd Qu.:5.000 3rd Qu.:0.8325
		3:11 Max. :8.000 Max. :2.7200
	\end{frame}
	%==============================================================================================%
	\begin{frame}[fragile]
		1.12 Working directories
		You can change your working directly by using the appropriate options on the File menu. To
		determine the current working directory, you can use the getwd() command. To change the
		working directory , we would use the setwd() command. This is quite important as objects
		will be imported and exported to and from the specified directory.
	\end{frame}
	%==============================================================================================%
	\begin{frame}[fragile]
		> getwd()
		[1] "C:/Users/Kevin"
		>
		> setwd("C:/Users/Kevin/Documents")
		>
		> getwd()
		[1] "C:/Users/Kevin/Documents"
	\end{frame}
	%==============================================================================================%
	\begin{frame}[fragile]
		1.13 Coming Unstuck
		If you are having trouble with a piece of code that is currently compiling , all you have to do
		is press ESC, just like many other computing environments.
	\end{frame}
	%==============================================================================================%
	\begin{frame}[fragile]
		1.14 Quitting the R environment
		As the front page text indicates, all you have to do to quite the workspace is to type in q().
		You will then be prompted to save your work.
	\end{frame}
	%==============================================================================================%
	\begin{frame}
		
		1.15 Data Objects
		As mentioned previously, R saves data as objects. Examples of data objects are
		• Vectors
		• Lists
		• Dataframes
		• Matrices
		The simple objects we have created previously are simply single element vectors.
	\end{frame}
	%==============================================================================================%
	\begin{frame}
		
		1.16 Listing all items in a workspace
		To list all items in an R environment, we use the ls() function. This provides a list of all data
		objects accessible. Another useful command is objects().
		> ls()
		[1] "a" "A" "authors" "b" "books"
		[6] "C" "D" "ex1" "Gerb" "Lst"
		[11] "m" "m1" "op" "presidents" "r"
		[16] "showSmooth" "sm" "sm.3RS" "sm2" "sm3"
		[21] "Trig" "Vec1" "x" "X" "x.at"
		[26] "x1" "x2" "x3R" "y" "Y"
		[31] "y.at"
	\end{frame}
	%==============================================================================================%
	\begin{frame}
		
		1.17 Removing items
		Sometimes it is desirable to save a subset of your workspace instead of the entire workspace.
		One option is to use the rm() function to remove unwanted objects right before exiting your R
		session; another possibility is to use the save() function.
		
	\end{frame}
	%==============================================================================================%
	\begin{frame}
		1.18 Saving and Loading R Data Objects
		In situations where a good deal of processing must be used on a raw dataset in order to prepare
		it for analysis, it may be prudent to save the R objects you create in their internal binary form.
		One attractive feature of this scheme is that the objects created can be read by R programs
		running on different computer architectures than the one on which they were created, making it
		very easy to move your data between different computers. Each time an R session is completed,
		you are prompted to save the workspace image, which is a binary file called .RData in the
		working directory.
	\end{frame}
	%==============================================================================================%
	\begin{frame}
		Whenever R encounters such a file in the working directory at the beginning of a session,
		it automatically loads it making all your saved objects available again. So one method for
		
		saving your work is to always save your workspace image at the end of an R session. If you
		would like to save your workspace image at some other time during your R session, you can use
		the save.image() function, which, when called with no arguments, will also save the current
		workspace to a file called .RData in the working directory.

	\end{frame}
	%==============================================================================================%
	\begin{frame}
				2 Introduction to R (Continued)
		2.1 Three particularly useful commands
		1. help()
		2. summary()
		3. help.start()
		
	\end{frame}
	%==============================================================================================%
	\begin{frame}
		
		2.2 Changing GUI options
		We can change the GUI options using the GUI preferences option on the Edit menu. (Important
		when teaching R) A demonstration will be done in class.
	\end{frame}
	%==============================================================================================%
	\begin{frame}
		
		2.3 Colours
		R supported a massive number of colours. Type in colours() (or colors()) to see what colours
		are supported.
		2.4 Use of the Semi-Colon Operator
		The semi-colon operator at the end of each line of code is not necessary in general, but using it
		overcomes errors due to copying and pasting from document soft copies. In other programming
		languages, such as Octave, using the semicolon in this way has a distinct purpose.
	\end{frame}
	%==============================================================================================%
	\begin{frame}
		2.5 The apropos() Function
		This function is very useful for determining what functions are available for a particular topic,
		although the process requires a great deal of trial and error. Suppose we are looking for a
		command to compute the correlation coefficient. We would use a very short string (e.g. cor)
		that would plausibly be part of useful function names.
		apropos("cor")
	\end{frame}
	%==============================================================================================%
	\begin{frame}
		
		2.6 History
		The command history() is used to obtain the last 25 commands used by R
		history()

\end{frame}
%==============================================================================================%
\begin{frame}
	
		2.7 The sessionInfo() Function
		To get a description of the version of R and its attached packages used in the current session,
		we can use the sessionInfo() function
		sessionInfo()
	\end{frame}
	%==============================================================================================%
	\begin{frame}[fragile]
		
		2.8 Time and date functions
		The commands Sys.time() and Sys.Date() returns the system’s idea of the current date
		with and without time. We can perform some simple algebraic calculations to compute time
		differences (i.e. to find out how long some code took to compile.
		
	\end{frame}
	%==============================================================================================%
	\begin{frame}[fragile]
				
		\begin{verbatim}
		> X1=Sys.time()
		> #Wait a few seconds
		>
		> X2=Sys.time()
		> X2-X1 Time difference of 8.439614 secs
		>
		> Sys.Date() [1] "2012-09-01"
		\end{verbatim}
		\end{frame}
		%==============================================================================================%
		\begin{frame}	
		\frametitle{2.9 Logical States}
		Logical states TRUE and FALSE are simply specified as such, all in capital letters. The
		shortcuts T and F are also recognized
		
	\end{frame}
	%==============================================================================================%
	\begin{frame}
		
		
		2.10 Missing Data
		In some cases the entire contents of a vector may not be known. For example, missing data
		from a particular data set. A place can be reserved for this by assigning it the special value
		NA.
		NA is a logical constant of length 1 which contains a missing value indicator. NA stands
		for Not Available.
		
	\end{frame}
	%==============================================================================================%
	\begin{frame}[fragile]
		
\frametitle{2.11 Files in the Working Directory}
		It is possibel to call an R script from the working directory, using the \texttt{source()} function.
			\begin{framed}
				\begin{verbatim}
		source("myfunctions.r")
		source("mydata.r")
			\end{verbatim}
		\end{framed}	
		We can also send code put directly to a file in the working directory, using the sink()
		command. The first time the command is used, the name of the created file is specified.
		Subsequent commands print output directly to this file, until the command is used again to
		cease the operation.
	\end{frame}
	%==============================================================================================%
	\begin{frame}[fragile]
	\begin{framed}
	\begin{verbatim}
		> sink("IrisSum.txt")
		> summary(iris)
		> sink()
		>
	\end{verbatim}
	\end{framed}	
	\end{frame}

%==============================================================================================%
\begin{frame}
	\frametitle{3 Inspecting a Data Set}
	\begin{itemize}
		\item dim()
		\item nrow() and ncol()
		\item names()
		\item summary()
		\item tail()
		\item head()
		\item describe() (from the psych package)
	\end{itemize}
\end{frame}
%==============================================================================================%
\begin{frame}
	
	\frametitle{3.1 Dimensions of a data set}
	We have remarked that some data sets are very large. This is perhaps a good place to consider
	summary information about data objects. For a simple vector, a useful command to determine
	the length (remark: sample size) is the function length().
	> Y=4:18
	> length(Y)
	[1] 15
	For more complex data sets ( and data frames which we will see later) , we have several
	tools for assessing the size of data.
\end{frame}
%==============================================================================================%
\begin{frame}[fragile]
	\begin{verbatim}
	> dim(iris) # dimensions of data set
	[1] 150 5
	> nrow(iris) # number of rows
	[1] 150
	> ncol(iris) # number of columns
	[1] 5
	\end{verbatim}
	
\end{frame}
%==============================================================================================%
\begin{frame}[fragile]
	We can also determine the row names and column names using the functions rownames()
	and colnames(). If there are no specific row or column names, the command will just return
	the indices.
	> colnames(iris)
	[1] "Sepal.Length" "Sepal.Width" "Petal.Length" "Petal.Width" "Species"
	
\end{frame}
%==============================================================================================%
\begin{frame}
	\frametitle{3.2 The summary() command}
	\begin{itemize}
		\item The command summary() is one of the most useful commands in R. 
		\item It is a generic function used
		to produce result summaries of the results of various functions. 
		\item The function invokes particular
		methods which depend on the class of the first argument. 
		\item In other words, R picks out the most
		suitable type of summary for that data.
	\end{itemize}
\end{frame}
%==============================================================================================%
\begin{frame}[fragile]
	> summary(iris)
	Sepal.Length Sepal.Width Petal.Length Petal.Width Species
	Min. :4.300 Min. :2.000 Min. :1.000 Min. :0.100 setosa :50
	1st Qu.:5.100 1st Qu.:2.800 1st Qu.:1.600 1st Qu.:0.300 versicolor:50
	Median :5.800 Median :3.000 Median :4.350 Median :1.300 virginica :50
	Mean :5.843 Mean :3.057 Mean :3.758 Mean :1.199
	3rd Qu.:6.400 3rd Qu.:3.300 3rd Qu.:5.100 3rd Qu.:1.800
	Max. :7.900 Max. :4.400 Max. :6.900 Max. :2.500
	>
	Summary is particularly useful for manipulating data from more complex data objects.
\end{frame}
%==============================================================================================%
\begin{frame}
	
	\frametitle{3.3 Structure of a Data Object}
	The structure, class and storage mode of an object can be determined using the following
	commands. Try out a few.
	\begin{itemize}
		\item  str()
		\item  class()
		\item  mode()
	\end{itemize}
	
	
\end{frame}
%==============================================================================================%
\begin{frame}[fragile]
	\begin{framed}
		\begin{verbatim}
		> class(Nile)
		[1] "ts"
		> mode(Nile)
		[1] "numeric"
		>
		
		\end{verbatim}
	\end{framed}
	
\end{frame}
%==============================================================================================%
\begin{frame}[fragile]
	\begin{verbatim}
	> a
	[1] 6
	> mode(a)
	[1] "numeric"
	> class(a)
	[1] "numeric"
	> str(a)
	num 6
	>
	> class(iris)
	[1] "data.frame"
	> mode(iris)
	[1] "list"
	\end{verbatim}
	
\end{frame}
%==============================================================================================%
\begin{frame}
	4 Packages
	4.1 Packages
	A Package in R is a file containing a collection of objects which have some common purpose.
	Packages enhance the capabilties and scope for research in a certain field. For example the
	package MASS contains objects associated with the Venables and Ripleys ”Modern Applied
	Statistics with S”. Some examples of packages are Actuar, written for actuarial science, and
	QRMlib, which complements the Quantitative Risk Management The command library()
	lists all the available packages. 
	
	To load a particular package, for example MASS, we would
	write
	library(MASS)
	
\end{frame}
	%=================================================================== %
	\begin{frame}[fragile]
		
		\frametitle{Packages}
		\begin{itemize}
			\item The CRAN package repository features 6107 available packages. 
			\item Packages contain
			various functions and data sets for numerous purposes, e.g.
			\textbf{\textit{ggplot2}} package, \textbf{\textit{AER}} package, \textbf{\textit{survival}} package, etc.
			\item Some packages are part of the basic installation. Others can be
			downloaded from CRAN.
			\item To access all of the functions and data sets in a particular package,
			it must be loaded into the workspace. 
			\item For example, to load the
			\textbf{\textit{ggplot2}} package:
		\end{itemize}
		\begin{framed}
			\begin{verbatim}
			install.packages(ggplot2)
			library(ggplot2)
			\end{verbatim}
		\end{framed}
	\end{frame}
%==============================================================================================%
\begin{frame}
\frametitle{4.2 Using and Installing packages}
\begin{itemize}
\item Many packages come with R. To use them in an R session, you need to load the package, as
	previously demonstrated.
\item Some packages are not automatically installed when you install R but they need to be downloaded
	and installed individually. 
\item We must first install them using the install.packages()
	function, which typically downloads the package from CRAN and installs it for use. (follow the
	instructions as indicated).
\end{itemize}
\end{frame}
%==============================================================================================%
\begin{frame}[fragile]
	\begin{framed}
		\begin{verbatim}
		install.packages("ggplot2")
		install.packages("qcc")
		install.packages("sqldf")
		install.packages("RMongo")
		install.packages("randomforest")
		\end{verbatim}
	\end{framed}
	
\end{frame}
%==============================================================================================%
\begin{frame}
\frametitle{4.2.1 Version of R}
	Many packages will require you to have the most recent version of R and also other packages.
	It is a good idea to update regularly.
\end{frame}
%==============================================================================================%
\begin{frame}
	5 Data Creation, Data Entry, Data Import and Export
\end{frame}
%==============================================================================================%
\begin{frame}[fragile]
\frametitle{5.1 The \texttt{c()} command}
	To create a simple data set, known as a vector, we use the c() command to create the vector.
\begin{framed}
\begin{verbatim}
	> Y=c(1,2,4,8,16 ) #create a data vector with specified elements
	> Y
	[1] 1 2 4 8 16
	5.1.1 Vector of Numeric Values
	Numvec = c(10,13,15,19,25);
	5.1.2 Vector of Character Values
	Charvec = c("LouLou","Oscar","Rasher");
\end{verbatim}
\end{framed}
	
\end{frame}
%==============================================================================================%
\begin{frame}
	5.1.3 Vector of Logical Values
	Charvec = c(TRUE,TRUE,FALSE,TRUE);
	
	Vectors can be bound together either by row or by column.
	> X=1:3; Y=4:6
	> cbind(X,Y)
	X Y
	[1,] 1 4
	[2,] 2 5
	[3,] 3 6
	>
	> rbind(X,Y)
	[,1] [,2] [,3]
	X 1 2 3
	Y 4 5 6
\end{frame}
%==============================================================================================%
\begin{frame}
\frametitle{5.2 The scan() command}
\begin{itemize}
\item	The scan() function is a useful method of inputting data quickly. 
\item You can use to quickly copy
	and paste values into the R environment. It is best used in the manner as described in the
	following example. 
\item Create a variable ”X” and use the \texttt{scan()} function to populate it with
	values. 
\item Type in a value, and then press return. Once you have entered all the values, press
	return again to return to normal operation.
\end{itemize}
\end{frame}
%==============================================================================================%
\begin{frame}[fragile]
	> X=scan()
	1: 4
	2: 5
	3: 5
	4: 6
	5:
	Read 4 items
	Remark: Try the edit() command on object X.
\end{frame}
%==============================================================================================%
\begin{frame}[fragile]
	5.2.1 Characters with the scan() command
	The scan() command can also be used forinputting character data.
	> Y=scan(what=" ")
	1: LouLou
	2: Oscar
	3: Rasher
	4:
	Read 3 items
	> Y
	[1] "LouLou" "Oscar" "Rasher"
\end{frame}
%==============================================================================================%
\begin{frame}[fragile]
	5.3 Using the data editor
	5.4 Spreadsheet Interface
	R provides a spreadsheet interface for editing the values of existing data sets. We use the
	command \texttt{data.entry()}, and name of the data object as the argument. (Also try out the
	fix() command)
	\begin{framed}
	\begin{verbatim}
		> data.entry(X) # Edit the data set and exit interface
		> X
	\end{verbatim}
	\end{framed}

\end{frame}





\end{document}
