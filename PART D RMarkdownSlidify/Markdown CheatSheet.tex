
\documentclass[]{article}
\voffset=-1.5cm
\oddsidemargin=0.0cm
\textwidth = 470pt
\usepackage[utf8]{inputenc}
\usepackage[english]{babel}
\usepackage{framed}
\usepackage{graphicx}
\usepackage{enumerate}% http://ctan.org/pkg/enumerate
\usepackage{multicol}
\usepackage{amsmath}
\usepackage{amssymb}
\begin{document}
	\subsection*{Headers}
\begin{framed}
\begin{verbatim}
# Header 1 #
## Header 2 ##
### Header 3 ###             (Hashes on right are optional)
#### Header 4 ####
##### Header 5 #####
\end{verbatim}
\end{framed}

\begin{framed}
\begin{verbatim}
## Markdown plus h2 with a custom ID ##         {#id-goes-here}
[Link back to H2](#id-goes-here)

This is a paragraph, which is text surrounded by whitespace. 
Paragraphs can be on one 
line (or many), and can drone on for hours.  
\end{verbatim}
\end{framed}

\subsection*{Links}
\begin{framed}
	\begin{verbatim}
Here is a Markdown link to [Warped](h\begin{framed}
\begin{verbatim}
ttp://warpedvisions.org), and a literal . 
Now some SimpleLinks, like one to [google] 

(automagically links to are-you-
feeling-lucky), 

a [wiki: test] link to a Wikipedia page, and a link to 
[foldoc: CPU]s at foldoc.  

Now some inline markup like _italics_,  **bold**, and `code()`. 
Note that underscores in 
words are ignored in Markdown Extra.
\end{verbatim}
\end{framed}
\newpage
\begin{framed}
	\begin{verbatim}
	
![picture alt](/images/photo.jpeg "Title is optional")     

> Blockquotes are like quoted text in email replies
>> And, they can be nested

\end{verbatim}
\end{framed}
\subsection*{Lists}
\begin{framed}
	\begin{verbatim}
	
* Bullet lists are easy too
- Another one
+ Another one

1. A numbered list
2. Which is numbered
3. With periods and a space
\end{verbatim}
\end{framed}
And now some code:
\begin{framed}
	\begin{verbatim}
	


// Code is just text indented a bit
which(is_easy) to_remember();

~~~
\end{verbatim}
\end{framed}
\begin{framed}
	\begin{verbatim}
// Markdown extra adds un-indented code blocks too

if (this_is_more_code == true && !indented) {
	// tild wrapped code blocks, also not indented
}

~~~

Text with  
two trailing spaces  
(on the right)  
can be used  
for things like poems  

### Horizontal rules

* * * *
****
--------------------------
\end{verbatim}
\end{framed}

\begin{framed}
\begin{verbatim}
<div class="custom-class" markdown="1">
This is a div wrapping some Markdown plus.  
Without the DIV attribute, it ignores the block. 
</div>

## Markdown plus tables ##

| Header | Header | Right  |
| ------ | ------ | -----: |
|  Cell  |  Cell  |   $10  |
|  Cell  |  Cell  |   $20  |

* Outer pipes on tables are optional
* Colon used for alignment (right versus left)
\end{verbatim}
\end{framed}
\newpage
\begin{framed}
	\begin{verbatim}
## Markdown plus definition lists ##

Bottled water
: $ 1.25
: $ 1.55 (Large)

Milk
Pop
: $ 1.75

* Multiple definitions and terms are possible
* Definitions can include multiple paragraphs too

*[ABBR]: Markdown plus abbreviations (produces an <abbr> tag
\end{verbatim}
\end{framed}
\end{document}
